\chapter{El Problema}

	\section{Planteamiento del problema}	
El color y la apariencia de la piel humana son importantes en el campo de la medicina. Durante el diagn\'{o}stico de enfermedades de la piel, la observaci\'{o}n cuidadosa y la evaluaci\'{o}n visual del \'{a}rea afectada es siempre el primer paso, y el m\'{a}s importante. Esto es seguido generalmente por una escisi\'{o}n o biopsia por punci\'{o}n, en la que se extrae una muestra de tejido de la piel para un an\'{a}lisis microsc\'{o}pico. La observaci\'{o}n visual suele ser subjetiva, y los pacientes a menudo se someten a cicatrices y dolor durante la biopsia. Por otro lado, las t\'{e}cnicas \'{o}pticas son por lo general no invasivas y sus resultados son a menudo objetivos. Durante el diagn\'{o}stico no invasivo, no se crea ninguna ruptura en la piel, y los pacientes no se someten al dolor ni a cicatrices durante el tratamiento (Bersha, 2010).

Los avances tecnol\'{o}gicos de la actualidad permiten emplear t\'{e}cnicas de \'{o}ptica, capaces de estudiar  las propiedades estructurales y bioqu\'{i}micas del tejido biol\'{o}gico de manera precisa y no invasiva. Los instrumentos que emplean tales t\'{e}cnicas son de gran ayuda para los m\'{e}dicos dermat\'{o}logos, raz\'{o}n por la cual dichos instrumentos han tomado suma importancia en el \'{a}rea de la medicina dermatol\'{o}gica.

Hoy en d\'{i}a existen diferentes tipos de estudios \'{o}pticos in-situ, in-vivo e invitro del tejido biol\'{o}gico, como lo es la espectroscop\'{i}a de reflectancia difusa (ERD). \cite{Perez} asegura que con esta t\'{e}cnica es  posible estudiar las propiedades bioqu\'{i}micas y las condiciones estructurales de un tejido biol\'{o}gico, analizando la interacci\'{o}n luz-tejido de una manera no invasiva.

En este sentido, el Centro de Investigaciones M\'{e}dicas y Biotecnol\'{o}gicas de la Universidad de Carabobo (CIMBUC) dispone de un espectrofot\'{o}metro de reflexi\'{o}n difusa, denominado MiniScan XE Plus. Este dispositivo fue creado por la empresa HunterLab, la cual lo describe como un instrumento que aplica la t\'{e}cnica de ERD, utilizado para medir la reflectancia de espec\'{i}menes como una funci\'{o}n de longitud de onda.

Ahora bien, el CIMBUC hace uso de este instrumento a trav\'{e}s del \'{u}nico software disponible para su utilizaci\'{o}n, designado HunterLab Universal Software. \'{E}ste es un software comercial y privativo capaz de ejecutarse en sistemas operativos Windows, desde la versi\'{o}n 95 hasta la versi\'{o}n XP. Dicho software contiene funciones que abarcan la utilizaci\'{o}n del MiniScan XE Plus y de otros instrumentos ofrecidos por HunterLab; su interfaz gr\'{a}fica de usuario est\'{a} en ingl\'{e}s, adem\'{a}s de que los resultados que genera no poseen el formato de gesti\'{o}n de informaci\'{o}n de pacientes con el que trabajan los dermat\'{o}logos del CIMBUC. Por \'{u}ltimo, este software fue descontinuado en el a\~{n}o 2008.

Tomando en cuenta lo mencionado anteriormente, se tiene que el HunterLab Universal Software es privativo y est\'{a} descontinuado, por lo tanto no existe la posibilidad de modificarlo ni de extenderlo. Dicho software est\'{a} en ingl\'{e}s y ofrece funciones ajenas al uso exclusivo del MiniScan XE Plus, causando que su interfaz gr\'{a}fica de usuario contenga m\'{a}s opciones de las necesarias para manejar tal instrumento, y que sea dif\'{i}cil de entender por los dermat\'{o}logos. Todo esto sumado al hecho de que los resultados generados por este software deben exportarse y almacenarse manualmente, porque los mismos no poseen el formato con el que trabajan los dermat\'{o}logos, produce la necesidad de contar con asistencia t\'{e}cnica especializada para su debida utilizaci\'{o}n y ralentiza las consultas con los pacientes.

De lo antedicho se desprende que el HunterLab Universal Software posee una interfaz gr\'{a}fica de usuario poco amigable, y el costo del tiempo de capacitaci\'{o}n para su uso correcto podr\'{i}a ser alto. Dicho software no podr\'{a} modificarse ni extenderse, por lo tanto no se fomentar\'{a} el uso del instrumento en cuesti\'{o}n, disminuyendo su potencial. El formato de los resultados de este software convertir\'{a} las consultas de los dermat\'{o}logos con los pacientes en una labor ineficiente en t\'{e}rminos de tiempo. Por \'{u}ltimo, no se fomentar\'{a} el desarrollo de nuevas funciones que utilicen los resultados de dicho software como insumo, sosegando as\'{i} la posibilidad de realizar an\'{a}lisis m\'{a}s complejos, y de proveer resultados que les permitan a los dermat\'{o}logos establecer diagn\'{o}sticos m\'{a}s completos sobre patolog\'{i}as dermatol\'{o}gicas en pacientes.

Motivado a lo anterior, se desarroll\'{o} un software modificable y extensible, con una interfaz gr\'{a}fica de usuario amigable, con funciones para el uso exclusivo del MiniScan XE Plus, que genera resultados relevantes para los dermat\'{o}logos empleando el formato utilizado por ellos para registrar las consultas con los pacientes, y siguiendo los lineamientos de la ingenier\'{i}a del software pertinentes.

Finalmente, con esta investigaci\'{o}n se espera fomentar la utilizaci\'{o}n del \mbox{MiniScan} XE Plus, reducir el tiempo de las consultas con los pacientes, aportar un software sobre el que se puedan desarrollar nuevas investigaciones que conlleven a an\'{a}lisis m\'{a}s complejos, y por \'{u}ltimo, realizar diagn\'{o}sticos m\'{a}s completos y diversos sobre patolog\'{i}as dermatol\'{o}gicas presentes en pacientes.

	\section{Objetivos}

		\subsection{Objetivo general}
	Desarrollar un software para el espectrofot\'{o}metro MiniScan XE Plus, usado en el diagn\'{o}stico de patolog\'{i}as dermatol\'{o}gicas en pacientes, tomando como caso de estudio el CIMBUC.
		\subsection{Objetivos espec\'{i}ficos}
			\begin{itemize}
				\item Investigar el estado del arte relacionado a la investigaci\'{o}n: t\'{e}cnicas de \'{o}ptica y colorimetr\'{i}a, MiniScan XE Plus, HunterLab Universal Software y atributos de calidad del software.
				
				\item Seleccionar una metodolog\'{i}a de investigaci\'{o}n y una metodolog\'{i}a de desarrollo para el nuevo software.
				
				\item Desarrollar el software, siguiendo las metodolog\'{i}as \mbox{seleccionadas}.
				
				\item Realizar las pruebas para el nuevo software.
				
				\item Elaborar los manuales de usuario y de instalaci\'{o}n para el uso del software.
			\end{itemize}
\newpage
	\section{Justificaci\'{o}n de la investigaci\'{o}n}
	
	Empezando con la interfaz gr\'{a}fica de usuario, \cite{Sommerville} se\~{n}ala que el dise\~{n}o cuidadoso de la misma es una parte fundamental del proceso de dise\~{n}o general del software. Si un software debe alcanzar su potencial m\'{a}ximo, es fundamental que su interfaz gr\'{a}fica de usuario sea dise\~{n}ada para ajustarse a las habilidades, experiencia y expectativas de sus usuarios previstos. Un buen dise\~{n}o de la interfaz gr\'{a}fica de usuario es cr\'{i}tico para la confiabilidad del software. Muchos de los llamados errores de usuario son causados porque las interfaces gr\'{a}ficas de usuario no consideran las habilidades de los usuarios reales y su entorno de trabajo.

	Dicho lo anterior, el dise\~{n}o de la interfaz gr\'{a}fica de usuario del HunterLab Universal Software es la principal raz\'{o}n por la cual los dermat\'{o}logos requieren de personal t\'{e}cnico especializado que los asista al momento de utilizarlo. Esto porque dicha interfaz est\'{a} en ingl\'{e}s, ofrece funciones que no son necesarias para la utilizaci\'{o}n del MiniScan XE Plus y sus resultados no poseen un formato adecuado. Por estas razones los dermat\'{o}logos perciben este software comercial como no intuitivo, ni auto descriptivo ni amigable, temiendo cometer errores al utilizarlo sin asistencia y generar resultados err\'{o}neos, poniendo en riesgo la fiabilidad del diagn\'{o}stico y, en consecuencia, la salud de los pacientes.

	Con respecto a software de calidad, \cite{Sommerville} explica lo siguiente: as\'{i} como los servicios que proveen, los productos de software tienen un cierto n\'{u}mero de atributos asociados que reflejan su calidad. Estos atributos no est\'{a}n directamente relacionados con lo que hace el software; m\'{a}s bien, reflejan su comportamiento durante su ejecuci\'{o}n, en la estructura y la organizaci\'{o}n del programa fuente y en la documentaci\'{o}n asociada.

El conjunto espec\'{i}fico de atributos que se esperan de un software de calidad depende obviamente de su aplicaci\'{o}n. En la tabla 1.1 se puede apreciar la generalizaci\'{o}n de estos atributos.

		\begin{table}[t]
			\small
			\caption[Atributos esenciales de un software de calidad]{\textit{Atributos esenciales de un software de calidad} (Fuente: Sommerville, 2005).}
			\centering
			\setlength{\extrarowheight}{\altocelda}
			\begin{tabulary}{\anchotabla}{|c|J|}
				
				\hline
				\thead{\textbf{\small{Caracter\'{i}stica}}} & \thead{\textbf{\small{Descripci\'{o}n}}}\\ \hline
			\textbf{Mantenibilidad} & El software debe describirse de tal forma que pueda evolucionar  para cumplir las necesidades de cambio de los 					clientes. \'{E}ste es un atributo cr\'{i}tico, debido a que el cambio en el software es una consecuencia inevitable de un cambio en el entorno de negocios.\\ \hline
			\textbf{Confiabilidad} & Este atributo tiene un gran n\'{u}mero de caracter\'{i}sticas, incluyendo la fiabilidad, la protecci\'{o}n y la seguridad. El software confiable no debe causar da\~{n}os f\'{i}sicos o econ\'{o}micos en caso de que ocurra una falla del sistema.\\ \hline
			\textbf{Eficiencia} & El software no debe hacer que se malgasten los recursos del sistema, como la memoria y los ciclos de procesamiento. Por lo tanto, la eficiencia incluye tiempos de respuesta y de procesamiento, utilizaci\'{o}n de la memoria, etc\'{e}tera.\\ \hline
			\textbf{Usabilidad} & El software debe ser f\'{a}cil de utilizar, sin esfuerzo adicional por parte del usuario para quien est\'{a} dise\~{n}ado. Esto significa que debe tener una interfaz gr\'{a}fica de usuario apropiada, y una documentaci\'{o}n adecuada.\\ \hline
			\end{tabulary}
		\end{table}

Debido a que el HunterLab Universal Software es privativo, el CIMBUC no dispone de su c\'{o}digo fuente, de manera que este software no puede modificarse ni adaptarse a necesidades espec\'{i}ficas, y por lo tanto, no posee el primer atributo esencial para un software de calidad: la mantenibilidad. Por la misma raz\'{o}n, no se puede determinar con certidumbre el segundo atributo: la confiabilidad. Por \'{u}ltimo, la usabilidad de este software es baja, ya que la interfaz gr\'{a}fica de usuario es poco amigable.

Las razones descritas anteriormente justifican la necesidad del desarrollo de un software para el uso del MiniScan XE Plus que sea amigable, modificable, extensible y que cumpla con los atributos esenciales de calidad; que emplee el formato de historia m\'{e}dica con el que trabajan dermat\'{o}logos, y que ofrezca las funciones que ellos necesitan para realizar an\'{a}lisis y establecer diagn\'{o}sticos de patolog\'{i}as dermatol\'{o}gicas en pacientes. Por \'{u}ltimo, se ha tomado como caso de estudio el CIMBUC.