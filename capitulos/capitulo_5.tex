\chapter{Conclusiones y recomendaciones}

\section{Conclusiones}

	La elaboraci\'{o}n de la presente investigaci\'{o}n ha cumplido los objetivos planteados. Para lograr esto fue necesario un an\'{a}lisis detallado de la bibliograf\'{i}a, la revisi\'{o}n de algunas investigaciones previas, y el estudio riguroso del material proporcionado por el personal de soporte t\'{e}cnico de la empresa HunterLab.
	
	El aporte general de este trabajo de investigaci\'{o}n se centra en proveer un software libre para operar el MiniScan XE Plus, el cual dispone de las funciones necesarias para que los dermat\'{o}logos puedan establecer diagn\'{o}sticos de patolog\'{i}as dermatol\'{o}gicas en pacientes.
	
	Desde el comienzo del proceso de desarrollo del software se opt\'{o} por trabajar con un servicio gratuito de control de versiones, lo cual permiti\'{o} tener almacenado el c\'{o}digo fuente del software de manera centralizada. Esto ayud\'{o} a tener una mejor organizaci\'{o}n durante el desarrollo.
	
	Debido a las limitaciones encontradas durante la investigaci\'{o}n, se concluye que es necesaria la utilizaci\'{o}n de algunos archivos de HunterLab para lograr la comunicaci\'{o}n entre el software resultante y el MiniSan XE Plus. Adicionalmente, debido a esta limitaci\'{o}n el software resultante no puede captar ni interpretar las se\~{n}ales de los botones del MiniScan XE Plus, ya que estos archivos no ofrecen esta caracter\'{i}stica para ser utilizada fuera del HunterLab Universal Software.
	
	La instalaci\'{o}n de este software no es tan simple como podr\'{i}a llegar a ser, como consecuencia de la necesidad de utilizar algunos archivos del HunterLab. El software resultante no puede habilitarse para ser multiplataforma debido a esta raz\'{o}n.

	Durante las pruebas de funcionalidad y usabilidad realizadas al software se hizo notorio el nivel de aceptaci\'{o}n y la satisfacci\'{o}n de los clientes a los que iba dirigido. Adicionalmente, se realiz\'{o} un manual de usuario para la correcta utilizaci\'{o}n del software, que explica detalladamente con tablas y con im\'{a}genes la permisolog\'{i}a de sus usuarios y las funciones que ofrece, por lo que su curva de aprendizaje es baja.
	
	Durante el transcurso del presente trabajo de investigaci\'{o}n ocurri\'{o} un intercambio multidisciplinario, debido a que se utiliz\'{o} la computaci\'{o}n para implementar teor\'{i}as de la f\'{i}sica, \'{o}ptica y colorimetr\'{i}a, y as\'{i} resolver un problema en el \'{a}rea de la medicina dermatol\'{o}gica.

	En definitiva, se concluye que el software resultante cumple con todos los objetivos establecidos en esta investigaci\'{o}n, ajustandose a las necesidades de los dermat\'{o}logos, garantizando un mejor aprovechamiento del MiniScan XE Plus, creando una base sobre la cual se pueden realizar trabajos futuros que modifiquen, mejoren y extiendan dicho software.

\newpage

\section{Recomendaciones}

	De acuerdo a las conclusiones alcanzadas con el trabajo, se han generado una serie de recomendaciones, las cuales se presentan a continuaci\'{o}n.

\begin{itemize}

	\item Permitir la visualizaci\'{o}n, consulta y exportaci\'{o}n de varias muestras al mismo tiempo.
	
	\item Desarrollar un arhivo controlador para el MiniScan XE Plus que no dependa del kit MSXE.ocx del HunterLab, para as\'{i} simplificar el proceso de instalaci\'{o}n del Spectrasoft, lograr que el mismo sea capaz de captar e interpretar las se\~{n}ales de los botones del MiniScan XE Plus, y posibilitar que \'{e}ste sea multiplataforma.
	
	\item Incluir la f\'{o}rmula desarrollada por los investigadores del CIMBUC para el c\'{a}lculo del coeficiente de esparcimiento de la epidermis, la cual no estuvo lista durante el tiempo en el que se llev\'{o} a cabo este trabajo de investigaci\'{o}n.
	
	\item Crear una nueva base de datos que habilite la gesti\'{o}n de muestras experimentales para los investigadores del CIMBUC.
	
	\item Adaptar el Spectrasoft a una arquitectura de modelo cliente-servidor para permitir la conexi\'{o}n con bases de datos remotas en sistemas distribuidos.
	
	\item Disponer de un ambiente \textit{QA} para realizar pruebas de rendimiento y de base de datos al Spectrasoft, previas a su puesta en producci\'{o}n.
	
	\item Incluir la carga de muestras por archivo al Spectrasoft.
\end{itemize}