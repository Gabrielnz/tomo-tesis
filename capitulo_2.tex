\chapter{\label{cap:2}Marco Te\'{o}rico}

	\section{Antecedentes}	
		\begin{itemize}
		
			\item
				La sensaci\'{o}n de la luz es producida por radiaci\'{o}n electromagn\'{e}tica visible, que est\'{a} dentro de los l\'{i}mites de longitud de onda de 380 a 780 nan\'{o}metros. Dentro de este rango, la radiaci\'{o}n electromagn\'{e}tica produce la sensaci\'{o}n de luz azul, luz verde, y luz roja; estas sensaciones son denominadas valores triest\'{i}mulo. Una de las maneras de producir un color de forma digital, es mezclando estos valores. La \textit{\mbox{Commission} Internationale de l'Eclairage} \textit{(CIE)}, defini\'{o} en 1964 un est\'{a}ndar para calcular los valores triest\'{i}mulo de un color, representandolos como coordenadas de cromaticidad \textit{XYZ}, lo que es mejor conocido como el sistema tricrom\'{a}tico \textit{CIE 1964}. Ahora bien, el espacio \textit{CIE L*a*b*}, es un sistema definido por la misma comisi\'{o}n en 1976 para la \mbox{transformaci\'{o}n} de las coordenadas de cromaticidad mencionadas, a unas coordenadas representables en un espacio de tres dimensiones. \cite{Schanda} proporciona las f\'{o}rmulas utilizadas para el c\'{a}lculo de las coordenadas de cromaticidad y de las coordenadas resultantes de este espacio, las cuales fueron implementadas en el nuevo software como funciones para determinar ciertas propiedades \'{o}pticas, presentes en la piel de los pacientes.
			
			\item La melanina que se encuentra distribuida en la epidermis es el  principal agente absorbente de la piel, y, por lo tanto, determina en gran parte el color de la misma \cite{Narea}. La t\'{e}cnica empleada para recuperar el coeficiente de absorci\'{o}n en la epidermis de la piel humana \cite{Narea} est\'{a} fue implementada en el nuevo software, para determinar el nivel de concentraci\'{o}n de melanina en la piel de los pacientes.
			
			\item En el art\'{i}culo titulado ``Recuperaci\'{o}n del Coeficiente de Absorci\'{o}n de la Epidermis en la Piel Humana'' de \cite{Narea}, se obtiene el coeficiente de absorci\'{o}n de la epidermis en la piel humana a partir de datos espectrales. El procedimiento para obtener dicho coeficiente es implementado en el nuevo software para determinar el nivel de concentraci\'{o}n de melanina en la epidermis de un paciente.
			
			\item En la tesis de maestr\'{i}a de \cite{Bersha} titulada ``Spectral Imaging and Analysis of Human Skin'', se calcula el \'{i}ndice de eritema partiendo de la coordenada \textit{a*}, correspondiente al espacio de color \textit{CIELAB}. El m\'{e}todo para la obtenci\'{o}n del \'{i}ndice mencionado es implementado en el nuevo software, para determinar el nivel inflamatorio de la epidermis de un paciente.
		\end{itemize}

	\section{Observaci\'{o}n Directa}
		\begin{itemize}
			\item \textbf{HunterLab Universal Software:} Es un software propietario de 16-bit dise\~{n}ado para el Sistema Operativo Microsoft Windows Version 3.x, con la posibilidad de ejecutarse en Windows 95, Windows 2000, Windows NT y Windows XP, descontinuado en el a\~{n}o 2008. Este software dispone de algunas de funcionalidades desarrolladas en el nuevo software, raz\'{o}n por la cual es una importante referencia.
		
			\item \textbf{MiniScanXE Plus OCX Kit:} Es un archivo de control ActiveX dise\~{n}ado por HunterLab para controlar y/o realizar mediciones con el ``MiniScan XE Plus'', utilizando Visual Basic for Applications (VBA). Su principal objetivo es proveer a los desarrolladores con un componente reutilizable de software que da acceso a las caracteristicas m\'{a}s comunmente utilizadas por el instrumento. La interfaz p\'{u}blica que expone este archivo es utilizada para realizar la comunicaci\'{o}n entre el ``MiniScan XE Plus'' y el nuevo software.
		\end{itemize}