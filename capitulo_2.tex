\chapter{Marco Te\'{o}rico}

Este cap\'{i}tulo presenta los trabajos relacionados, la observaci\'{o}n directa y las bases te\'{o}ricas que conforman los antecedentes de la investigaci\'{o}n que sustentan el desarrollo del presente trabajo de investigaci\'{o}n.

	\section{Antecedentes}
	
		\subsection{Antecedentes de la investigaci\'{o}n}
			
			Primero se tiene el art\'{i}culo cient\'{i}fico titulado \textit{<<Comparing Quantitative Measures of Erythema, Pigmentation and Skin Response using Reflectometry>>}, realizado por \cite{Wagner}, en la Universidad del Estado de Pensilvania, Estados Unidos, y publicado por Pigment Cell Res. En este art\'{i}culo se obtiene el \'{i}ndice de eritema, que es utilizado para determinar el nivel inflamatorio de la epidermis de un paciente. El m\'{e}todo utilizado en este art\'{i}culo para su obtenci\'{o}n fue implementado en el nuevo software.
			
			Por	\'{u}ltimo est\'{a} el art\'{i}culo cient\'{i}fico titulado \textit{<<Recuperaci\'{o}n del Coeficiente de Absorci\'{o}n de la Epidermis en la Piel Humana>>}, realizado por \cite{Narea}, en la Universidad de Carabobo, Venezuela, y publicado por la Sociedad Espa\~{n}ola de \'{O}ptica. En este art\'{i}culo se determina el coeficiente de absorci\'{o}n, que es un par\'{a}metro \'{o}ptico asociado a la piel, el cual indica el nivel de concentraci\'{o}n de melanina presente en la epidermis de un paciente. La t\'{e}cnica empleada en dicho art\'{i}culo para calcular este coeficiente fue implementada en el nuevo software.

	\subsection{Observaci\'{o}n directa}
		
			El \textit{<<HunterLab Universal Software>>}, es un software comercial y privativo de 16 bits dise\~{n}ado para el sistema operativo Microsoft Windows versi\'{o}n 3.x, con la posibilidad de ejecutarse en Windows 95, Windows 98, Windows 2000 y Windows XP. Fue creado para la utilizaci\'{o}n del MiniScan XE Plus, adem\'{a}s de otros instrumentos de la empresa HunterLab, y descontinuado en el a\~{n}o 2008. Este software dispone de algunas de las funciones que fueron desarrolladas en el nuevo software, raz\'{o}n por la cual es una referencia importante de observaci\'{o}n.

			El archivo denominado \textit{<<MSXE + OCX>>}, es una hoja de c\'{a}lculo habilitada para la ejecuci\'{o}n de macroinstrucciones de Microsoft Excel, que fue proporcionada por el personal de soporte t\'{e}cnico de HunterLab como un ejemplo para utilizar el MiniScan XE Plus, empleando el uso de un kit de control denominado MiniScan XE Plus OCX Kit \textit{(MSXE.ocx)}. Este kit fue dise\~{n}ado por la empresa HunterLab para dar acceso a las caracteristicas comunmente utilizadas por dicho instrumento. El c\'{o}digo contenido en la hoja de c\'{a}lculo se emple\'{o} como referencia para el manejo del kit \textit{MSXE.ocx}.

	\section{Bases te\'{o}ricas}
	
	\subsection{Espectro de luz visible}
	
	\subsection{Espectroscop\'{i}a de reflectancia difusa}

	\subsection{Curva de reflectancia difusa}
	
	\subsection{Curva de absorbancia aparente}

	\subsection{Iluminante est\'{a}ndar D65}
	
	\subsection{Observante est\'{a}ndar de 10\degree}
	
	\begin{table}[h]
		\small
		\caption[Valores para las f\'{o}rmulas]{\textit{Valores para las f\'{o}rmulas} (Fuente: Autor).}
		\centering
		\setlength{\extrarowheight}{\altocelda}
		\begin{tabulary}{\anchotabla}{|c|c|c|c|c|}
			\hline
			$\lambda$, nm & $S(\lambda)$ & $\overline{x}(\lambda)$ & $\overline{y}(\lambda)$ & $\overline{z}(\lambda)$\\ \hline
			400 & 82.7549 & 0.019110 & 0.002004 & 0.086011\\ \hline
			410 & 91.4860 & 0.084736 & 0.008756 & 0.389366\\ \hline
			420 & 93.4318 & 0.204492 & 0.021391 & 0.972542\\ \hline
			430 & 86.6823 & 0.314679 & 0.038676 & 1.553480\\ \hline
			440 & 104.865 & 0.383734 & 0.062077 & 1.967280\\ \hline
			450 & 117.008 & 0.370702 & 0.089456 & 1.994800\\ \hline
			460 & 117.812 & 0.302273 & 0.128201 & 1.745370\\ \hline
			470 & 114.861 & 0.195618 & 0.185190 & 1.317560\\ \hline
			480 & 115.923 & 0.080507 & 0.253589 & 0.772125\\ \hline
			490 & 108.811 & 0.016172 & 0.339133 & 0.415254\\ \hline
			500 & 109.354 & 0.003816 & 0.460777 & 0.218502\\ \hline
			510 & 107.802 & 0.037465 & 0.606741 & 0.112044\\ \hline
			520 & 104.790 & 0.117749 & 0.761757 & 0.060709\\ \hline
			530 & 107.689 & 0.236491 & 0.875211 & 0.030451\\ \hline
			540 & 104.405 & 0.376772 & 0.961988 & 0.013676\\ \hline
			550 & 104.046 & 0.529826 & 0.991761 & 0.003988\\ \hline
			560 & 100.000 & 0.705224 & 0.997340 & 0.000000\\ \hline
			570 & 96.3342 & 0.705224 & 0.955552 & 0.000000\\ \hline
			580 & 95.7880 & 1.014160 & 0.868934 & 0.000000\\ \hline
			590 & 88.6856 & 1.118520 & 0.777405 & 0.000000\\ \hline
			600 & 90.0062 & 1.123990 & 0.658341 & 0.000000\\ \hline
			610 & 89.5991 & 1.030480 & 0.527963 & 0.000000\\ \hline
			620 & 87.6987 & 0.856297 & 0.398057 & 0.000000\\ \hline
			630 & 83.2886 & 0.647467 & 0.283493 & 0.000000\\ \hline
			640 & 83.6992 & 0.431567 & 0.179828 & 0.000000\\ \hline
			650 & 80.0268 & 0.268329 & 0.107633 & 0.000000\\ \hline
			660 & 80.2146 & 0.152568 & 0.060281 & 0.000000\\ \hline
			670 & 82.2778 & 0.081261 & 0.031800 & 0.000000\\ \hline
			680 & 78.2842 & 0.040851 & 0.015905 & 0.000000\\ \hline
			690 & 69.7213 & 0.019941 & 0.007749 & 0.000000\\ \hline
			700 & 71.6091 & 0.009577 & 0.003718 & 0.000000\\ \hline
		\end{tabulary}
		
	\end{table}
	
	\subsection{Coordenadas de cromaticidad CIE XYZ}
	
	\subsection{Espacio del color CIELAB}

	\subsection{Coeficiente de absorci\'{o}n}
	
	\subsection{Coeficiente de esparcimiento}
	
	\subsection{\'{I}ndice de eritema}