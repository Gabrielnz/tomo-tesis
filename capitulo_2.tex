\chapter{\label{cap:2}Marco Te\'{o}rico}

	\section{Antecedentes}	
		\begin{itemize}
		
			\item
				La luz es una sensaci\'{o}n producida por radiaci\'{o}n electromagn\'{e}tica visible, que est\'{a} dentro del rango de longitud de onda de 380 a 780 nan\'{o}metros. Dentro de este rango, la radiaci\'{o}n electromagn\'{e}tica produce la sensaci\'{o}n de luz azul, luz verde, y luz roja, las cuales son denominadas valores triest\'{i}mulo. La \textit{\mbox{Commission} Internationale de l'Eclairage} \textit{(CIE)}, defini\'{o} en 1964 un est\'{a}ndar para calcular los valores triest\'{i}mulo de un color, representandolos como coordenadas de cromaticidad \textit{XYZ}, lo que es mejor conocido como el sistema tricrom\'{a}tico \textit{CIE 1964}. Ahora bien, el espacio \textit{CIE L*a*b*}, es un sistema definido por la misma comisi\'{o}n en 1976 para la \mbox{transformaci\'{o}n} de las coordenadas de cromaticidad mencionadas, a unas coordenadas representables en un espacio de tres dimensiones. El libro titulado <<Colorimetry: Understanding the CIE System>> editado por \cite{Schanda}, proporciona las f\'{o}rmulas utilizadas para el c\'{a}lculo de las coordenadas de cromaticidad \textit{CIE} y de las coordenadas  del espacio \textit{CIE L*a*b*}, las cuales fueron implementadas en el nuevo software como funciones para ayudar a determinar ciertos par\'{a}metros \'{o}pticos, presentes en la piel de los pacientes.
			
			\item La piel es un medio biol\'{o}gico que se comporta como un medio turbio de m\'{u}ltiples capas. Tales medios poseen unos par\'{a}metros \'{o}pticos asociados a ellos, entre los cuales se encuentra el coeficiente de absorci\'{o}n. En el art\'{i}culo titulado <<Recuperaci\'{o}n del Coeficiente de Absorci\'{o}n de la Epidermis en la Piel Humana>> de \cite{Narea}, se afirma que la melanina que se encuentra distribuida en la epidermis de la piel, es el  principal agente absorbente de la misma, y por lo tanto determina en gran parte su color. El nivel de concentraci\'{o}n de la melanina presente en la epidermis de los pacientes, se representa mediante el coeficiente de absorci\'{o}n. La t\'{e}cnica empleada en el art\'{i}culo mencionado para determinar este coeficiente, fue implementada como una funci\'{o}n en el nuevo software.
			
			\item El eritema es un t\'{e}rmino m\'{e}dico dermatol\'{o}gico, utilizado para describir el enrojecimiento de la piel condicionado por una inflamaci\'{o}n. En la tesis de maestr\'{i}a de \cite{Bersha} titulada <<Spectral Imaging and Analysis of Human Skin>>, se calcula el \'{i}ndice de eritema a partir de la coordenada \textit{a*} correspondiente al espacio de color \textit{CIE L*a*b*}, y tomando en cuenta el coeficiente de absorci\'{o}n de la melanina. El m\'{e}todo para la obtenci\'{o}n del \'{i}ndice mencionado fue implementado en el nuevo software, para determinar el nivel inflamatorio de la epidermis en la piel de un paciente.
		\end{itemize}

	\section{Observaci\'{o}n Directa}
		\begin{itemize}
		
			\item \textbf{Archivo de ejemplo MSXE + OCX:} es una hoja de c\'{a}lculo habilitada para la ejecuci\'{o}n de macroinstrucciones de Microsoft Excel, que fue proporcionada por el personal de soporte t\'{e}cnico de HunterLab como un ejemplo para la utilizaci\'{o}n del MiniScan XE Plus, sin necesidad de emplear el HunterLab Universal Software. El c\'{o}digo que contiene este archivo se utiliz\'{o} como referencia para establecer la comunicaci\'{o}n entre el nuevo software y el instrumento en cuesti\'{o}n.

			\newpage
			
			\item \textbf{HunterLab Universal Software:} es un software comercial y privativo de 16 bits dise\~{n}ado para el sistema operativo Microsoft Windows version 3.x, con la posibilidad de ejecutarse en Windows 95, Windows 2000 y Windows XP. Fue creado para la utilizaci\'{o}n del MiniScan XE Plus, adem\'{a}s de otros instrumentos de la empresa HunterLab, y descontinuado en el a\~{n}o 2008. Este software dispone de algunas de las funciones que est\'{a}n siendo desarrolladas en el nuevo software, raz\'{o}n por la cual es una referencia importante de observaci\'{o}n.

		\end{itemize}