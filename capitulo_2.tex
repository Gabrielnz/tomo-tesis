\chapter{\label{cap:2}Marco Te\'{o}rico}

	\section{Antecedentes}	
		\begin{itemize}
		
			\item
				la \textit{\mbox{Commission} Internationale de l'Eclairage} \cite{CIE}, defini\'{o} en 1964 un est\'{a}ndar para calcular las coordenadas de cromaticidad que representan los valores triest\'{i}mulo de un color, mejor conocido como sistema tricrom\'{a}tico CIE 1964. \cite{Schanda} describe el procedimiento utilizado para calcular estas coordenadas, y el mismo se implement\'{o} como una funci\'{o}n en el nuevo software.	
			
			\item En el art\'{i}culo titulado ``Recuperaci\'{o}n del Coeficiente de Absorci\'{o}n de la Epidermis en la Piel Humana'' de \cite{Narea}, se obtiene el coeficiente de absorci\'{o}n de la epidermis en la piel humana a partir de datos espectrales. El procedimiento para obtener dicho coeficiente es implementado en el nuevo software para determinar el nivel de concentraci\'{o}n de melanina en la epidermis de un paciente.
			
			\item En la tesis de maestr\'{i}a de \cite{Bersha} titulada ``Spectral Imaging and Analysis of Human Skin'', se calcula el \'{i}ndice de eritema partiendo de la coordenada \textit{a*}, correspondiente al espacio de color \textit{CIELAB}. El m\'{e}todo para la obtenci\'{o}n del \'{i}ndice mencionado es implementado en el nuevo software, para determinar el nivel inflamatorio de la epidermis de un paciente.
		\end{itemize}

	\section{Observaci\'{o}n Directa}
		\begin{itemize}
			\item \textbf{HunterLab Universal Software:} Es un software propietario de 16-bit dise\~{n}ado para el Sistema Operativo Microsoft Windows Version 3.x, con la posibilidad de ejecutarse en Windows 95, Windows 2000, Windows NT y Windows XP, descontinuado en el a\~{n}o 2008. Este software dispone de algunas de funcionalidades desarrolladas en el nuevo software, raz\'{o}n por la cual es una importante referencia.
		
			\item \textbf{MiniScanXE Plus OCX Kit:} Es un archivo de control ActiveX dise\~{n}ado por HunterLab para controlar y/o realizar mediciones con el ``MiniScan XE Plus'', utilizando Visual Basic for Applications (VBA). Su principal objetivo es proveer a los desarrolladores con un componente reutilizable de software que da acceso a las caracteristicas m\'{a}s comunmente utilizadas por el instrumento. La interfaz p\'{u}blica que expone este archivo es utilizada para realizar la comunicaci\'{o}n entre el ``MiniScan XE Plus'' y el nuevo software.
		\end{itemize}