\chapter{\label{cap:2}Marco Te\'{o}rico}

	\section{Antecedentes}	
			
			En primer lugar se tiene el libro titulado  \textbf{\textit{Colorimetry: Understanding the CIE System}}, que fue editado por Schanda en la Universidad de Panonia, Hungr\'{i}a, y publicado por Wiley en Hoboken, Nueva Jersey, en el a\~{n}o 2007. En este libro se establece el c\'{a}lculo de las coordenadas de cromaticidad \textit{CIE XYZ 1964}, y del espacio \textit{CIE L*a*b* 1976}. Las coordenadas de cromaticidad \textit{CIE XYZ} son valores triest\'{i}mulo, que producen la percepci\'{o}n del color; por otro lado, el espacio \textit{CIE L*a*b*} es un sistema para transformar dichas coordenadas, y as\'{i} poder representarlas en un espacio de tres dimensiones. Las f\'{o}rmulas que son utilizadas en este libro para calcular tales coordenadas, fueron implementadas en el nuevo software para ayudar a determinar ciertos par\'{a}metros \'{o}pticos presentes en la piel de un paciente.
			
			Luego se tiene el art\'{i}culo de Narea y otros, titulado \textbf{\textit{Recuperaci\'{o}n del Coeficiente de Absorci\'{o}n de la Epidermis en la Piel Humana}}, que fue publicado por la Sociedad Espa\~{n}ola de \'{O}ptica en el a\~{n}o 2015. En este art\'{i}culo se determina el coeficiente de absorci\'{o}n, que es un par\'{a}metro \'{o}ptico asociado a la piel. Este par\'{a}metro indica el nivel de concentraci\'{o}n de melanina presente en la epidermis de la piel de un paciente. La t\'{e}cnica empleada en este art\'{i}culo para calcular este coeficiente fue implementada en el nuevo software.
			
			Por \'{u}ltimo est\'{a} la tesis de maestr\'{i}a de Bersha, titulada \textbf{\textit{Spectral Imaging and Analysis of Human Skin}}, y defendida en la Universidad del Este de Finlandia, en el a\~{n}o 2010. En este trabajo se obtiene el \'{i}ndice de eritema a partir de la coordenada \textit{a*} correspondiente al espacio de color \textit{CIE L*a*b* 1976}, y tomando en cuenta el coeficiente de absorci\'{o}n de la melanina. Dicho \'{i}ndice es utilizado para determinar el nivel inflamatorio de la epidermis en la piel de un paciente, y el m\'{e}todo para su obtenci\'{o}n fue implementado en el nuevo software.

	\section{Observaci\'{o}n Directa}
		
			El \textbf{\textit{HunterLab Universal Software}}, es un software comercial y privativo de 16 bits dise\~{n}ado para el sistema operativo Microsoft Windows version 3.x, con la posibilidad de ejecutarse en Windows 95, Windows 2000 y Windows XP. Fue creado para la utilizaci\'{o}n del MiniScan XE Plus, adem\'{a}s de otros instrumentos de la empresa HunterLab, y descontinuado en el a\~{n}o 2008. Este software dispone de algunas de las funciones que fueron desarrolladas en el nuevo software, raz\'{o}n por la cual fue una referencia importante de observaci\'{o}n.

			El archivo denominado \textbf{\textit{MSXE + OCX}}, es una hoja de c\'{a}lculo habilitada para la ejecuci\'{o}n de macroinstrucciones de Microsoft Excel, que fue proporcionada por el personal de soporte t\'{e}cnico de HunterLab como un ejemplo para utilizar el MiniScan XE Plus, empleando el uso de otro archivo denominado MiniScan XE Plus OCX Kit \textit{(MSXE.ocx)}. Este \'{u}ltimo archivo fue dise\~{n}ado por la empresa HunterLab para dar acceso a las caracteristicas comunmente utilizadas por dicho instrumento. El c\'{o}digo contenido en la hoja de c\'{a}lculo se emple\'{o} como referencia para observar el manejo del archivo \textit{MSXE.ocx}.

		