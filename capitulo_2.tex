\chapter{Marco Te\'{o}rico}

Este cap\'{i}tulo presenta los trabajos relacionados, la observaci\'{o}n directa y las bases te\'{o}ricas que conforman los antecedentes de la investigaci\'{o}n que sustentan el desarrollo del presente trabajo de investigaci\'{o}n.

	\section{Antecedentes}
	
		\subsection{Antecedentes de la investigaci\'{o}n}
			
			Primero se tiene el art\'{i}culo cient\'{i}fico titulado \textit{<<Comparing Quantitative Measures of Erythema, Pigmentation and Skin Response using Reflectometry>>}, realizado por \cite{Wagner}, en la Universidad del Estado de Pensilvania, Estados Unidos, y publicado por Pigment Cell Res. En este art\'{i}culo se obtiene el \'{i}ndice de eritema, que es utilizado para determinar el nivel inflamatorio de la epidermis en la piel de un paciente. El m\'{e}todo utilizado en este art\'{i}culo para su obtenci\'{o}n fue implementado en el nuevo software.
			
			Por	\'{u}ltimo est\'{a} el art\'{i}culo cient\'{i}fico titulado \textit{<<Recuperaci\'{o}n del Coeficiente de Absorci\'{o}n de la Epidermis en la Piel Humana>>}, realizado por \cite{Narea}, en la Universidad de Carabobo, Venezuela, y publicado por la Sociedad Espa\~{n}ola de \'{O}ptica. En este art\'{i}culo se determina el coeficiente de absorci\'{o}n, que es un par\'{a}metro \'{o}ptico asociado a la piel, el cual indica el nivel de concentraci\'{o}n de melanina presente en la epidermis de la piel de un paciente. La t\'{e}cnica empleada en dicho art\'{i}culo para calcular este coeficiente fue implementada en el nuevo software.

	\subsection{Observaci\'{o}n directa}
		
			El \textit{<<HunterLab Universal Software>>}, es un software comercial y privativo de 16 bits dise\~{n}ado para el sistema operativo Microsoft Windows versi\'{o}n 3.x, con la posibilidad de ejecutarse en Windows 95, Windows 98, Windows 2000 y Windows XP. Fue creado para la utilizaci\'{o}n del MiniScan XE Plus, adem\'{a}s de otros instrumentos de la empresa HunterLab, y descontinuado en el a\~{n}o 2008. Este software dispone de algunas de las funciones que fueron desarrolladas en el nuevo software, raz\'{o}n por la cual es una referencia importante de observaci\'{o}n.

			El archivo denominado \textit{<<MSXE + OCX>>}, es una hoja de c\'{a}lculo habilitada para la ejecuci\'{o}n de macroinstrucciones de Microsoft Excel, que fue proporcionada por el personal de soporte t\'{e}cnico de HunterLab como un ejemplo para utilizar el MiniScan XE Plus, empleando el uso de un kit de control denominado MiniScan XE Plus OCX Kit \textit{(MSXE.ocx)}. Este kit fue dise\~{n}ado por la empresa HunterLab para dar acceso a las caracteristicas comunmente utilizadas por dicho instrumento. El c\'{o}digo contenido en la hoja de c\'{a}lculo se emple\'{o} como referencia para el manejo del kit \textit{MSXE.ocx}.

	\section{Bases te\'{o}ricas}
	
	\subsection{Espectro de luz visible}
	
	\subsection{Espectroscop\'{i}a de reflectancia difusa}

	\subsection{Curva de reflectancia difusa}
	
	\subsection{Curva de absorbancia aparente}
	
	\subsection{Observante est\'{a}ndar de 10\degree}
	
	\subsection{Iluminante est\'{a}ndar D65}
	
	\subsection{Coordenadas de cromaticidad CIE XYZ}
	
	\subsection{Espacio del color CIELAB}

	\subsection{Coeficiente de absorci\'{o}n}
	
	\subsection{Coeficiente de esparcimiento}
	
	\subsection{\'{I}ndice de eritema}