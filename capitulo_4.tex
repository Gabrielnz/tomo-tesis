\chapter{Resultados}

En este cap\'{i}tulo se muestran los resultados que fueron alcanzados en el desarrollo del presente trabajo de investigaci\'{o}n, incluyendo las fases metodol\'{o}gicas, la base de datos, las tecnolog\'{i}as y los recursos utilizados.

\section{Fases metodol\'{o}gicas}

\subsection{Visi\'{o}n}
	
	\subsubsection{Enunciado del problema}
	
	El problema que se presenta es que se est\'{a} utilizando el HunterLab Universal Software para el manejo del espectrofot\'{o}metro de reflexi\'{o}n difusa MiniScan XE Plus. Dicho software est\'{a} en ingl\'{e}s, es comercial, privativo y fue descontinuado; esto afecta a los dermat\'{o}logos del Centro de Investigaciones M\'{e}dicas y Biotecnol\'{o}gicas de la Universidad de Carabobo (CIMBUC).
	
	El impacto causado por esto es que los dermat\'{o}logos encuentran el HunterLab Universal Software dif\'{i}cil de utilizar, e imposible de adaptar a sus necesidades, lo que ralentiza la actividad de consulta con sus pacientes, genera la necesidad de disponer de personal especializado para su debido uso, y disminuye el potencial de dicho instrumento.
	
	Una soluci\'{o}n satisfactoria ser\'{i}a disponer de un software para el uso del \mbox{MiniScan} XE Plus que est\'{e} en espa\~{n}ol, que sea amigable y mantenible, permitiendo que se adapte a las necesidades de los dermat\'{o}logos.
	
	\subsubsection{Descripci\'{o}n de los usuarios}
	
		\begin{table}[h]
		\small
		\caption[Actores del negocio]{\textit{Actores del negocio} (Fuente: Autor).}
		\centering
		\setlength{\extrarowheight}{\altocelda}
		\begin{tabulary}{\anchotabla}{|c|J|}
			\hline
			\thead{\textbf{\small{Actor}}} & \thead{\textbf{\small{Descripci\'{o}n}}}\\ \hline
			\textbf{Administrador} & 
			
			Consulta las historias m\'{e}dicas de los pacientes.
			
			Consulta las muestras pertenecientes a las historias m\'{e}dicas.
			
			Gestiona los usuarios.
		\\ \hline
			\textbf{Dermat\'{o}logo} & 
			Realiza mediciones.
			
			Gestiona las historias m\'{e}dicas de los pacientes.
			
			Gestiona las muestras pertenecientes a las historias m\'{e}dicas.
			
			Realiza diagn\'{o}sticos a partir de los resultados de las muestras.
		\\ \hline
			\textbf{Investigador} &
			
			Consulta las historias m\'{e}dicas de los pacientes.
			
			Consulta las muestras pertenecientes a las historias m\'{e}dicas.
			
			Realiza an\'{a}lisis sobre los resultados de las mediciones.\\ \hline
		\end{tabulary}
	\end{table}
	
	\begin{table}[h]
		\small
		\caption[Actores del software]{\textit{Actores del software} (Fuente: Autor).}
		\centering
		\setlength{\extrarowheight}{\altocelda}
		\begin{tabulary}{\anchotabla}{|c|J|c|c|}
			\hline
			\thead{\textbf{\small{Actor}}} & \thead{\textbf{\small{Responsabilidad}}} & \thead{\textbf{\small{Experiencia}}} & \thead{\textbf{\small{Uso}}}\\ \hline
			
			\textbf{Administrador} &
			
			Crear, consultar, modificar y eliminar usuarios.
			
			Consultar historias m\'{e}dicas de pacientes.
			
			Consultar muestras de pacientes. &
			Alta &
			Alto\\ \hline
			
			\textbf{Dermat\'{o}logo} &
			
			Crear, consultar, modificar y eliminar historias m\'{e}dicas de pacientes.
			
			Realizar mediciones.
			
			Crear, consultar, modificar y eliminar muestras de pacientes. &
			Baja &
			Alto\\ \hline
			
			\textbf{Investigador} &
			
			Consultar historias m\'{e}dicas de pacientes.
			
			Consultar muestras de pacientes. &
			Media &
			Alto\\ \hline
		\end{tabulary}
	\end{table}
	
	\subsubsection{Resumen del producto}
	
	El software desarrollado, denominado a partir de ahora Spectrasoft, es una aplicaci\'{o}n para el uso del MiniScan XE Plus, la recuperaci\'{o}n de los datos de medici\'{o}n de dicho instrumento, la generaci\'{o}n de resultados relevantes y la gesti\'{o}n de los mismos, el cu\'{a}l est\'{a} orientado a las actividades m\'{e}dicas dermatol\'{o}gicas del Centro de Investigaciones M\'{e}dicas y Biotecnol\'{o}gicas de la Universidad de Carabobo (CIMBUC). La tabla 4.3 resume los beneficios y las caracter\'{i}sticas m\'{a}s importantes que provee el producto.
	
	\begin{table}[h]
		\small
		\caption[Beneficios y caracter\'{i}sticas principales del producto]{\textit{Beneficios y caracter\'{i}sticas principales del producto} (Fuente: Autor).}
		\centering
		\setlength{\extrarowheight}{\altocelda}
		\begin{tabulary}{\anchotabla}{|J|J|}
			\hline
			\thead{\textbf{\small{Beneficio al cliente}}} & \thead{\textbf{\small{Caracter\'{i}stica que lo soporta}}}\\ \hline
			Se puede conectar, calibrar y realizar mediciones con el MiniScan XE Plus. & 
			Comunicaci\'{o}n con el MiniScan XE Plus y acceso a las caracter\'{i}sticas comunmente utilizadas por el mismo.\\ \hline
			Se dispone de informaci\'{o}n relevante para el diagn\'{o}stico de patolog\'{i}as dermatol\'{o}gicas en la piel de los pacientes. &
			Muestra de los datos espectrales obtenidos de las mediciones, representaci\'{o}n gr\'{a}fica de los mismos, y c\'{a}lculo de valores adicionales asociados a dichos datos.\\ \hline
			Se pueden gestionar los usuarios, las historias m\'{e}dicas y las muestras generadas de las mediciones. &
			Manejo de una base de datos que almacena toda la informaci\'{o}n referente a los usuarios, las historias m\'{e}dicas y las muestras, permitiendo su gesti\'{o}n por medio del Spectrasoft.\\ \hline
			El Spectrasoft se puede utilizar con facilidad. &
			Interfaz gr\'{a}fica de usuario en espa\~{n}ol, que ofrece \'{u}nicamente las funciones necesarias para gestionar la informaci\'{o}n que necesitan los usuarios. \\ \hline
			El Spectrasoft se puede adaptar a las futuras necesidades de sus usuarios. &
			C\'{o}digo abierto del proyecto disponible en su totalidad para realizar cualquier modificaci\'{o}n y/o extensi\'{o}n.\\ \hline
		\end{tabulary}
	\end{table}
	
	\subsubsection{Principales restricciones}
	
	El software se desarrolla utilizando el lenguaje de programaci\'{o}n C++, empleando \'{u}nicamente tecnolog\'{i}as gratuitas, y, en la medida de lo posible, de c\'{o}digo abierto. Este se ejecuta en sistemas operativos Windows actuales. Por \'{u}ltimo, la comunicaci\'{o}n entre el software y el MiniScan XE Plus se logra tanto por medio de un puerto serial, como por medio de un adaptador USB.
	
\subsection{Pila del producto \textit{(product backlog)}}
	La lista de requerimientos funcionales que necesitan ser realizados a lo largo del desarrollo del Spectrasoft, son el producto de una reuni\'{o}n que se tuvo con los dermat\'{o}logos y los investigadores del CIMBUC, al igual que de la observaci\'{o}n directa efectuada a las funciones que el HunterLab Universal Software ofrece para el manejo del MiniScan XE Plus. En la tabla 4.4 se muestran dichos requerimientos.
	
	\begin{table}[h]
		\small
		\caption[Requerimientos funcionales del software]{\textit{Requerimientos funcionales del software} (Fuente: Autor).}
		\centering
		\setlength{\extrarowheight}{\altocelda}
		\begin{tabulary}{\anchotabla}{|c|J|c|}
			\hline
			\thead{\textbf{\small{C\'{o}digo}}} & \thead{\textbf{\small{Requerimiento}}} & \thead{\textbf{\small{Prioridad}}}\\ \hline
			
			\textbf{RF01} & Conectar y desconectar el MiniScan XE Plus. & Esencial\\ \hline
			
			\textbf{RF02} & Estandarizar el MiniScan XE Plus. & Esencial\\ \hline
			
			\textbf{RF03} & Recuperar los 31 puntos espectrales de una medici\'{o}n con el MiniScan XE Plus y mostrarlos en su forma num\'{e}rica. & Esencial\\ \hline

			\textbf{RF04} & Graficar una curva de reflectancia difusa a partir de los 31 puntos espectrales recuperados. & Esencial\\ \hline
			\textbf{RF05} & Graficar una curva de absorbancia aparente a partir de los 31 puntos espectrales recuperados. & Esencial\\ \hline
			\textbf{RF06} & Calcular y mostrar las coordenadas de cromaticidad CIE xyz. & Esencial\\ \hline
			
			\textbf{RF07} & Calcular y mostrar las coordenadas del espacio CIELAB. & Esencial\\ \hline

			\textbf{RF08} & Calcular y mostrar el coeficiente de absorci\'{o}n de la epidermis. & Esencial\\ \hline

			\textbf{RF09} & Calcular y mostrar el coeficiente de esparcimiento de la epidermis. & Esencial\\ \hline

			\textbf{RF010} & Calcular y mostrar el \'{i}ndice de eritema. & Esencial\\ \hline

			\textbf{RF11} & Almacenar la informaci\'{o}n de los usuarios, las historias m\'{e}dicas y las muestras en una base de datos. & Esencial\\ \hline			

			\textbf{RF12} & Gestionar la creaci\'{o}n, consulta, modificaci\'{o}n y eliminaci\'{o}n de los usuarios. & Esencial\\ \hline
			
			\textbf{RF13} & Gestionar la creaci\'{o}n, consulta, modificaci\'{o}n y eliminaci\'{o}n de las historias m\'{e}dicas de pacientes. & Esencial\\ \hline
			
			\textbf{RF14} & Gestionar la creaci\'{o}n, consulta, modificaci\'{o}n y eliminaci\'{o}n de las muestras pertenecientes a las historias m\'{e}dicas. & Esencial\\ \hline	
			
			\textbf{RF15} & Exportar los datos de una muestra a un archivo port\'{a}til. & Esencial\\ \hline
		\end{tabulary}
	\end{table}
	
\subsection{Requerimientos no funcionales}
	
	De la misma forma que los requerimientos funcionales, la lista de los requerimientos no funcionales fue definida en la misma reuni\'{o}n con los dermat\'{o}logos y los investigadores, tomando en cuenta las restricciones del entorno en donde se va a ejecutar el Spectrasoft. En la tabla 4.5 se describen estos requerimientos no funcionales.
	
	\begin{table}[h]
		\small
		\caption[Requerimientos no funcionales del software]{\textit{Requerimientos no funcionales del software} (Fuente: Autor).}
		\centering
		\setlength{\extrarowheight}{\altocelda}
		\begin{tabulary}{\anchotabla}{|c|J|c|}
			\hline
			\thead{\textbf{\small{C\'{o}digo}}} & \thead{\textbf{\small{Requerimiento}}} & \thead{\textbf{\small{Prioridad}}}\\ \hline
			\textbf{RNF01} & El software debe ser f\'{a}cil de utilizar, por lo que debe cumplir con el atributo de usabilidad de un software de calidad. & Esencial\\ \hline
			\textbf{RNF02} & El software debe ser capaz de adaptarse a las necesidades de los dermat\'{o}logos, raz\'{o}n por la cual debe cumplir con el atributo de mantenibilidad de un software de calidad. & Esencial\\ \hline
			\textbf{RNF03} & El software debe desarrollarse empleando \'{u}nicamente herramientas y tecnolog\'{i}as gratuitas, que adem\'{a}s sean, en la medida de lo posible, de c\'{o}digo abierto. & Esencial\\ \hline
			\textbf{RNF04} & El software debe ser capaz de ejecutarse en sistemas Windows actuales, con arquitecturas de 32 bits y 64 bits. & Esencial\\ \hline
			\textbf{RNF05} & El software debe conectarse con el MiniScan XE Plus por medio de un puerto serial o de un adaptador USB. & Esencial\\ \hline
			\textbf{RNF06} & El archivo port\'{a}til al que se exportan los resultados de una medici\'{o}n debe ser abierto por un visualizador/editor de hojas de c\'{a}lculo. & Esencial\\ \hline
			\textbf{RNF07} & El software debe desarrollarse utilizando el lenguaje de programaci\'{o}n orientada a objetos C++. & Esencial\\ \hline
		\end{tabulary}
	\end{table}

\subsection{Casos de uso}

\subsection{Glosario}

\newpage

\section{Base de datos}

	\subsection{Diagrama ER de la base de datos}

	\begin{figure}[H]
		\centering
		\includegraphics[scale=0.55]{img/diagramaER.png}
			\caption[Diagrama ER de la base de datos]{\textit{Diagrama ER de la base de datos} (Fuente: Autor).}
	\end{figure}
	
	\subsection{Descripci\'{o}n de las tablas de la base de datos}
	
		\begin{itemize}
				
				\item \textbf{historia:} almacena los datos referentes a la historia m\'{e}dica de cada uno de los pacientes registrados.
				
				\item \textbf{usuario:} guarda la informaci\'{o}n de cada uno de los usuarios que pueden acceder al software, que pueden ser administradores, dermat\'{o}logos o investigadores.
				
				\item \textbf{muestra:} contiene los datos relevantes de las muestras que son tomadas a los pacientes. Dichas muestras siempre est\'{a}n interrelacionadas con la historia m\'{e}dica del paciente al que pertenece, y la c\'{e}dula del usuario que la tom\'{o}.

				\item \textbf{datos\_espectrales:} contiene los 31 puntos espectrales resultantes de la medici\'{o}n de cada muestra, empleando el uso del MiniScan XE Plus.
				
				\item \textbf{datos\_adicionales:} almacena los datos que son calculados a partir de los 31 puntos espectrales de cada muestra, estos son las coordenadas de cromaticiad CIE xyz, las coordenadas del espacio del color CIELAB, y el coeficiente de esparcimiento, el coeficiente de absorci\'{o}n, y, por \'{u}ltimo, el \'{i}ndice de eritema.
				
		\end{itemize}
		
\section{Tecnolog\'{i}as y recursos utilizados}

	\subsection{Tecnolog\'{i}as}
	
		\begin{itemize}
			
			\item \textbf{Qt:} es un \textit{framework} de desarrollo de aplicaciones multiplataforma para sistemas operativos de escritorio, sistemas integrados, y sistemas m\'{o}viles. Se utiliz\'{o} la versi\'{o}n \textit{open source} 5.4.1 de este \textit{framework} para el desarrollo del Spectrasoft.
			
			\item \textbf{Visual Studio:} es un entorno integrado de desarrollo o \textit{IDE} para crear aplicaciones en varias plataformas, como Windows, Android y iOS. La versi\'{o}n 2013 de este \textit{IDE} fue utilizada para desarrollar una librer\'{i}a escrita en Visual Basic.NET, la cual act\'{u}a como intermediaria entre el archivo MSXE.ocx y el \textit{framework} Qt, para as\'{i} utilizar las caracter\'{i}sticas del MiniScan XE Plus junto con el Spectrasoft.
			
			\item \textbf{QCustomPlot:} es un \textit{widget open source} para Qt que permite realizar el trazado y la visualizaci\'{o}n de datos. Este \textit{widget} fue empleado por el Spectrasoft para visualizar la curva de reflectancia difusa y la curva de absorbancia aparente asociadas a los 31 puntos espectrales resultantes de las mediciones.
			
			\item \textbf{QtXlsx:} es una librer\'{i}a \textit{open source} para Qt que permite leer y escribir archivos con extensi\'{o}n xlsx. Esta librer\'{i}a fue utilizada para implementar en el Spectrasoft la opci\'{o}n de exportar los resultados de una muestra en un archivo port\'{a}til y manejable por aplicaciones de hojas de c\'{a}lculo.
		\end{itemize}
	
	\subsection{Recursos}
	
		\begin{itemize}
		
			\item \textbf{MiniScan XE Plus:} es un instrumento de medici\'{o}n del color creado por la empresa HunterLab, de dise\~{n}o compacto y port\'{a}til, que emplea la t\'{e}cnica de espectroscop\'{i}a de reflectancia difusa, el cual se puede apreciar en la figura 4.2. Este instrumento mide la cantidad de luz que refleja una muestra dentro de un rango de longitudes de onda que va desde los 400 hasta los 700 nan\'{o}metros, generando como resultado 31 puntos espectrales dentro de ese rango, que son el insumo principal del Spectrasoft.
			
		\end{itemize}
		
			\begin{figure}[H]
		\centering
		\includegraphics[scale=1]{img/MiniScanXEPlus.png}
			\caption[MiniScan XE Plus]{\textit{MiniScan XE Plus} (Fuente: HunterLab, 2006).}
	\end{figure}