\chapter{\label{cap:4}Resultados}

\section{Requerimientos funcionales}
\FloatBarrier
\vline
	\begin{table}[htb]
		\small
		\caption{\textbf{Tabla 6.} \textit{Requerimientos funcionales del software} (Fuente: Elaboraci\'{o}n propia).}
		\centering
		\setlength{\extrarowheight}{5pt}
		\begin{tabulary}{15.5cm}{|c|L|c|}
			\hline
			\thead{\textbf{\small{C\'{o}digo}}} & \thead{\textbf{\small{Requerimiento}}} & \thead{\textbf{\small{Prioridad}}}\\ \hline
			\textbf{REQF01} & Recuperar los 31 puntos espectrales de la medici\'{o}n con el MiniScan XE Plus y mostrarlos en su forma num\'{e}rica. & Esencial\\ \hline

			\textbf{REQF03} & Representar los 31 puntos espectrales en forma de una curva de reflectancia difusa. & Esencial\\ \hline
			\textbf{REQF04} & Representar los 31 puntos espectrales en forma de una curva de absorbancia aparente. & Esencial\\ \hline
			\textbf{REQF05} & Calcular y mostrar las coordenadas de cromaticidad \textit{CIE XYZ 1964}. & Esencial\\ \hline
			\textbf{REQF06} & Calcular y mostrar las coordenadas del espacio \textit{CIE L*a*b* 1976}. & Esencial\\ \hline
			\textbf{REQF07} & Calcular y mostrar el coeficiente de absorci\'{o}n de la epidermis. & Esencial\\ \hline
			\textbf{REQF08} & Calcular y mostrar el coeficiente de esparcimiento de la epidermis. & Esencial\\ \hline
			\textbf{REQF09} & Calcular y mostrar el \'{i}ndice de eritema. & Esencial\\ \hline
			\textbf{REQF010} & Guardar los resultados en un archivo port\'{a}til, empleando el uso del formato de historia m\'{e}dica de los dermat\'{o}logos. & Esencial\\ \hline
			\textbf{REQF011} & Abrir los resultados guardados previamente en el archivo port\'{a}til. & Esencial\\ \hline
		\end{tabulary}
	\end{table}
\FloatBarrier %you shall not pass table!!

\section{Requerimientos no funcionales}
\FloatBarrier
\vline
	\begin{table}[htb]
		\small
		\caption{\textbf{Tabla 7.} \textit{Requerimientos no funcionales del software} (Fuente: Elaboraci\'{o}n propia).}
		\centering
		\setlength{\extrarowheight}{5pt}
		\begin{tabulary}{15.5cm}{|c|L|c|}
			\hline
			\thead{\textbf{\small{C\'{o}digo}}} & \thead{\textbf{\small{Requerimiento}}} & \thead{\textbf{\small{Prioridad}}}\\ \hline
			\textbf{REQNF01} & El software debe ser capaz de ejecutarse en sistemas Windows actuales, con arquitecturas de 32 bits y 64 bits. & Esencial\\ \hline
			\textbf{REQNF02} & El software debe conectarse con el MiniScan XE Plus por medio de un adaptador USB. & Esencial\\ \hline
			\textbf{REQNF03} & El archivo port\'{a}til que maneja el software debe poder ser abierto por un visualizador/editor de hojas de c\'{a}lculo. & Esencial\\ \hline
			\textbf{REQNF04} & El software debe desarrollarse utilizando el lenguaje de programaci\'{o}n orientada a objetos C++. & Esencial\\ \hline
		\end{tabulary}
	\end{table}
\FloatBarrier %you shall not pass table!!