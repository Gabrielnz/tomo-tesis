\chapter{Resultados}

En este cap\'{i}tulo se muestran los resultados que fueron alcanzados en el desarrollo del presente trabajo de investigaci\'{o}n, siguiendo la metodolog\'{i}a Investigaci\'{o}n-Acci\'{o}n, y la configuraci\'{o}n de la metodolog\'{i}a SCRUM junto con los artefactos inclu\'{i}dos de la metodolog\'{i}a RUP.

\section{Visi\'{o}n}
	
	\subsection{Enunciado del problema}
	
	El problema que se presenta es que se est\'{a} utilizando el HunterLab Universal Software para el manejo del espectrofot\'{o}metro de reflexi\'{o}n difusa MiniScan XE Plus, que est\'{a} en ingl\'{e}s, es comercial, privativo y fue descontinuado; esto afecta a los dermat\'{o}logos y los investigadores del Centro de Investigaciones M\'{e}dicas y Biotecnol\'{o}gicas de la Universidad de Carabobo (CIMBUC).
	
	El impacto causado por esto es que los dermat\'{o}logos y los investigadores encuentran el HunterLab Universal Software dif\'{i}cil de utilizar, e imposible de adaptar a sus necesidades, lo que ralentiza la actividad de consulta con los pacientes, genera la necesidad de disponer de personal especializado para su debido uso, y disminuye el potencial de dicho instrumento.
	
	Una soluci\'{o}n satisfactoria ser\'{i}a disponer de un software para el uso del \mbox{MiniScan} XE Plus que est\'{e} en espa\~{n}ol, que sea amigable y mantenible, permitiendo que se adapte a las necesidades de los dermat\'{o}logos y los investigadores.
	
	\subsection{Descripci\'{o}n de los usuarios}
	
		\begin{table}[h]
		\small
		\caption[Actores del negocio]{\textit{Actores del negocio} (Fuente: Autor).}
		\centering
		\setlength{\extrarowheight}{\altocelda}
		\begin{tabulary}{\anchotabla}{|c|J|}
			\hline
			\thead{\textbf{\small{Actor}}} & \thead{\textbf{\small{Descripci\'{o}n}}}\\ \hline
			\textbf{Dermat\'{o}logo} & 
			Realiza mediciones a los pacientes.
			
			Determina diagn\'{o}sticos a los pacientes.
			
			Gestiona los resultados de las mediciones.
		\\ \hline
			\textbf{Investigador} &
			Realiza mediciones a los pacientes.
			
			Realiza actividades de investigaci\'{o}n.
			
			Gestiona los resultados de las mediciones.\\ \hline
		\end{tabulary}
	\end{table}
	
			\begin{table}[h]
		\small
		\caption[Usuarios del software]{\textit{Usuarios del software} (Fuente: Autor).}
		\centering
		\setlength{\extrarowheight}{\altocelda}
		\begin{tabulary}{\anchotabla}{|c|L|c|c|}
			\hline
			\thead{\textbf{\small{Usuario}}} & \thead{\textbf{\small{Responsabilidad}}} & \thead{\textbf{\small{Experiencia}}} & \thead{\textbf{\small{Uso}}}\\ \hline
			\textbf{Dermat\'{o}logo} &
			Calibrar el instrumento.
			
			Realizar mediciones.
			
			Gestionar los resultados. &
			Baja &
			Alto\\ \hline
			
			\textbf{Investigador} &
			Calibrar el instrumento.
			
			Realizar mediciones.
			
			Gestionar los resultados. &
			Media &
			Alto\\ \hline
		\end{tabulary}
	\end{table}
	
	\subsection{Resumen del producto}
	
	El software desarrollado, denominado a partir de ahora Spectrasoft, es una aplicaci\'{o}n para el uso del MiniScan XE Plus, la recuperaci\'{o}n de los datos de medici\'{o}n de dicho instrumento, la generaci\'{o}n de resultados relevantes y la gesti\'{o}n de los mismos, el cu\'{a}l est\'{a} orientado a las actividades m\'{e}dicas dermatol\'{o}gicas y de investigaci\'{o}n del Centro de Investigaciones M\'{e}dicas y Biotecnol\'{o}gicas de la Universidad de Carabobo (CIMBUC). La tabla 4.3 resume los beneficios y las caracter\'{i}sticas m\'{a}s importantes que provee el producto.
	
	\begin{table}[h]
		\small
		\caption[Beneficios y caracter\'{i}sticas principales del producto]{\textit{Beneficios y caracter\'{i}sticas principales del producto} (Fuente: Autor).}
		\centering
		\setlength{\extrarowheight}{\altocelda}
		\begin{tabulary}{\anchotabla}{|J|J|}
			\hline
			\thead{\textbf{\small{Beneficio al cliente}}} & \thead{\textbf{\small{Caracter\'{i}stica que lo soporta}}}\\ \hline
			Los dermat\'{o}logos y los investigadores pueden conectarse con el MiniScan XE Plus, calibrarlo y realizar mediciones con este instrumento. & 
			Comunicaci\'{o}n con el MiniScan XE Plus y acceso a las caracter\'{i}sticas comunmente utilizadas por el mismo.\\ \hline
			Se dispone de informaci\'{o}n relevante para el diagn\'{o}stico de patolog\'{i}as dermatol\'{o}gicas en la piel de los pacientes. &
			Muestra de los datos espectrales obtenidos de las mediciones, representaci\'{o}n gr\'{a}fica de los mismos, y c\'{a}lculo coordenadas, \'{i}ndices y coeficientes asociados a dichos datos.\\ \hline
			Se pueden gestionar los resultados de las mediciones empleando el uso del formato con el que trabajan los dermat\'{o}logos y los investigadores. &
			Guardado de los resultados en archivos port\'{a}tiles que poseen el formato que utilizan los dermat\'{o}logos y los investigadores, y cargado de los mismos al Spectrasoft para su visualizaci\'{o}n.\\ \hline
			Los dermat\'{o}logos y los investigadores pueden utilizar el Spectrasoft con facilidad. &
			Interfaz gr\'{a}fica de usuario en espa\~{n}ol, la cual ofrece \'{u}nicamente las funciones que necesitan los dermat\'{o}logos y los investigadores.\\ \hline
			El Spectrasoft se puede adaptar a futuras necesidades de los dermat\'{o}logos y los investigadores. &
			C\'{o}digo abierto disponible para realizar cualquier modificaci\'{o}n y/o extensi\'{o}n.\\ \hline
		\end{tabulary}
	\end{table}
	
	\subsection{Principales restricciones}
	
	El software se desarrolla utilizando el lenguaje de programaci\'{o}n C++ para favorecer su rendimiento, empleando \'{u}nicamente tecnolog\'{i}as gratuitas, y, en la medida de lo posible, de c\'{o}digo abierto. Este debe ser capaz de ejecutarse en sistemas operativos Windows actuales. Por \'{u}ltimo, la comunicaci\'{o}n entre el software y el MiniScan XE Plus se logra tanto por medio de un puerto serial, como por medio de un adaptador USB.
	
\section{Pila del producto \textit{(product backlog)}}
	La lista de requerimientos funcionales que necesitan ser realizados a lo largo del desarrollo del Spectrasoft, son el producto de una reuni\'{o}n que se tuvo con los dermat\'{o}logos y los investigadores del CIMBUC, al igual que de la observaci\'{o}n directa efectuada a las funciones que el HunterLab Universal Software ofrece para el manejo del MiniScan XE Plus. En la tabla 4.4 se muestran dichos requerimientos.
	
	\begin{table}[h]
		\small
		\caption[Requerimientos funcionales del software]{\textit{Requerimientos funcionales del software} (Fuente: Autor).}
		\centering
		\setlength{\extrarowheight}{\altocelda}
		\begin{tabulary}{\anchotabla}{|c|J|c|}
			\hline
			\thead{\textbf{\small{C\'{o}digo}}} & \thead{\textbf{\small{Requerimiento}}} & \thead{\textbf{\small{Prioridad}}}\\ \hline
			
			\textbf{RF01} & Estandarizar el MiniScan XE Plus. & Esencial\\ \hline
			
			\textbf{RF02} & Recuperar los 31 puntos espectrales de la medici\'{o}n con el MiniScan XE Plus y mostrarlos en su forma num\'{e}rica. & Esencial\\ \hline

			\textbf{RF03} & Graficar una curva de reflectancia difusa a partir de los 31 puntos espectrales recuperados. & Esencial\\ \hline
			\textbf{RF04} & Graficar una curva de absorbancia aparente a partir de los 31 puntos espectrales recuperados. & Esencial\\ \hline
			\textbf{RF05} & Calcular y mostrar las coordenadas del sistema \textit{CIE XYZ 1964}. & Esencial\\ \hline
			\textbf{RF06} & Calcular y mostrar las coordenadas del sistema \textit{CIE L*a*b* 1976}. & Esencial\\ \hline
			\textbf{RF07} & Calcular y mostrar el coeficiente de absorci\'{o}n de la epidermis. & Esencial\\ \hline
			\textbf{RF08} & Calcular y mostrar el coeficiente de esparcimiento de la epidermis. & Esencial\\ \hline
			\textbf{RF09} & Calcular y mostrar el \'{i}ndice de eritema. & Esencial\\ \hline
			\textbf{RF010} & Guardar los resultados en un archivo port\'{a}til, empleando el uso del formato de historia m\'{e}dica de los dermat\'{o}logos. & Esencial\\ \hline
			\textbf{RF11} & Abrir y cargar los resultados guardados en el archivo port\'{a}til. & Esencial\\ \hline
		\end{tabulary}
	\end{table}
	
\section{Requerimientos no funcionales}
	
	De la misma forma que los requerimientos funcionales, la lista de los requerimientos no funcionales fue definida en la misma reuni\'{o}n con los dermat\'{o}logos y los investigadores, tomando en cuenta las restricciones del entorno en donde se va a ejecutar el Spectrasoft. En la tabla 4.5 se describen estos requerimientos no funcionales.
	
	\begin{table}[h]
		\small
		\caption[Requerimientos no funcionales del software]{\textit{Requerimientos no funcionales del software} (Fuente: Autor).}
		\centering
		\setlength{\extrarowheight}{\altocelda}
		\begin{tabulary}{\anchotabla}{|c|J|c|}
			\hline
			\thead{\textbf{\small{C\'{o}digo}}} & \thead{\textbf{\small{Requerimiento}}} & \thead{\textbf{\small{Prioridad}}}\\ \hline
			\textbf{RNF01} & El software debe ser f\'{a}cil de utilizar, por lo que debe cumplir con el atributo de usabilidad de un software de calidad. & Esencial\\ \hline
			\textbf{RNF02} & El software debe ser capaz de adaptarse a las necesidades de los dermat\'{o}logos, raz\'{o}n por la cual debe cumplir con el atributo de mantenibilidad de un software de calidad. & Esencial\\ \hline
			\textbf{RNF03} & El software debe desarrollarse empleando \'{u}nicamente tecnolog\'{i}as gratuitas, y, en la medida de lo posible, de c\'{o}digo abierto. & Esencial\\ \hline
			\textbf{RNF04} & El software debe ser capaz de ejecutarse en sistemas Windows actuales, con arquitecturas de 32 bits y 64 bits. & Esencial\\ \hline
			\textbf{RNF05} & El software debe conectarse con el MiniScan XE Plus por medio de un puerto serial o de un adaptador USB. & Esencial\\ \hline
			\textbf{RNF06} & El archivo port\'{a}til que maneja el software debe ser abierto por un visualizador/editor de hojas de c\'{a}lculo. & Esencial\\ \hline
			\textbf{RNF07} & El software debe desarrollarse utilizando el lenguaje de programaci\'{o}n orientada a objetos C++. & Esencial\\ \hline
		\end{tabulary}
	\end{table}

\section{Casos de uso}

\section{Glosario}
