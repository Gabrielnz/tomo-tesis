\chapter{Resultados}

En este cap\'{i}tulo se muestran los resultados que fueron alcanzados en el desarrollo del presente trabajo de investigaci\'{o}n, siguiendo la metodolog\'{i}a Investigaci\'{o}n-Acci\'{o}n, y la configuraci\'{o}n de la metodolog\'{i}a SCRUM junto con los artefactos inclu\'{i}dos de la metodolog\'{i}a RUP.

\section{Pila del producto \textit{(product backlog)}}
	\blindtext
	
	\begin{table}[t]
		\small
		\caption[Requerimientos funcionales del software]{\textit{Requerimientos funcionales del software} (Fuente: Autor).}
		\centering
		\setlength{\extrarowheight}{\altocelda}
		\begin{tabulary}{\anchotabla}{|c|J|c|}
			\hline
			\thead{\textbf{\small{C\'{o}digo}}} & \thead{\textbf{\small{Requerimiento}}} & \thead{\textbf{\small{Prioridad}}}\\ \hline
			\textbf{RF01} & Recuperar los 31 puntos espectrales de la medici\'{o}n con el MiniScan XE Plus y mostrarlos en su forma num\'{e}rica. & Esencial\\ \hline

			\textbf{RF02} & Graficar una curva de reflectancia difusa a partir de los 31 puntos espectrales recuperados. & Esencial\\ \hline
			\textbf{RF03} & Graficar una curva de absorbancia aparente a partir de los 31 puntos espectrales recuperados. & Esencial\\ \hline
			\textbf{RF04} & Calcular y mostrar las coordenadas del sistema \textit{CIE XYZ 1964}. & Esencial\\ \hline
			\textbf{RF05} & Calcular y mostrar las coordenadas del sistema \textit{CIE L*a*b* 1976}. & Esencial\\ \hline
			\textbf{RF06} & Calcular y mostrar el coeficiente de absorci\'{o}n de la epidermis. & Esencial\\ \hline
			\textbf{RF07} & Calcular y mostrar el coeficiente de esparcimiento de la epidermis. & Esencial\\ \hline
			\textbf{RF08} & Calcular y mostrar el \'{i}ndice de eritema. & Esencial\\ \hline
			\textbf{RF09} & Guardar los resultados en un archivo port\'{a}til, empleando el uso del formato de historia m\'{e}dica de los dermat\'{o}logos. & Esencial\\ \hline
			\textbf{RF10} & Abrir y cargar los resultados guardados en el archivo port\'{a}til. & Esencial\\ \hline
		\end{tabulary}
	\end{table}
	\FloatBarrier
	
\section{Requerimientos no funcionales}
	\blindtext
	
	\begin{table}[t]
		\small
		\caption[Requerimientos no funcionales del software]{\textit{Requerimientos no funcionales del software} (Fuente: Autor).}
		\centering
		\setlength{\extrarowheight}{\altocelda}
		\begin{tabulary}{\anchotabla}{|c|J|c|}
			\hline
			\thead{\textbf{\small{C\'{o}digo}}} & \thead{\textbf{\small{Requerimiento}}} & \thead{\textbf{\small{Prioridad}}}\\ \hline
			\textbf{RNF01} & El software debe ser f\'{a}cil de utilizar, por lo que debe cumplir con el atributo de usabilidad de un software de calidad. & Esencial\\ \hline
			\textbf{RNF02} & El software debe ser capaz de adaptarse a las necesidades de los dermat\'{o}logos, raz\'{o}n por la cual debe cumplir con el atributo de mantenibilidad de un software de calidad. & Esencial\\ \hline
			\textbf{RNF03} & El software debe ser capaz de ejecutarse en sistemas Windows actuales, con arquitecturas de 32 bits y 64 bits. & Esencial\\ \hline
			\textbf{RNF04} & El software debe conectarse con el MiniScan XE Plus por medio de un adaptador USB. & Esencial\\ \hline
			\textbf{RNF05} & El archivo port\'{a}til que maneja el software debe poder ser abierto por un visualizador/editor de hojas de c\'{a}lculo. & Esencial\\ \hline
			\textbf{RNF06} & El software debe desarrollarse utilizando el lenguaje de programaci\'{o}n orientada a objetos C++. & Esencial\\ \hline
		\end{tabulary}
	\end{table}
	
\section{Otra secci\'{o}n}
	\blindtext