\chapter*{Introducci\'{o}n}

\addcontentsline{toc}{chapter}{Introducci\'{o}n}

	La espectroscop\'{i}a de reflectancia difusa es una t\'{e}cnica \'{o}ptica con la cual es  posible estudiar las propiedades bioqu\'{i}micas y las condiciones estructurales de un tejido biol\'{o}gico. Los instrumentos que emplean t\'{e}cnicas como \'{e}sta son de gran ayuda para los dermat\'{o}logos, raz\'{o}n por la cual tales instrumentos han tomado suma importancia en el \'{a}rea de la medicina dermatol\'{o}gica.

El Centro de Investigaciones M\'{e}dicas y Biotecnol\'{o}gicas de la Universidad de Carabobo (CIMBUC) dispone de un espectrofot\'{o}metro de reflexi\'{o}n difusa denominado MiniScan XE Plus. Este instrumento se utiliza a trav\'{e}s del \'{u}nico software disponible para su manejo, designado HunterLab Universal Software.

El HunterLab Universal Software es un software comercial y privativo que fue descontinuado en el a\~{n}o 2008. Su interfaz gr\'{a}fica de usuario est\'{a} en ingl\'{e}s y contiene m\'{a}s funciones de las necesarias para manejar el instrumento en estudio, lo que lo hace poco amigable y dif\'{i}cil de entender por los dermat\'{o}logos.

Esta investigaci\'{o}n se centr\'{o} en desarrollar un software amigable, modificable y extensible para la utilizaci\'{o}n del MiniScan XE Plus, ajust\'{a}ndose a las necesidades de los dermat\'{o}logos, con el fin de sentar una base que permite la realizaci\'{o}n de nuevas investigaciones que conlleven a la implementaci\'{o}n de t\'{e}cnicas que empleen an\'{a}lisis m\'{a}s complejos, y como resultado, diagn\'{o}sticos m\'{a}s completos y diversos sobre patolog\'{i}as dermatol\'{o}gicas presentes en pacientes.

\newpage
\thispagestyle{plain}
	El presente trabajo de investigaci\'{o}n est\'{a} estructurado en cinco cap\'{i}tulos, los cuales son descritos a continuaci\'{o}n.

	El \textbf{cap\'{i}tulo I} describe el contexto de la problem\'{a}tica que motiva el desarrollo de este trabajo de investigaci\'{o}n, se indican los objetivos a cumplir con el mismo y se explican las razones que justifican la necesidad de su realizaci\'{o}n.

	En el \textbf{cap\'{i}tulo II} se presentan los trabajos que anteceden esta investigaci\'{o}n, las observaciones directas realizadas para llevarla a cabo, y se explican las bases te\'{o}ricas que sustentan el desarrollo de las funciones que debe ofrecer el software producto de la misma.

	En el \textbf{cap\'{i}tulo III} se describen las metodolog\'{i}as de investigaci\'{o}n y de desarrollo que se emplearon para planificar, dise\~{n}ar y desarrollar el software propuesto en el presente trabajo de investigaci\'{o}n.

	El \textbf{cap\'{i}tulo IV} muestra los resultados que fueron alcanzados en el desarrollo del presente trabajo de investigaci\'{o}n, incluyendo las fases metodol\'{o}gicas, las tecnolog\'{i}as y los recursos utilizados, la interfaz gr\'{a}fica de usuario del software resultante y las pruebas realizadas.

	Finalmente, en el \textbf{cap\'{i}tulo V} se establecen las conclusiones a las que se lleg\'{o} con el presente trabajo de investigaci\'{o}n, y las recomendaciones para trabajos futuros.

	
	