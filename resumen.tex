\begin{center}
	\membrete
	\vfill
	\textbf{Resumen}
\end{center}

	\noindent
El espectrofot\'{o}metro de reflexi\'{o}n difusa, denominado MiniScan XE Plus, es un instrumento de medici\'{o}n utilizado por el Centro de Investigaciones M\'{e}dicas y Biotecnol\'{o}gicas de la Universidad de Carabobo (CIMBUC), que ayuda a los m\'{e}dicos dermat\'{o}logos a establecer diagn\'{o}sticos sobre patolog\'{i}as en la piel de pacientes, de manera precisa y sin necesidad de realizar biopsias. No obstante, el software comercial disponible para la utilizaci\'{o}n de tal instrumento es poco amigable, dif\'{i}cil de utilizar, e imposible de modificar y extender. La presente investigaci\'{o}n tiene como objetivo desarrollar un software amigable, modificable y extensible, que se ajuste a las necesidades de los dermat\'{o}logos para as\'{i} garantizar un mejor aprovechamiento del instrumento en cuesti\'{o}n.

	\noindent
	\textbf{Palabras claves:} espectrofot\'{o}metro de reflexi\'{o}n difusa, MiniScan XE Plus, an\'{a}lisis bioqu\'{i}mico de la piel, desarrollo de software, mantenibilidad y usabilidad.

\vfill
\begin{minipage}[t]{0.45\textwidth}
	\begin{flushleft}
		\begin{center}
			\autor
		\end{center}
	\end{flushleft}
\end{minipage}
\begin{minipage}[t]{0.45\textwidth}
	\begin{flushright}
		\begin{center}
			\tutores
		\end{center}
	\end{flushright}
\end{minipage}
\vfill