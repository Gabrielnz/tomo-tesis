\renewenvironment{abstract}{
  \vspace*{\fill}
  \begin{center}%
    \bfseries\abstractname
  \end{center}}%

\begin{abstract}
	\noindent
El espectrofot\'{o}metro de reflexi\'{o}n difusa, denominado MiniScan XE Plus, es un instrumento de medici\'{o}n utilizado por el Centro de Investigaciones M\'{e}dicas y Biotecnol\'{o}gicas de la Universidad de Carabobo (CIMBUC), que ayuda a los dermat\'{o}logos a establecer diagn\'{o}sticos sobre patolog\'{i}as en la piel de pacientes, de manera precisa y sin necesidad de realizar biopsias. No obstante, el software comercial disponible para la utilizaci\'{o}n de tal instrumento es poco amigable, dif\'{i}cil de utilizar e imposible de modificar y extender. La presente investigaci\'{o}n tiene como objetivo desarrollar un software amigable, modificable y extensible, que se ajuste a las necesidades de los dermat\'{o}logos y que garantice un mejor aprovechamiento del instrumento en cuesti\'{o}n.

	\noindent
	\textbf{Palabras claves:} espectrofot\'{o}metro, an\'{a}lisis bioqu\'{i}mico de la piel, dermatolog\'{i}a, software privativo, software libre.
	\vfill
\end{abstract}

\vfill

\selectlanguage{english}
\begin{abstract}
	\noindent
The diffuse reflectance spectrophotometer, called MiniScan XE Plus, is a measurement instrument used by the Medical Research and Biotechnology Center at the University of Carabobo (CIMBUC), which helps dermatologists to establish pathologies diagnoses in the skin of patients precisely, without need for biopsy. However, the available commercial software for the use of such an instrument is unfriendly, difficult to use and impossible to modify and extend. This research aims to develop a friendly, modifiable and expandable software that meets the needs of dermatologists and ensures a better use of the instrument itself.

	\noindent
	\textbf{Keywords:} spectrophotometer, biochemical analysis of the skin, dermatology, privative software, open source software.
	\vfill
\end{abstract}

\selectlanguage{spanish}