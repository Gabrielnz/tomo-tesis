\documentclass[12pt, a4paper]{report}
\usepackage[top=3cm, bottom=3cm, left=4cm,right=3cm]{geometry}
\usepackage{setspace} %interlineado
\setlength{\parindent}{7mm} % sangria
\setlength{\parskip}{\baselineskip} %espacio entre parrafos
\usepackage{microtype} %realiza micro-arreglos en los parrafos para que queden mejor justificados
%\renewcommand{\rmdefault}{phv} % Arial
%\renewcommand{\sfdefault}{phv} % Arial
\usepackage{placeins} % ayuda a que las tablas no queden en medio de los textos
\usepackage{tabulary} %tabla que ajusta celdas al texto
\usepackage[font={footnotesize}]{caption} % tamano para la descripcion de las tablas
\captionsetup{labelformat=empty} % elimina el prefijo de los nombres de las tablas
\usepackage{array}
\usepackage{float}
\usepackage{makecell}
\usepackage{graphicx}
\usepackage[spanish,es-tabla]{babel}
\usepackage[authoryear,datebegin]{flexbib}
\bibliographystyle{flexbib}
\usepackage{fancyhdr}%header de los capitulos
\fancyhead{}
\fancyhead[L]{\nouppercase{\leftmark}}

\newcommand{\membrete}{
	\includegraphics[width=0.15\textwidth]{./img/logo-uc.png}~\\[1cm]

	UNIVERSIDAD DE CARABOBO \\	
	Facultad Experimental de Ciencias y Tecnolog\'{i}a\\
	Departamento de Computaci\'{o}n
	\vfill
	\textbf{SOFTWARE PARA EL ESPECTROFOT\'{O}METRO MINISCAN XE PLUS USADO EN EL DIAGN\'{O}STICO DE PATOLOG\'{I}AS DERMATOL\'{O}GICAS EN PACIENTES. CASO DE ESTUDIO: CIMBUC.}
}

%\newcommand{\titulo}{\textbf{SOFTWARE PARA EL ESPECTROFOT\'{O}METRO MINISCAN XE PLUS USADO EN EL DIAGN\'{O}STICO DE PATOLOG\'{I}%AS DERMATOL\'{O}GICAS EN PACIENTES. CASO DE ESTUDIO: CIMBUC.}}

\newcommand{\fecha}{Naguanagua, \today.}

\newcommand{\anchotabla}{14cm}

\newcommand{\altocelda}{2pt}

%\newcommand{\sangriareferencias}{\hspace{7mm}}

%\newcommand{\espacioreferencias}{\vspace{2.5mm}}

\renewcommand\thechapter{\Roman{chapter}}

\renewcommand\thesection{\arabic{chapter}.\arabic{section}}

\selectlanguage{spanish}

\begin{document}
	\begin{titlepage}
	\begin{center}
	\membrete
	\vfill
	\autor
	\vfill
	\tutores
	\vfill
	\fecha
	\end{center}
\end{titlepage}
	\begin{titlepage}
	\begin{center}
	\membrete
	\vfill
	Trabajo Especial de Grado\\
	\titulo
	\vfill
	\textbf{Autor:}\\
	Gabriel A. N\'{u}\~{n}ez N.\\
	\vfill
	\textbf{Tutores:} \\
	Prof. Patricia Guerrero \\
	Prof. Harold Vasquez
	\vfill
	Trabajo Especial de Grado presentado ante la ilustre Universidad de Carabobo, como credencial para optar por el t\'{i}tulo de Licenciado en Computaci\'{o}n.
	\vfill
	\fecha
	\end{center}
\end{titlepage}
	\chapter*{Dedicatoria}

A mi abuela, quien me cri\'{o} desde mis cuatro meses de vida, y a quien llamo mam\'{a} desde que tengo uso de la raz\'{o}n. Este trabajo es para ti, representa la culminaci\'{o}n de una etapa donde estuviste presente d\'{i}a a d\'{i}a, gracias por el \'{a}nimo, los consejos y los valores que me has inculcado, por haberme ense\~{n}ado que con Dios de la mano todo se puede, por haber salido adelante conmigo, por haberme hecho el hombre de bien que soy hoy, por todos los sacrificios que hiciste en silencio, y m\'{a}s que nada por todo el amor que has brindado. Te amo mam\'{a}.

\begin{flushright}
	\textbf{Gabriel A. N\'{u}\~{n}ez N.}
\end{flushright}

	\chapter*{Agradecimientos}

	A Dios, por acompa\~{n}arme en todo momento, por ser mi mejor maestro, y por haberme ayudado a alcanzar esta meta, la culminaci\'{o}n de mi carrera.

	A mi familia, por apoyarme en todo momento, por estar presente para celebrar los logros, compartir la alegr\'{i}a y superar las dificultades.

	A mis amigos, quienes me ayudaron a tomar las cosas con calma, me ayudaron a despejar mi mente innumerables veces, y me dieron el \'{a}nimo necesario para terminar este largo trayecto a tiempo.

	A mi profesora Patricia, por haberme ayudado y aconsejado desde el primero hasta el \'{u}ltimo d\'{i}a de este recorrido, y haberme mostrado que siempre se pueden mejorar las cosas.

	A mi profesor Harold, por saber cu\'{a}ndo exigirme m\'{a}s, ser justo, y mostrarme el nivel de excelencia que puedo alcanzar, a pesar de la distancia.

	Al equipo de profesores e investigadores que hace vida en el CIMBUC, por haberme brindado una experiencia \'{u}nica al trabajar con ellos.

	A la empresa HunterLab, por haber respondido todas mis dudas, y por haberme proporcionado la ayuda que necesit\'{e} e incluso m\'{a}s.


\begin{flushright}
	\textbf{Gabriel A. N\'{u}\~{n}ez N.}
\end{flushright}

	\renewenvironment{abstract}{
  \vspace*{\fill}
  \begin{center}%
    \bfseries\abstractname
  \end{center}}%

\begin{abstract}
	\noindent
El espectrofot\'{o}metro de reflexi\'{o}n difusa, denominado MiniScan XE Plus, es un instrumento de medici\'{o}n utilizado por el Centro de Investigaciones M\'{e}dicas y Biotecnol\'{o}gicas de la Universidad de Carabobo (CIMBUC), que ayuda a los dermat\'{o}logos a establecer diagn\'{o}sticos sobre patolog\'{i}as en la piel de pacientes de manera precisa, y sin necesidad de realizar biopsias. No obstante, el software comercial disponible para la utilizaci\'{o}n de tal instrumento es poco amigable, dificil de utilizar e imposible de modificar y extender. La presente investigaci\'{o}n tiene como objetivo desarrollar un software amigable, modificable y extensible, que se ajuste a las necesidades de los dermat\'{o}logos y que garantice un mejor aprovechamiento del instrumento en cuesti\'{o}n.

	\noindent
	\textbf{Palabras claves:} espectrofot\'{o}metro, an\'{a}lisis bioqu\'{i}mico de la piel, biopsia, ingenier\'{i}a biom\'{e}dica, software privativo.
	\vfill
\end{abstract}

\vfill

\selectlanguage{english}
\begin{abstract}
	\noindent
The diffuse reflectance spectrophotometer, called MiniScan XE Plus, is a measurement instrument used by the Medical Research and Biotechnology Center at the University of Carabobo (CIMBUC), which helps dermatologists to establish pathologies diagnoses in the skin of patients precisely, without need for biopsy. However, the available commercial software for the use of such an instrument is unfriendly, difficult to use and impossible to modify and extend. This research aims to develop a friendly, modifiable and expandable software that meets the needs of dermatologists and ensures a better use of the instrument itself.

	\noindent
	\textbf{Keywords:} spectrophotometer, biochemical analysis of the skin, biopsy, biomedical engineering, privative software.
	\vfill
\end{abstract}

\selectlanguage{spanish}
	\pagestyle{fancy}
	\chapter*{Introducci\'{o}n}

\addcontentsline{toc}{chapter}{Introducci\'{o}n}

	La espectroscop\'{i}a de reflectancia difusa es una t\'{e}cnica \'{o}ptica con la cual es  posible estudiar las propiedades bioqu\'{i}micas y las condiciones estructurales de un tejido biol\'{o}gico. Los instrumentos que emplean t\'{e}cnicas como \'{e}sta son de gran ayuda para los dermat\'{o}logos, raz\'{o}n por la cual tales instrumentos han tomado suma importancia en el \'{a}rea de la medicina dermatol\'{o}gica.

El Centro de Investigaciones M\'{e}dicas y Biotecnol\'{o}gicas de la Universidad de Carabobo (CIMBUC) dispone de un espectrofot\'{o}metro de reflexi\'{o}n difusa denominado MiniScan XE Plus. Este instrumento se utiliza a trav\'{e}s del \'{u}nico software disponible para su manejo, designado HunterLab Universal Software.

El HunterLab Universal Software es un software comercial y privativo que fue descontinuado en el a\~{n}o 2008. Su interfaz gr\'{a}fica de usuario est\'{a} en ingl\'{e}s y contiene m\'{a}s funciones de las necesarias para manejar el instrumento en estudio, lo que lo hace poco amigable y dif\'{i}cil de entender por los dermat\'{o}logos.

Esta investigaci\'{o}n se centr\'{o} en desarrollar un software amigable, modificable y extensible para la utilizaci\'{o}n del MiniScan XE Plus, ajustandose a las necesidades de los dermat\'{o}logos, a fin de sentar una base que permite la realizaci\'{o}n de nuevas investigaciones que conlleven a la implementaci\'{o}n de t\'{e}cnicas que empleen an\'{a}lisis m\'{a}s complejos, y como resultado, diagn\'{o}sticos m\'{a}s completos y diversos sobre patolog\'{i}as dermatol\'{o}gicas presentes en pacientes.

\newpage
\thispagestyle{plain}
	El presente trabajo de investigaci\'{o}n est\'{a} estructurado en cinco cap\'{i}tulos, los cuales son descritos a continuaci\'{o}n.

	El \textbf{cap\'{i}tulo I} describe el contexto de la problem\'{a}tica que motiva el desarrollo de este trabajo de investigaci\'{o}n, se indican los objetivos a cumplir con el mismo y se explican las razones que justifican la necesidad de su realizaci\'{o}n.

	En el \textbf{cap\'{i}tulo II} se presentan los trabajos que anteceden esta investigaci\'{o}n, las observaciones directas realizadas para llevarla a cabo, y se explican las bases te\'{o}ricas que sustentan el desarrollo de las funciones que debe ofrecer el software producto de la misma.

	En el \textbf{cap\'{i}tulo III} se describen la metodolog\'{i}a de investigaci\'{o}n y la metodolog\'{i}a de desarrollo que se emplearon para planificar, dise\~{n}ar y desarrollar el software propuesto en el presente trabajo de investigaci\'{o}n.

	El \textbf{cap\'{i}tulo IV} muestra los resultados que fueron alcanzados en el desarrollo del presente trabajo de investigaci\'{o}n, incluyendo las fases metodol\'{o}gicas, las tecnolog\'{i}as y los recursos utilizados, la interfaz gr\'{a}fica de usuario del software resultante y las pruebas realizadas.

	Finalmente, en el \textbf{cap\'{i}tulo V} se establecen las conclusiones a las que se lleg\'{o} con el presente trabajo de investigaci\'{o}n, y las recomendaciones consideradas a tomar para trabajos futuros, tomando en cuenta dichas conclusiones.

	
	
	\newpage
	\spacing{1.5} % interlineado
	\chapter{\label{cap:1}El Problema}

\begin{chapquote}{Richard Stallman}
<<Software privativo significa que priva a los usuarios de su libertad.>>
\end{chapquote}

	\section{Planteamiento del Problema}	
\cite{Bersha} indica que durante el diagn\'{o}stico de enfermedades de la piel, la observaci\'{o}n cuidadosa y la evaluaci\'{o}n visual del \'{a}rea sospechada es siempre el primer paso, y el m\'{a}s importante. Esto es seguido generalmente por una escisi\'{o}n o biopsia por punci\'{o}n, en la que se extrae una muestra de tejido de la piel para un an\'{a}lisis microsc\'{o}pico. La observaci\'{o}n visual suele ser subjetiva y los pacientes a menudo se someten a cicatrices y dolor durante la biopsia. Por otro lado, las t\'{e}cnicas \'{o}pticas son por lo general no invasivas y sus resultados son a menudo objetivos. Durante el diagn\'{o}stico no invasivo no se crea ninguna ruptura en la piel, y los pacientes no se someten al dolor ni a cicatrices durante el tratamiento.

Los avances tecnol\'{o}gicos en la actualidad permiten emplear t\'{e}cnicas de \'{o}ptica con la capacidad de estudiar  las propiedades estructurales y bioqu\'{i}micas del tejido biol\'{o}gico de manera precisa y no invasiva. Los instrumentos que emplean tales t\'{e}cnicas son de gran ayuda para los m\'{e}dicos dermat\'{o}logos, raz\'{o}n por la cual han tomado suma importancia en el \'{a}rea m\'{e}dica dermatol\'{o}gica.

Hoy d\'{i}a existen diferentes tipos de estudios \'{o}pticos in-situ, in-vivo e invitro del tejido biol\'{o}gico, como la Espectroscop\'{i}a de Reflectancia Difusa (ERD). \cite{Perez-Gallardo} asegura que con esta t\'{e}cnica es  posible estudiar las propiedades bioqu\'{i}micas y las condiciones estructurales de un tejido biol\'{o}gico, analizando la interacci\'{o}n luz-tejido de una manera no invasiva.

En este sentido, el Centro de Investigaciones M\'{e}dicas y Biotecnol\'{o}gicas de la Universidad de Carabobo (CIMBUC) dispone de un espectrofot\'{o}metro de reflexi\'{o}n difusa, denominado MiniScan XE Plus, creado por la empresa HunterLab. Esta empresa lo describe como un instrumento utilizado para medir la transmisi\'{o}n y/o reflectancia de espec\'{i}menes, como una funci\'{o}n de longitud de onda, que aplica la t\'{e}cnica de ERD. 

Ahora bien, el CIMBUC hace uso de este instrumento a trav\'{e}s del software disponible para su utilizaci\'{o}n, designado HunterLab Universal Software (HLUS), que es un software comercial y privativo de 16 bits, dise\~{n}ado para el sistema operativo Microsoft Windows versi\'{o}n 3.x, con la posibilidad de ejecutarse en Windows 95, Windows 2000 y Windows XP. Este software contiene funciones que abarcan la utilizaci\'{o}n del MiniScan XE Plus, y de otros instrumentos ofrecidos por HunterLab; adem\'{a}s fue descontinuado en el a\~{n}o 2008. La interfaz gr\'{a}fica de usuario de este software est\'{a} en ingl\'{e}s. Por \'{u}ltimo, los resultados que genera este software no poseen el formato de gesti\'{o}n de informaci\'{o}n de pacientes con el que trabajan los dermat\'{o}logos del CIMBUC.

Tomando en cuenta lo mencionado anteriormente, se tiene que el HLUS es un software privativo y que est\'{a} descontinuado, por lo tanto no existe la posibilidad de modificarlo ni extenderlo; ofrece funciones ajenas al uso exclusivo del MiniScan XE Plus, causando que la interfaz gr\'{a}fica de usuario contenga m\'{a}s opciones de las necesarias para manejar tal instrumento. Asimismo, como consecuencia de que la interfaz gr\'{a}fica de usuario est\'{e} en ingl\'{e}s, esta es dif\'{i}cil de entender por los dermat\'{o}logos. Sumado al hecho de que los resultados generados por este software no poseen el formato con el que trabajan los dermat\'{o}logos, haciendo necesario su traspaso manual, lo que produce a una ralentizaci\'{o}n en las consultas con pacientes. Todo esto conlleva a que los dermat\'{o}logos requieran de asistencia t\'{e}cnica especializada para la debida utilizaci\'{o}n de dicho software.

De lo antedicho se desprende que, el HLUS posee una interfaz gr\'{a}fica de usuario poco amigable, y el costo del tiempo de capacitaci\'{o}n para su uso correcto podr\'{i}a ser alto. Este software no podr\'{a} modificarse ni extenderse, por lo tanto no se fomentar\'{a} el uso del instrumento en cuesti\'{o}n en el campo m\'{e}dico (p\'{u}blico o privado), disminuyendo su potencial. Por \'{u}ltimo, tampoco se fomentar\'{a} el desarrollo de nuevas aplicaciones que utilicen sus resultados como insumo, sosegando as\'{i} la posibilidad de realizar an\'{a}lisis m\'{a}s complejos, y de proveer a los dermat\'{o}logos con resultados que les permitan establecer diagn\'{o}sticos m\'{a}s completos.

	\section{Justificaci\'{o}n}

Con respecto a software de calidad, \cite{Sommerville} explica lo siguiente: as\'{i} como los servicios que proveen, los productos de software tienen cierto n\'{u}mero de atributos asociados que reflejan su calidad. Estos atributos no est\'{a}n directamente relacionados con lo que hace el software. M\'{a}s bien, reflejan su comportamiento durante su ejecuci\'{o}n, en la estructura y organizaci\'{o}n del programa fuente, y en la documentaci\'{o}n asociada.

El conjunto espec\'{i}fico de atributos que se espera de un software depende obviamente de su aplicaci\'{o}n. Esto se generaliza en el conjunto de atributos que se muestran en la Tabla 1, la cual contiene las caracter\'{i}sticas esenciales de un software de calidad.

		\begin{table}[htb]
			\caption{\textbf{Tabla 1.} \textit{Atributos esenciales de un buen software} (Fuente: Sommerville, 2005).}
			\label{tabla_1}
			\centering
			\setlength{\extrarowheight}{3.5pt}
			\begin{tabulary}{15cm}{|c|J|}
				\hline
				\thead{\textbf{Caracter\'{i}stica}} & \thead{\textbf{Descripci\'{o}n}}\\ \hline
			\textbf{Mantenibilidad} & El software debe describirse de tal forma que pueda evolucionar  para cumplir las necesidades de cambio de los 					clientes. Este es un atributo cr\'{i}tico, debido a que el cambio en el software es una consecuencia inevitable de un cambio en el entorno de negocios.\\ \hline
			\textbf{Confiabilidad} & La confiabilidad del software tiene un gran n\'{u}mero de caracter\'{i}sticas, incluyendo la fiabilidad, protecci\'{o}n y seguridad. El software confiable no debe causar da\~{n}os f\'{i}sicos o econ\'{o}micos en el caso de una falla del sistema.\\ \hline
			\textbf{Eficiencia} & El software no debe hacer que se malgasten los recursos del sistema, como la memoria y los ciclos de procesamiento. Por lo tanto, la eficiencia incluye tiempos de respuesta y de procesamiento, utilizaci\'{o}n de la memoria, etc\'{e}tera.\\ \hline
			\textbf{Usabilidad} & El software debe ser f\'{a}cil de utilizar, sin esfuerzo adicional por el usuario para quien est\'{a} dise\~{n}ado. Esto significa que debe tener una interfaz gr\'{a}fica de usuario apropiada y una documentaci\'{o}n adecuada.\\ \hline
			\end{tabulary}
		\end{table}
		\FloatBarrier %you shall not pass table!!

Debido a que el HLUS es privativo, el CIMBUC no dispone de su c\'{o}digo fuente, de manera que este software no puede ser cambiado ni adaptarse a necesidades espec\'{i}ficas, y, por lo tanto, no posee el primer atributo esencial para un software de calidad: la mantenibilidad. Por la misma raz\'{o}n, no se puede determinar con certidumbre el segundo atributo: la confiabilidad (madurez del software y tolerancia a fallas); adem\'{a}s de que no se puede evaluar completamente el nivel de protecci\'{o}n y seguridad del mismo. Por \'{u}ltimo, la usabilidad de este software es baja, ya que la interfaz gr\'{a}fica de usuario es poco amigable. Por estas razones, se desarroll\'{o} un software que cumpliese con los atributos esenciales que debe poseer un buen software.

\cite{Sommerville} se\~{n}ala que un dise\~{n}o cuidadoso de la interfaz gr\'{a}fica de usuario es parte fundamental del proceso de dise\~{n}o general del software. Si un software debe alcanzar su potencial m\'{a}ximo, es fundamental que su interfaz gr\'{a}fica de usuario sea dise\~{n}ada para ajustarse a las habilidades, experiencia y expectativas de sus usuarios previstos. Un buen dise\~{n}o de la interfaz gr\'{a}fica de usuario es cr\'{i}tico para la confiabilidad del software. Muchos de los llamados ``errores de usuario'' son causados por el hecho de que las interfaces gr\'{a}ficas de usuario no consideran las habilidades de los usuarios reales y su entorno de trabajo.

El dise\~{n}o de la interfaz gr\'{a}fica de usuario del HLUS es la principal raz\'{o}n por la cual los dermat\'{o}logos requieren de personal t\'{e}cnico que los asista al momento de utilizarlo. Esto porque dicha interfaz est\'{a} en idioma ingl\'{e}s, contiene funcionalidades que no son necesarias para la utilizaci\'{o}n de Espectrofot\'{o}metro y no proporciona el formato con el que trabajan los dermat\'{o}logos, lo que dificulta la utilizaci\'{o}n de dicha interfaz. Por estas razones los dermat\'{o}logos perciben este software comercial como no intuitivo, ni auto descriptivo ni amigable, temiendo cometer errores al utilizarlo por su propia cuenta y generar resultados err\'{o}neos, poniendo en riesgo el diagn\'{o}stico, y en consecuencia, la salud de los pacientes en consulta.

En conclusi\'{o}n, siguiendo los lineamientos de dise\~{n}o y calidad del software que se consideraron pertinentes, se desarroll\'{o} un software amigable, modificable y extensible, el cual ofrece las funciones que necesitan los dermat\'{o}logos para establecer diagn\'{o}sticos, emplea el formato de historia m\'{e}dica con el que trabajan, permite la exportaci\'{o}n de los resultados a un formato de archivo portable; por \'{u}ltimo y no menos importante, se cre\'{o} una base sobre la cual se prodr\'{a}n trabajar proyectos futuros que necesiten utilizar los resultados de este software como insumo.
	\newpage

	\section{Objetivos de la Investigaci\'{o}n}
En la siguiente secci\'{o}n se especifican los objetivos del trabajo, distinguiendo entre el objetivo general y los objetivos espec\'{i}ficos.
		\subsection{Objetivo General}
	Desarrollar un software para el Espectrofot\'{o}metro ``MiniScan XE Plus'', usado en el diagn\'{o}stico de patolog\'{i}as dermatol\'{o}gicas en pacientes, tomando como caso de estudio el CIMBUC.
		\subsection{Objetivos Espec\'{i}ficos}
			\begin{itemize}
				\item Investigar el estado del arte referente a las caracter\'{i}sticas de software para Espectrofot\'{o}metros de reflexi\'{o}n difusa, dise\~{n}o y calidad de software.
				\item Seleccionar una metodolog\'{i}a que gu\'{i}e el dise\~{n}o y desarrollo del nuevo software para el Espectrofot\'{o}metro ``MiniScan XE Plus''.
				\item Dise\~{n}ar el nuevo software siguiendo la metodolog\'{i}a seleccionada.
				\item Desarrollar el nuevo software, siguiendo la metodolog\'{i}a seleccionada.
				\item Dise\~{n}ar las pruebas para el nuevo software.
				\item Elaborar el manual de usuario del nuevo software.
			\end{itemize}
	\newpage
	\chapter{Marco Te\'{o}rico}

Este cap\'{i}tulo presenta los trabajos relacionados, la observaci\'{o}n directa y las bases te\'{o}ricas que conforman los antecedentes de la investigaci\'{o}n que sustentan el desarrollo del presente trabajo de investigaci\'{o}n.

	\section{Antecedentes}
	
		\subsection{Antecedentes de la investigaci\'{o}n}
			
			Primero se tiene el art\'{i}culo cient\'{i}fico titulado \textit{<<Comparing Quantitative Measures of Erythema, Pigmentation and Skin Response using Reflectometry>>}, realizado por \cite{Wagner}, en la Universidad del Estado de Pensilvania, Estados Unidos, y publicado por Pigment Cell Res. En este art\'{i}culo se obtiene el \'{i}ndice de eritema, que es utilizado para determinar el nivel inflamatorio de la epidermis de un paciente. El m\'{e}todo utilizado en este art\'{i}culo para su obtenci\'{o}n fue implementado en el nuevo software.
			
			Por	\'{u}ltimo est\'{a} el art\'{i}culo cient\'{i}fico titulado \textit{<<Recuperaci\'{o}n del Coeficiente de Absorci\'{o}n de la Epidermis en la Piel Humana>>}, realizado por \cite{Narea}, en la Universidad de Carabobo, Venezuela, y publicado por la Sociedad Espa\~{n}ola de \'{O}ptica. En este art\'{i}culo se determina el coeficiente de absorci\'{o}n, que es un par\'{a}metro \'{o}ptico asociado a la piel, el cual indica el nivel de concentraci\'{o}n de melanina presente en la epidermis de un paciente. La t\'{e}cnica empleada en dicho art\'{i}culo para calcular este coeficiente fue implementada en el nuevo software.

	\subsection{Observaci\'{o}n directa}
		
			El \textit{<<HunterLab Universal Software>>}, es un software comercial y privativo de 16 bits dise\~{n}ado para el sistema operativo Microsoft Windows versi\'{o}n 3.x, con la posibilidad de ejecutarse en Windows 95, Windows 98, Windows 2000 y Windows XP. Fue creado para la utilizaci\'{o}n del MiniScan XE Plus, adem\'{a}s de otros instrumentos de la empresa HunterLab, y descontinuado en el a\~{n}o 2008. Este software dispone de algunas de las funciones que fueron desarrolladas en el nuevo software, raz\'{o}n por la cual es una referencia importante de observaci\'{o}n.

			El archivo denominado \textit{<<MSXE + OCX>>}, es una hoja de c\'{a}lculo habilitada para la ejecuci\'{o}n de macroinstrucciones de Microsoft Excel, que fue proporcionada por el personal de soporte t\'{e}cnico de HunterLab como un ejemplo para utilizar el MiniScan XE Plus, empleando el uso de un kit de control denominado MiniScan XE Plus OCX Kit \textit{(MSXE.ocx)}. Este kit fue dise\~{n}ado por la empresa HunterLab para dar acceso a las caracteristicas comunmente utilizadas por dicho instrumento. El c\'{o}digo contenido en la hoja de c\'{a}lculo se emple\'{o} como referencia para el manejo del kit \textit{MSXE.ocx}.

	\section{Bases te\'{o}ricas}
	
	\subsection{Espectroscop\'{i}a de reflectancia difusa}
	
		La ERD (espectroscop\'{i}a de reflectancia difusa) es una t\'{e}cnica con la cual se puede estudiar tejido biol\'{o}gico. En el campo de las aplicaciones biom\'{e}dicas resulta \'{u}til para prop\'{o}sitos de diagn\'{o}stico, ya que se pueden estudiar tejidos de manera no invasiva, tambi\'{e}n ha demostrado ser una t\'{e}cnica de gran utilidad en aplicaciones de diagn\'{o}stico en varias situaciones modernas (P\'{e}rez, 2012).
		
		Para llevar a cabo una medici\'{o}n con ERD se requiere hacer incidir la luz de una fuente, cuyo espectro de emisi\'{o}n sea conocido, sobre el tejido que se quiere estudiar. La luz que logra propagarse en el tejido y que es re-emitida por este hacia la superficie irradiada, ser\'{a} capturada por alg\'{u}n dispositivo fotosensible (en el caso de esta investigaci\'{o}n, el MiniScan XE Plus), para posteriormente ser comparada con la luz incidente o espectro de referencia, y as\'{i} poder determinar qu\'{e} tanto cambi\'{o} dicho espectro despu\'{e}s de haber interactuado con el tejido.
		
		Normalmente la reflectancia difusa $R(\lambda)$ es multiplicada por un factor de 100 por los dispositivos fotosensibles, para representarla en forma de curva en una escala del 0\% al 100\%, a lo largo de puntos discretos que representan las longitudes de onda con las que trabaja dicho dispositivo; tal es el caso del MiniScan XE Plus.
	
	\subsection{Absorbancia aparente}
	
	Seg\'{u}n el \textit{Random House Kernerman Webster's College Dictionary} (2010), el espectro de absorci\'{o}n es la radiaci\'{o}n electromagn\'{e}tica en ciertas longitudes de onda que atraviesa un medio y que es absorbida por el mismo. En cierto modo, es el opuesto del espectro de reflectancia, es decir, es la luz que logra propagarse en el tejido y que no es re-emitida por este hacia la superficie irradiada. Por lo tanto, la absorbancia aparente es la luz que est\'{a} siendo absorbida aparentemente por el tejido en estudio.
	
	Como la absorbancia aparente es toda la luz que no est\'{a} siendo reflejada de vuelta al MiniScan XE Plus al emitir luz sobre el tejido en estudio, esta se puede calcular de la siguiente manera: $A(\lambda) = 100 - R(\lambda)$, y se puede representar en forma de curva, de la misma manera que la reflectancia difusa.

	\subsection{Iluminante est\'{a}ndar D65}
		
		El tipo de luz bajo el cual se observa un objeto puede afectar su apariencia. Para cuantificar estas fuentes de luz blanca, la \textit{Commission Internationale de l'Eclairage} (CIE) desarroll\'{o} iluminantes est\'{a}ndar para la medici\'{o}n del color.
		\cite{HunterLab} define el iluminante como una tabla cuantificable de n\'{u}meros que representan la energ\'{i}a relativa en comparaci\'{o}n con la longitud de onda de una fuente de luz. 
		
		El iluminante est\'{a}ndar D65, seg\'{u}n es descrito por la \cite{CIE}, tiene el prop\'{o}sito de representar la luz de d\'{i}a promedio, y tiene una temperatura de color correlacionada de aproximadamente 6500 K\degree. Los valores num\'{e}ricos que representan este iluminante, en contraste con las longitudes de onda en nan\'{o}metros (nm), se muestran en la tabla 2.1.
	
		\begin{table}[h]
		\small
		\caption[Valores del iluminante D65]{\textit{Valores del iluminante D65} (Fuente: CIE, 2004).}
		\centering
		\setlength{\extrarowheight}{\altocelda}
		\begin{tabulary}{\anchotabla}{|c|c|}
			\hline
			Longitud de onda $\lambda$ & Funci\'{o}n $S(\lambda)$\\ \hline
			400 nm & 82.7549\\ \hline
			410 nm & 91.4860\\ \hline
			420 nm & 93.4318\\ \hline
			430 nm & 86.6823\\ \hline
			440 nm & 104.865\\ \hline
			450 nm & 117.008\\ \hline
			460 nm & 117.812\\ \hline
			470 nm & 114.861\\ \hline
			480 nm & 115.923\\ \hline
			490 nm & 108.811\\ \hline
			500 nm & 109.354\\ \hline
			510 nm & 107.802\\ \hline
			520 nm & 104.790\\ \hline
			530 nm & 107.689\\ \hline
			540 nm & 104.405\\ \hline
			550 nm & 104.046\\ \hline
			560 nm & 100.000\\ \hline
			570 nm & 96.3342\\ \hline
			580 nm & 95.7880\\ \hline
			590 nm & 88.6856\\ \hline
			600 nm & 90.0062\\ \hline
			610 nm & 89.5991\\ \hline
			620 nm & 87.6987\\ \hline
			630 nm & 83.2886\\ \hline
			640 nm & 83.6992\\ \hline
			650 nm & 80.0268\\ \hline
			660 nm & 80.2146\\ \hline
			670 nm & 82.2778\\ \hline
			680 nm & 78.2842\\ \hline
			690 nm & 69.7213\\ \hline
			700 nm & 71.6091\\ \hline
		\end{tabulary}
	\end{table}
	
	\subsection{Observador est\'{a}ndar de 10\degree}
	
		\cite{HunterLab-applications} describe que en la observaci\'{o}n visual, el observador es el ojo humano que recibe la luz reflejada desde o a trav\'{e}s de un objeto, y el cerebro el cual percibe la visi\'{o}n. Debido a que los humanos perciben el color y la apariencia de distintas formas, subjetivamente, se han hecho intentos para estandarizar el observador humano como una representaci\'{o}n de lo que una persona promedio ve u observa.
		
		En 1964, se desarroll\'{o} la funci\'{o}n del observador est\'{a}ndar CIE de 10\degree, denominado de esa manera debido a que los experimentos llevados a cabo para establecer dicho est\'{a}ndar involucraron a sujetos que juzgaran colores, mientras observaban a trav\'{e}s de un agujero que les permit\'{i}a tener un campo de visi\'{o}n de 10\degree. Este observador est\'{a}ndar, en la forma de una funci\'{o}n matem\'{a}tica de la respuesta humana a cada longitud de onda de luz, es utilizado en c\'{a}lculos del color. 
		
		Los valores num\'{e}ricos de las funciones de coincidencia de colores que representan este est\'{a}ndar, en contraste con las longitudes de onda en nan\'{o}metros (nm), se muestran en la tabla 2.2.
	
	\begin{table}[h]
		\small
		\caption[Valores del observador de 10\degree]{\textit{Valores del observador de 10\degree} (Fuente: CIE, 2004).}
		\centering
		\setlength{\extrarowheight}{\altocelda}
		\begin{tabulary}{\anchotabla}{|c|c|c|c|}
			\hline
			Longitud de onda $\lambda$ & Funci\'{o}n $\overline{x}(\lambda)$ & Funci\'{o}n $\overline{y}(\lambda)$ & Funci\'{o}n $\overline{z}(\lambda)$\\ \hline
			400 nm & 0.019110 & 0.002004 & 0.086011\\ \hline
			410 nm & 0.084736 & 0.008756 & 0.389366\\ \hline
			420 nm & 0.204492 & 0.021391 & 0.972542\\ \hline
			430 nm & 0.314679 & 0.038676 & 1.553480\\ \hline
			440 nm & 0.383734 & 0.062077 & 1.967280\\ \hline
			450 nm & 0.370702 & 0.089456 & 1.994800\\ \hline
			460 nm & 0.302273 & 0.128201 & 1.745370\\ \hline
			470 nm & 0.195618 & 0.185190 & 1.317560\\ \hline
			480 nm & 0.080507 & 0.253589 & 0.772125\\ \hline
			490 nm & 0.016172 & 0.339133 & 0.415254\\ \hline
			500 nm & 0.003816 & 0.460777 & 0.218502\\ \hline
			510 nm & 0.037465 & 0.606741 & 0.112044\\ \hline
			520 nm & 0.117749 & 0.761757 & 0.060709\\ \hline
			530 nm & 0.236491 & 0.875211 & 0.030451\\ \hline
			540 nm & 0.376772 & 0.961988 & 0.013676\\ \hline
			550 nm & 0.529826 & 0.991761 & 0.003988\\ \hline
			560 nm & 0.705224 & 0.997340 & 0.000000\\ \hline
			570 nm & 0.705224 & 0.955552 & 0.000000\\ \hline
			580 nm & 1.014160 & 0.868934 & 0.000000\\ \hline
			590 nm & 1.118520 & 0.777405 & 0.000000\\ \hline
			600 nm & 1.123990 & 0.658341 & 0.000000\\ \hline
			610 nm & 1.030480 & 0.527963 & 0.000000\\ \hline
			620 nm & 0.856297 & 0.398057 & 0.000000\\ \hline
			630 nm & 0.647467 & 0.283493 & 0.000000\\ \hline
			640 nm & 0.431567 & 0.179828 & 0.000000\\ \hline
			650 nm & 0.268329 & 0.107633 & 0.000000\\ \hline
			660 nm & 0.152568 & 0.060281 & 0.000000\\ \hline
			670 nm & 0.081261 & 0.031800 & 0.000000\\ \hline
			680 nm & 0.040851 & 0.015905 & 0.000000\\ \hline
			690 nm & 0.019941 & 0.007749 & 0.000000\\ \hline
			700 nm & 0.009577 & 0.003718 & 0.000000\\ \hline
		\end{tabulary}
		
	\end{table}
	
	\subsection{Coordenadas de cromaticidad CIE xyz}
	
		La sensaci\'{o}n de luz es producida por radiaci\'{o}n electromagn\'{e}tica visible, que cae dentro de los l\'{i}mites de longitud de onda de 380 nan\'{o}metros y 780 nan\'{o}metros. La radiaci\'{o}n proveniente de la regi\'{o}n de longitud de onda corta de la misma produce usualmente la sensaci\'{o}n de luz azul, la radiaci\'{o}n con longitudes de onda entre 520 nan\'{o}metros y 550 nan\'{o}metros son vistas como luz verde, y por encima de alrededor de los 650 nan\'{o}metros se percibe usualmente la luz de color rojo. Estos l\'{i}mites no est\'{a}n bien definidos, y la percepci\'{o}n actual depende fuertemente del estado de adaptaci\'{o}n del ojo y del est\'{i}mulo de luz que rodea el objeto o tejido en estudio (Schanda, 2007).
		
		La CIE defini\'{o} un est\'{a}ndar para calcular los valores de estos est\'{i}mulos, denominandolos valores triest\'{i}mulo XYZ \'{o} sistema tricrom\'{a}tico CIE XYZ. Tomando en cuenta el rango de longitudes de onda con el que opera el \mbox{MiniScan} XE Plus, estos valores son calculados utilizando las siguientes f\'{o}rmulas.
		
		$$X = k \int_{400 \text{ nm}}^{700 \text{ nm}} R(\lambda) S(\lambda) \overline{x}(\lambda)d\lambda$$
		
		$$Y = k \int_{400 \text{ nm}}^{700 \text{ nm}} R(\lambda) S(\lambda) \overline{y}(\lambda)d\lambda$$
		
		$$Z = k \int_{400 \text{ nm}}^{700 \text{ nm}} R(\lambda) S(\lambda) \overline{z}(\lambda)d\lambda$$
		
		En donde $R(\lambda)$ es el factor de reflectancia difusa, $S(\lambda)$ es la distribuci\'{o}n de energ\'{i}a espectral relativa de un iluminante est\'{a}ndar, en este caso del iluminante D65 (v\'{e}ase la tabla 2.1), $\overline{x}(\lambda)$, $\overline{y}(\lambda)$ y $\overline{z}(\lambda)$ son las funciones de correspondencia del color, dado el observador est\'{a}ndar CIE de 10\degree (v\'{e}ase la tabla 2.2), y por \'{u}ltimo, $k$ es una constante que se calcula con la f\'{o}rmula mostrada a continuaci\'{o}n.
		
		$$k = \frac{100}{\sum_{\lambda} \overline{y}(\lambda)d\lambda}$$
		
		De acuerdo con la recomendaci\'{o}n de CIE, la integraci\'{o}n puede llevarse a cabo por sumatoria num\'{e}rica a intervalos de longitud de onda, $\bigtriangleup\lambda$, equivalentes a 10 nan\'{o}metros, para el caso del MiniScan XE Plus.
		
		$$X = k \sum_{\lambda} R(\lambda) S(\lambda) \overline{x}(\lambda)\bigtriangleup\lambda$$
		
		$$Y = k \sum_{\lambda} R(\lambda) S(\lambda) \overline{y}(\lambda)\bigtriangleup\lambda$$
		
		$$Z = k \sum_{\lambda} R(\lambda) S(\lambda) \overline{z}(\lambda)\bigtriangleup\lambda$$
		
		Ahora bien, el est\'{i}mulo de un color se puede describir completamente por los tres valores triest\'{i}mulo, pero esta descripci\'{o}n no es muy f\'{a}cilmente concebible. Es dif\'{i}cil imaginar un est\'{i}mulo si solamente se dan sus valores triest\'{i}mulo, y frecuentemente no se buscan los valores absolutos de los mismos. En tales casos se pueden utilizar las coordenadas de cromaticidad xyz.
		
		Finalmente, las coordenadas de cromaticidad xyz se definen con las f\'{o}rmulas mostradas a continuaci\'{o}n.
		
		$$x = \frac{X}{X+Y+Z}$$
		
		$$y = \frac{Y}{X+Y+Z}$$
		
		$$z = \frac{Z}{X+Y+Z}$$
	
	\subsection{Espacio del color CIELAB}
	
		Seg\'{u}n \cite{Schanda}, los est\'{i}mulos del color son tridimensionales, y la solicitud de extender el espacio del color uniforme a un espacio tridimensional ya hab\'{i}a sido expresada en los a\~{n}os 60. En 1976 se acept\'{o} la recomendaci\'{o}n para el diagrama de espacio del color uniforme CIELAB (L*a*b*), que no es m\'{a}s que un sistema para transformar las coordenadas de cromaticidad CIE xyz. Este espacio del color est\'{a} definido por las ecuaciones descritas a continuaci\'{o}n.
		
		$$L^* = 116f(Y/Y_n) - 16$$
		
		$$a^* = 500[f(X/X_n) - f(Y/Y_n)]$$
		
		$$b^* = 200[f(Y/Y_n) - f(Z/Z_n)]$$
		
		$$\text{Donde } f(X/X_n) = (X/X_n)^{1/3} \text{ si } (X/X_n) > (24/116)^3$$
		$$f(X/X_n) = (841/108)(X/X_n) + 16/116 \text{ si } (X/X_n) \leq (24/116)^3$$
		
		$$\text{Y donde } f(Y/Y_n) = (Y/Y_n)^{1/3} \text{ si } (Y/Y_n) > (24/116)^3$$
		$$f(Y/Y_n) = (841/108)(Y/Y_n) + 16/116 \text{ si } (Y/Y_n) \leq (24/116)^3$$
		
		$$\text{Y donde } f(Z/Z_n) = (Z/Z_n)^{1/3} \text{ si } (Z/Z_n) > (24/116)^3$$
		$$f(Z/Z_n) = (841/108)(Z/Z_n) + 16/116 \text{ si } (Z/Z_n) \leq (24/116)^3$$
		
		En donde $X$, $Y$, $Z$ son los valores triest\'{i}mulo del est\'{i}mulo del color considerado del objeto o tejido en estudio y $X_n$, $Y_n$, $Z_n$ son los valores triest\'{i}mulo de la fuente de luz. Para el caso del iluminante est\'{a}ndar D65, y tomando en cuenta el observador est\'{a}ndar de 10\degree, los valores de $X_n$, $Y_n$, $Z_n$ son $X_n = 94.81$ $Y_n = 100.00$ y $Z_n = 107.32$.

	\subsection{Coeficiente de absorci\'{o}n}
	
	\subsection{Coeficiente de esparcimiento}
	
	\subsection{\'{I}ndice de eritema}
	\newpage
	\chapter{\label{cap:3}Marco Metodol\'{o}gico}

	\section{Metodolog\'{i}a Investigaci\'{o}n-Acci\'{o}n}
	La Investigaci\'{o}n-Acci\'{o}n se orienta a la acci\'{o}n y al cambio, a la focalizaci\'{o}n de un problema y posee un modelo de proceso ``org\'{a}nico'' que engloba tanto etapas sistem\'{a}ticas como iterativas, ayudando a resolver as\'{i} problemas pr\'{a}cticos y a expandir el conocimiento cient\'{i}fico.

	Esta metodolog\'{i}a tiene una doble finalidad: generar un beneficio al cliente de la investigaci\'{o}n y al mismo tiempo, generar conocimiento de investigaci\'{o}n relevante. Por lo tanto, esta metodolog\'{i}a es una forma de investigar de car\'{a}cter colaborativo que busca unir teor\'{i}a y pr\'{a}ctica entre investigadores y practicantes mediante un proceso naturaleza c\'{i}clica.

	La representaci\'{o}n m\'{a}s habitual de la Investigaci\'{o}n-Acci\'{o}n es la descrita por Baskerville (1999), la cual se muestra a continuacien forma de cinco fases que conforman un ciclo (Ver Figura 2), que se describen a continuaci\'{o}n.
	\FloatBarrier %you shall not pass table!!
	\begin{figure}
		\centering
		\includegraphics[scale=0.77]{img/investigacion-accion.png}
			\caption{\textbf{Figura 1.} \textit{Car\'{a}cter c\'{i}clico de Investigaci\'{o}n-Acci\'{o}n} (Fuente: Baskerville, 1999).}
	\end{figure}
	\FloatBarrier %you shall not pass table!!
	\begin{itemize}
		\item \textbf{Fase de diagn\'{o}stico:} Se realiza el proceso de identificaci\'{o}n de los problemas primarios de la investigaci\'{o}n.
		\item \textbf{Fase de planificaci\'{o}n:} Se especifican las acciones que se llevaran a cabo para solucionar los problemas primarios.
		\item \textbf{Fase de acci\'{o}n:} Se ejecutan las acciones planificadas en la fase anterior.
		\item \textbf{Fase de evaluaci\'{o}n u observaci\'{o}n:} Se efect\'{u}a una evaluaci\'{o}n de los resultados obtenidos, para observar, conocer y documentar los efectos de las acciones que fueron realizadas.
		\item \textbf{Fase de reflexi\'{o}n:} Se toman los conocimientos adquiridos en la investigaci\'{o}n-acci\'{o}n. Si las acciones ejecutadas no fueron exitosas, los conocimientos pueden proporcionar la base para el diagn\'{o}stico de un nuevo ciclo de investigaci\'{o}n-acci\'{o}n.
	\end{itemize}

En la Tabla 2 se muestran las actividades del presente proyecto, haciendo correspondencia a cada una de las fases de la Investigaci\'{o}n-Acci\'{o}n.
	\FloatBarrier %you shall not pass table!!
	\begin{table}[htb]
		\small
		\centering
		\setlength{\extrarowheight}{5pt}
		\begin{tabulary}{15cm}{|c|L|}
			\hline
			\textbf{Fase} & \textbf{Actividades}\\ \hline
			\textbf{Diagn\'{o}stico} & Identificar los problemas y limitaciones que presenta el software comercial del ``MiniScan XE Plus''.\\ \hline
			\textbf{Planificaci\'{o}n} & Seleccionar la metodolog\'{i}a de desarrollo, determinar los requisitos del software y realizar un plan de trabajo.
\\ \hline
			\textbf{Acci\'{o}n} & Desarrollar el software, tomando en cuenta los requisitos identificados previamente, los lineamientos de ingenier\'{i}a del software, est\'{a}ndares de dise\~{n}o y calidad de software.\\ \hline
			\textbf{Evaluaci\'{o}n} & Realizar las pruebas de funcionalidad del software en cuesti\'{o}n y de su interfaz gr\'{a}fica de usuario.\\ \hline
			\textbf{Reflexi\'{o}n} & Presentar los resultados y los an\'{a}lisis de las pruebas realizadas.\\ \hline
	\end{tabulary}
		\caption{\textbf{Tabla 2.} \textit{Actividades del proyecto seg\'{u}n metodolog\'{i}a Investigaci\'{o}n-Acci\'{o}n }		(Fuente: Elaboración propia).}
	\end{table}
	\FloatBarrier %you shall not pass table!!
	
	\section{Metodolog\'{i}a de Desarrollo de Software}
Para el desarrollo del software que cumpla con los objetivos planteados en esta investigaci\'{o}n y tomando en cuenta los lineamientos planteados por la ingenier\'{i}a del software, con el objetivo de obtener un software que sea fiable y que funcione eficientemente (Pressman, 2002), se ha realizado una revisi\'{o}n del enfoque que deber\'{i}a tener la metodolog\'{i}a de desarrollo a utilizar.

Seg\'{u}n Sommerville (2005, p. 361), en los a\~{n}os 80 y principios de los 90, exist\'{i}a una opini\'{o}n general de que la mejor forma de obtener un mejor software era a trav\'{e}s de una planificaci\'{o}n cuidadosa del proyecto, una garant\'{i}a de calidad formalizada, la utilizaci\'{o}n de m\'{e}todos de an\'{a}lisis y dise\~{n}o soportados por herramientas CASE, y procesos de desarrollo de software controlados y rigurosos. El software que segu\'{i}a lo mencionado previamente era desarrollado por grandes equipos que a veces trabajaban para compa\~{n}\'{i}as diferentes. A menudo estaban dispersos geogr\'{a}ficamente y trabajaban en el software durante largos periodos de tiempo.

Ahora bien, debido a que no se dispone de un equipo grande para el desarrollo del software objetivo de la presente investigaci\'{o}n, y a que no se trabajar\'{a} en \'{e}ste durante un largo periodo de tiempo, se utilizar\'{a} una metodolog\'{i}a de desarrollo de enfoque \'{a}gil. Acorde con Sommerville (2005, p. 362), los m\'{e}todos \'{a}giles dependen de un enfoque iterativo para la especificaci\'{o}n, desarrollo y entrega del software, y est\'{a}n pensados para entregar software funcional de forma r\'{a}pida a los clientes, quienes pueden entonces proponer que se incluyan en iteraciones posteriores del software nuevos requerimientos o cambios en los mismos. Si bien los m\'{e}todos \'{a}giles proponen procesos diferentes para el desarrollo y entrega incrementales de software, comparten unos principios en com\'{u}n, los cuales son ilustrados en la Tabla 3.

	\begin{table}[htb]
		\small
		\centering
		\setlength{\extrarowheight}{5pt}
		\begin{tabulary}{15cm}{|c|L|}
			\hline
			\textbf{Principio} & \textbf{Descripci\'{o}n}\\ \hline
			\textbf{Participaci\'{o}n del cliente} & Los clientes deben estar fuertemente implicados en todo el proceso de desarrollo. Su papel es proporcionar y priorizar nuevos requerimientos del software y evaluar las iteraciones del sistema.\\ \hline
			\textbf{Entrega incremental} & El software se desarrolla en incrementos, donde el cliente especifica los requerimientos a incluir en cada incremento.\\ \hline
			\textbf{Personas, no procesos} & Se deben reconocer y explotar las habilidades del equipo de desarrollo. Se les debe dejar desarrollar sus propias formas de trabajar, sin procesos formales.\\ \hline
			\textbf{Aceptar el cambio} & Se debe contar con que los requerimientos del software cambian, por lo que el software se dise\~{n}a para dar cabida a estos cambios.\\ \hline
			\textbf{Mantener la simplicidad} & Se debe centrar la simplicidad tanto en el software a desarrollar como en el proceso de desarrollo. Donde sea posible, se trabaja activamente para eliminar la complejidad del software.\\ \hline
		\end{tabulary}
		\caption{\textbf{Tabla 3.} \textit{Principios de los m\'{e}todos \'{a}giles} (Fuente: Sommerville, 2005).}
	\end{table}

		\subsection{Metodolog\'{i}a SCRUM}
De acuerdo con Schwaber y Sutherland (2013, p. 4), esta metodolog\'{i}a es un marco de trabajo por el cual las personas pueden acometer problemas complejos adaptativos, y a la vez entregar productos del m\'{a}ximo valor posible, productiva y creativamente. SCRUM no es un proceso o una t\'{e}cnica para construir productos; en lugar de eso, es un marco de trabajo dentro del cual se pueden emplear varias t\'{e}cnicas y procesos. 
			
		\newpage			
			
El marco de trabajo SCRUM consiste en los equipos SCRUM, roles, eventos, artefactos y reglas asociadas. Cada componente dentro del marco de trabajo sirve a un prop\'{o}sito espec\'{i}fico y es esencial para el \'{e}xito de SCRUM y para su uso (Schwaber y Sutherland, 2013, p. 4).

SCRUM se basa en la teor\'{i}a de control de procesos emp\'{i}rica. El empirismo asegura 	que el conocimiento procede de la experiencia y de tomar decisiones bas\'{a}ndose en lo que se conoce. Esta metodolog\'{i}a emplea un enfoque iterativo e incremental para optimizar la predictibilidad y el control de riesgo. La implementaci\'{o}n de este control de procesos est\'{a} soportada por tres pilares, los cuales se muestran en la Tabla 4.
			
			\begin{table}[htb]
				\small
				\centering
				\setlength{\extrarowheight}{5pt}
				\begin{tabulary}{15cm}{|c|L|}
					\hline
					\textbf{Pilar} & \textbf{Descripci\'{o}n}\\ \hline
					\textbf{Transparencia} & Los aspectos significativos del proceso deben ser visibles para aquellos que son responsables del resultado.\\ \hline
					\textbf{Inspecci\'{o}n} & Los usuarios SCRUM deben inspeccionar frecuentemente los artefactos de SCRUM y el proceso hacia un objetivo, para detectar variaciones.\\ \hline
					\textbf{Adaptaci\'{o}n} & Si un inspector determina que uno o m\'{a}s aspectos de un proceso se desv\'{i}an de l\'{i}mites aceptables, y que el producto resultante no ser\'{a} aceptable, el proceso o el material que est\'{a} siendo procesado debe ser ajustado.\\ \hline
			\end{tabulary}
			\caption{\textbf{Tabla 4.} \textit{Pilares del control de procesos de SCRUM} (Fuente: Elaboraci\'{o}n propia).}
			\end{table}
			\newpage
			
Adicionalmente a la utilizaci\'{o}n de la metodolog\'{i}a SCRUM, se incluyeron algunos
artefactos de la metodolog\'{i}a RUP (Rational Unified Process), para as\'{i} generar suficiente documentaci\'{o}n durante el dise\~{n}o y el desarrollo del nuevo software. La configuraci\'{o}n
 de la metodolog\'{i}a SCRUM utlizada en conjunto con los artefactos elegidos de la
 metodolog\'{i}a RUP, es la ilustrada en la Tabla 5.

			\begin{table}[htb]
				\small
				\centering
				\setlength{\extrarowheight}{5pt}
			\begin{tabulary}{15cm}{|L|L|}
				\hline
				\textbf{Artefactos SCRUM}\\ \hline
				\textbf{Backlog de producto: }Lista din\'{a}mica de las cosas que se deben hacer, sin especificar c\'{o}mo se deben hacer.\\ \hline
				\textbf{Backlog de sprint: }Recopilaci\'{o}n sint\'{e}tica de los \'{i}tems del backlog del producto, en donde se quiebran los \'{i}tems en tareas peque\~{n}as que no demanden una labor superior a una jornada de trabajo.\\ \hline
				\textbf{Incremento de funcionalidad: }Es lo que el equipo SCRUM entrega la final de cada sprint. El mismo debe asemejarse a un software funcionando, permitiendo implementarse operativamente sin restricciones en un ambiente productivo.\\ \hline
				\textbf{Artefactos RUP}\\ \hline
				\textbf{Documento de Visi\'{o}n: }Documento que define el alcance de alto nivel y prop\'{o}sito del producto.\\
\hline
				\textbf{Glosario: }Documento que define la terminolog\'{i}a empleada en los artefactos.\\ \hline
				\textbf{Configuraci\'{o}n de documentos de requerimientos:} Solamente requerimientos no funcionales.\\ \hline
		\textbf{Configuraci\'{o}n de documentos de arquitectura: } Solamente diagrama de Casos de Uso.\\ \hline
			\end{tabulary}
	\caption{\textbf{Tabla 5.} \textit{Configuraci\'{o}n de los artefactos a utilizar de SCRUM y RUP} (Fuente: Elaboraci\'{o}n propia).}
			\end{table}
	\newpage
	\chapter{Resultados}

\capIV

\section{Fases metodol\'{o}gicas}

\subsection{Visi\'{o}n}
	
	\subsubsection{Enunciado del problema}
	
	El problema que se presenta es que se est\'{a} utilizando el HunterLab Universal Software para el manejo del espectrofot\'{o}metro de reflexi\'{o}n difusa MiniScan XE Plus. Dicho software est\'{a} en ingl\'{e}s, es comercial, privativo y fue descontinuado; esto afecta a los dermat\'{o}logos del Centro de Investigaciones M\'{e}dicas y Biotecnol\'{o}gicas de la Universidad de Carabobo (CIMBUC).
	
	El impacto causado por esto es que los dermat\'{o}logos encuentran el HunterLab Universal Software dif\'{i}cil de utilizar, e imposible de adaptar a sus necesidades, lo que ralentiza la actividad de consulta con sus pacientes, genera la necesidad de disponer de personal especializado para su debido uso, y disminuye el potencial de dicho instrumento.
	
	Una soluci\'{o}n satisfactoria ser\'{i}a disponer de un software para el uso del \mbox{MiniScan} XE Plus que est\'{e} en espa\~{n}ol, que sea amigable y mantenible, permitiendo que se adapte a las necesidades de los dermat\'{o}logos.
	
	\subsubsection{Descripci\'{o}n de los usuarios}
	
		\begin{table}[h]
		\small
		\caption[Actores del negocio]{\textit{Actores del negocio} (Fuente: Autor).}
		\centering
		\setlength{\extrarowheight}{\altocelda}
		\begin{tabulary}{\anchotabla}{|c|J|}
			\hline
			\thead{\textbf{\small{Actor}}} & \thead{\textbf{\small{Descripci\'{o}n}}}\\ \hline
			\textbf{Administrador} &
			
			Realiza mediciones.
			
			Consulta las historias m\'{e}dicas de los pacientes y las muestras.
			
			Gestiona los usuarios.
		\\ \hline
			\textbf{Dermat\'{o}logo} &
			
			Realiza mediciones.
			
			Gestiona las historias m\'{e}dicas de los pacientes y las muestras.
			
			Realiza diagn\'{o}sticos a partir de los resultados de las muestras.
		\\ \hline
			\textbf{Investigador} &
			
			Realiza mediciones.
			
			Consulta las historias m\'{e}dicas de los pacientes y las muestras.
			
			Realiza an\'{a}lisis sobre los resultados de las mediciones.\\ \hline
		\end{tabulary}
	\end{table}
	
	\begin{table}[h]
		\small
		\caption[Actores del software]{\textit{Actores del software} (Fuente: Autor).}
		\centering
		\setlength{\extrarowheight}{\altocelda}
		\begin{tabulary}{\anchotabla}{|c|J|c|c|}
			\hline
			\thead{\textbf{\small{Actor}}} & \thead{\textbf{\small{Responsabilidad}}} & \thead{\textbf{\small{Experiencia}}} & \thead{\textbf{\small{Uso}}}\\ \hline
			
			\textbf{Administrador} &
			
			Manejar el MiniScan XE Plus.
			
			Crear, consultar, modificar y eliminar usuarios.
			
			Consultar historias m\'{e}dicas de pacientes.
			
			Consultar muestras de pacientes. &
			Alta &
			Alto\\ \hline
			
			\textbf{Dermat\'{o}logo} &
			
			Manejar el MiniScan XE Plus.
			
			Crear, consultar, modificar y eliminar historias m\'{e}dicas de pacientes.
			
			Crear, consultar, modificar y eliminar muestras de pacientes. &
			Baja &
			Alto\\ \hline
			
			\textbf{Investigador} &
			
			Manejar el MiniScan XE Plus.
			
			Consultar historias m\'{e}dicas de pacientes.
			
			Consultar muestras de pacientes. &
			Media &
			Alto\\ \hline
		\end{tabulary}
	\end{table}
	
	\subsubsection{Resumen del producto}
	
	El software desarrollado, denominado a partir de ahora Spectrasoft, es una aplicaci\'{o}n para el uso del MiniScan XE Plus, la recuperaci\'{o}n de los datos de medici\'{o}n de dicho instrumento, la generaci\'{o}n de resultados relevantes y la gesti\'{o}n de los mismos, el cual est\'{a} orientado a las actividades m\'{e}dicas dermatol\'{o}gicas del Centro de Investigaciones M\'{e}dicas y Biotecnol\'{o}gicas de la Universidad de Carabobo (CIMBUC). La tabla 4.3 resume los beneficios y las caracter\'{i}sticas m\'{a}s importantes que provee el producto.
	
	\begin{table}[h]
		\small
		\caption[Beneficios y caracter\'{i}sticas principales del producto]{\textit{Beneficios y caracter\'{i}sticas principales del producto} (Fuente: Autor).}
		\centering
		\setlength{\extrarowheight}{\altocelda}
		\begin{tabulary}{\anchotabla}{|J|J|}
			\hline
			\thead{\textbf{\small{Beneficio al cliente}}} & \thead{\textbf{\small{Caracter\'{i}stica que lo soporta}}}\\ \hline
			Se puede conectar, calibrar y realizar mediciones con el MiniScan XE Plus. & 
			Comunicaci\'{o}n con el MiniScan XE Plus y acceso a las caracter\'{i}sticas comunmente utilizadas por el mismo.\\ \hline
			Se dispone de informaci\'{o}n relevante para el an\'{a}lisis y diagn\'{o}stico de patolog\'{i}as dermatol\'{o}gicas en la piel de los pacientes. &
			Muestra de los datos espectrales obtenidos de las mediciones, representaci\'{o}n gr\'{a}fica de dichos datos, y c\'{a}lculo de valores adicionales asociados a los mismos.\\ \hline
			Se pueden gestionar los usuarios, las historias m\'{e}dicas y las muestras generadas de las mediciones. &
			Manejo de una base de datos que almacena toda la informaci\'{o}n referente a los usuarios, las historias m\'{e}dicas y las muestras, permitiendo su gesti\'{o}n por medio del Spectrasoft.\\ \hline
			El Spectrasoft se puede utilizar con facilidad. &
			Interfaz gr\'{a}fica de usuario en espa\~{n}ol, que ofrece \'{u}nicamente las funciones necesarias para gestionar la informaci\'{o}n que necesitan los usuarios. \\ \hline
			El Spectrasoft se puede adaptar a las futuras necesidades de sus usuarios. &
			C\'{o}digo abierto del proyecto disponible en su totalidad para realizar cualquier modificaci\'{o}n y/o extensi\'{o}n.\\ \hline
		\end{tabulary}
	\end{table}
	
	\subsubsection{Principales restricciones}
	
	El software se desarrolla utilizando el lenguaje de programaci\'{o}n C++, empleando \'{u}nicamente tecnolog\'{i}as gratuitas, y en la medida de lo posible, de c\'{o}digo abierto. Este se ejecuta en sistemas operativos Windows actuales. Por \'{u}ltimo, la comunicaci\'{o}n entre el software y el MiniScan XE Plus se logra tanto por medio de un puerto serial, como por medio de un adaptador USB.
	
\subsection{Pila del producto \textit{(product backlog)}}
	La lista de requerimientos funcionales que necesitan ser realizados a lo largo del desarrollo del Spectrasoft, son el producto de una reuni\'{o}n que se tuvo con los dermat\'{o}logos y los investigadores del CIMBUC, al igual que de la observaci\'{o}n directa efectuada a las funciones que el HunterLab Universal Software ofrece para el manejo del MiniScan XE Plus. En la tabla 4.4 se muestran dichos requerimientos.
	
	\begin{table}[h]
		\small
		\caption[Requerimientos funcionales del software]{\textit{Requerimientos funcionales del software} (Fuente: Autor).}
		\centering
		\setlength{\extrarowheight}{\altocelda}
		\begin{tabulary}{\anchotabla}{|c|J|c|}
			\hline
			\thead{\textbf{\small{C\'{o}digo}}} & \thead{\textbf{\small{Requerimiento}}} & \thead{\textbf{\small{Prioridad}}}\\ \hline
			
			\textbf{RF01} & Conectar y desconectar el MiniScan XE Plus. & Esencial\\ \hline
			
			\textbf{RF02} & Calibrar el MiniScan XE Plus. & Esencial\\ \hline
			
			\textbf{RF03} & Recuperar los 31 puntos espectrales de una medici\'{o}n con el MiniScan XE Plus y mostrarlos en su forma num\'{e}rica. & Esencial\\ \hline

			\textbf{RF04} & Graficar una curva de reflectancia difusa a partir de los 31 puntos espectrales recuperados. & Esencial\\ \hline
			\textbf{RF05} & Graficar una curva de absorbancia aparente a partir de los 31 puntos espectrales recuperados. & Esencial\\ \hline
			\textbf{RF06} & Calcular y mostrar las coordenadas de cromaticidad CIE xyz. & Esencial\\ \hline
			
			\textbf{RF07} & Calcular y mostrar las coordenadas del espacio CIELAB. & Esencial\\ \hline

			\textbf{RF08} & Calcular y graficar el coeficiente de absorci\'{o}n de la epidermis. & Esencial\\ \hline

			\textbf{RF09} & Calcular y graficar el coeficiente de esparcimiento de la epidermis. & Esencial\\ \hline

			\textbf{RF010} & Calcular y mostrar el \'{i}ndice de eritema. & Esencial\\ \hline

			\textbf{RF11} & Almacenar la informaci\'{o}n de los usuarios, las historias m\'{e}dicas, las muestras y los resultados de las mediciones en una base de datos. & Esencial\\ \hline			

			\textbf{RF12} & Gestionar la creaci\'{o}n, consulta, modificaci\'{o}n y eliminaci\'{o}n de los usuarios. & Esencial\\ \hline
			
			\textbf{RF13} & Gestionar la creaci\'{o}n, consulta, modificaci\'{o}n y eliminaci\'{o}n de las historias m\'{e}dicas de pacientes. & Esencial\\ \hline
			
			\textbf{RF14} & Gestionar la creaci\'{o}n, consulta, modificaci\'{o}n y eliminaci\'{o}n de las muestras pertenecientes a las historias m\'{e}dicas. & Esencial\\ \hline	
			
			\textbf{RF15} & Exportar los datos de una muestra a un archivo port\'{a}til. & Esencial\\ \hline
		\end{tabulary}
	\end{table}
	
\subsection{Requerimientos no funcionales}
	
	De la misma forma que los requerimientos funcionales, la lista de los requerimientos no funcionales fue definida en la misma reuni\'{o}n con los dermat\'{o}logos y los investigadores, tomando en cuenta las restricciones del entorno en donde se va a ejecutar el Spectrasoft. En la tabla 4.5 se describen estos requerimientos no funcionales.
	
	\begin{table}[h]
		\small
		\caption[Requerimientos no funcionales del software]{\textit{Requerimientos no funcionales del software} (Fuente: Autor).}
		\centering
		\setlength{\extrarowheight}{\altocelda}
		\begin{tabulary}{\anchotabla}{|c|J|c|}
			\hline
			\thead{\textbf{\small{C\'{o}digo}}} & \thead{\textbf{\small{Requerimiento}}} & \thead{\textbf{\small{Prioridad}}}\\ \hline
			\textbf{RNF01} & El software debe ser f\'{a}cil de utilizar, por lo que debe cumplir con el atributo de usabilidad de un software de calidad. & Esencial\\ \hline
			\textbf{RNF02} & El software debe ser capaz de adaptarse a las necesidades de los dermat\'{o}logos, raz\'{o}n por la cual debe cumplir con el atributo de mantenibilidad de un software de calidad. & Esencial\\ \hline
			\textbf{RNF03} & El software debe desarrollarse empleando \'{u}nicamente herramientas y tecnolog\'{i}as gratuitas, y en la medida de lo posible, de c\'{o}digo abierto. & Esencial\\ \hline
			\textbf{RNF04} & El software debe ser capaz de ejecutarse en sistemas Windows actuales, con arquitecturas de 32 bits y 64 bits. & Esencial\\ \hline
			\textbf{RNF05} & El software debe conectarse con el MiniScan XE Plus por medio de un puerto serial o de un adaptador USB. & Esencial\\ \hline
			\textbf{RNF06} & El archivo port\'{a}til al que se exportan los resultados de una medici\'{o}n debe ser abierto por un visualizador/editor de hojas de c\'{a}lculo. & Esencial\\ \hline
			\textbf{RNF07} & El software debe desarrollarse utilizando el lenguaje de programaci\'{o}n orientada a objetos C++. & Esencial\\ \hline
		\end{tabulary}
	\end{table}
	
\newpage

\subsection{Casos de uso}

 \begin{itemize}
 	
 	\item \textbf{Manejar el MiniScan XE Plus:}
 	
 		\begin{figure}[H]
		\centering
		\includegraphics[scale=0.4]{img/cu-manejar-miniscan.png}
			\caption[Caso de uso: manejar el MiniScan XE Plus]{\textit{ Caso de uso: manejar el MiniScan XE Plus} (Fuente: Autor).}
	\end{figure}
	
	 	\item \textbf{Gestionar medici\'{o}n:}
 	
 		\begin{figure}[H]
		\centering
		\includegraphics[scale=0.4]{img/cu-gestion-medicion.png}
			\caption[Caso de uso: gestionar medici\'{o}n]{\textit{ Caso de uso: gestionar medici\'{o}n} (Fuente: Autor).}
	\end{figure}

\newpage
	 	\item \textbf{Gestionar sesi\'{o}n:}
 	
 		\begin{figure}[H]
		\centering
		\includegraphics[scale=0.5]{img/cu-gestion-sesion.png}
			\caption[Caso de uso: gestionar sesi\'{o}n]{\textit{ Caso de uso: gestionar sesi\'{o}n} (Fuente: Autor).}
	\end{figure}
	
		 	\item \textbf{Gestionar historia:}
 	
 		\begin{figure}[H]
		\centering
		\includegraphics[scale=0.5]{img/cu-gestion-historia.png}
			\caption[Caso de uso: gestionar historia]{\textit{ Caso de uso: gestionar historia} (Fuente: Autor).}
	\end{figure}

\newpage
		\item \textbf{Gestionar muestra:}
 	
 		\begin{figure}[H]
		\centering
		\includegraphics[scale=0.5]{img/cu-gestion-muestra.png}
			\caption[Caso de uso: gestionar muestra]{\textit{ Caso de uso: gestionar muestra} (Fuente: Autor).}
	\end{figure}
	
			\item \textbf{Gestionar usuario:}
 	
 		\begin{figure}[H]
		\centering
		\includegraphics[scale=0.5]{img/cu-gestion-usuario.png}
			\caption[Caso de uso: gestionar usuario]{\textit{ Caso de uso: gestionar usuario} (Fuente: Autor).}
	\end{figure}
	
 \end{itemize}
 
	\subsubsection{Descripci\'{o}n de los casos de uso}
	
		\begin{itemize}
			\item \textbf{Manejar instrumento:} Permite a todos los usuarios conectar, desconectar y calibrar el MiniScan XE Plus, esto \'{u}ltimo haciendo uso de una trampa de luz y una cer\'{a}mica blanca.
			
			\item \textbf{Gestionar medici\'{o}n:} Permite a todos los usuarios efectuar una medici\'{o}n con el MiniScan XE Plus conectado, visualizar los resultados obtenidos de la misma, y borrarlos.
			
			\item \textbf{Consultar medici\'{o}n:} Permite a todos los usuarios visualizar tanto la informaci\'{o}n obtenida directamente del MiniScan XE Plus, como la informaci\'{o}n calculada a partir de la misma, como la curva de reflectancia difusa, la curva de absorbancia aparente, las coordenadas de cromaticidad CIE xyz, las coordenadas CIELAB y el \'{i}ndice de eritema de una medici\'{o}n realizada.
			
			\item \textbf{Gestionar sesi\'{o}n:} Permite a cualquier usuario iniciar sesi\'{o}n para acceder a las funciones pertinentes para su rol, consultar y modificar su informaci\'{o}n, cambiar su contrase\~{n}a, y cerrar sesi\'{o}n.
			
			\item \textbf{Gestionar historia:} Permite a todos los usuarios consultar la informaci\'{o}n perteneciente a las historias m\'{e}dicas, y s\'{o}lo permite a los usuarios con el rol de dermat\'{o}logo registrar, modificar y eliminar dichas historias. 
			
			\item \textbf{Gestionar muestra:} Permite a todos los usuarios consultar la informaci\'{o}n referente a las muestras pertenecientes a una historia m\'{e}dica de un paciente determinado, as\'{i} como exportar dicha informaci\'{o}n, y s\'{o}lo permite a los usuarios con el rol de dermat\'{o}logo registrar, modificar y eliminar tales muestras.
			
			\item \textbf{Gestionar usuario:} Permite registrar usuarios, consultar su informaci\'{o}n, cambiar los roles de dichos usuarios, cambiar sus contrase\~{n}as y eliminarlos. Esto solamente puede ser efectuado por usuarios con el rol de administrador.
		\end{itemize}

\subsection{Glosario}

\begin{itemize}
	\item \textbf{Espectrofot\'{o}metro de reflexi\'{o}n difusa: }
	
	\item \textbf{Software: }	
	
	\item \textbf{Software privativo: }
	
	\item \textbf{Medici\'{o}n: }
	
	\item \textbf{Calibraci\'{o}n: }
	
	\item \textbf{Usuario: }
	
	\item \textbf{Rol de usuario: }
	
	\item \textbf{Sesi\'{o}n: }
	
	\item \textbf{Historia m\'{e}dica: }
	
	\item \textbf{Muestra: }
	
	\item \textbf{Patolog\'{i}a dermatol\'{o}gica: }
	
	\item \textbf{Datos espectrales: }
	
	\item \textbf{Base de datos: }
	
	\item \textbf{Interfaz gr\'{a}fica de usuario: }
	
	\item \textbf{C\'{o}digo abierto: }
	
	\item \textbf{Lenguaje de programaci\'{o}n: }
	
	\item \textbf{Sistema operativo: }
	
	\item \textbf{Puerto serial: }
	
	\item \textbf{Adaptador USB: }
	
	\item \textbf{Reflectancia difusa: }
	
	\item \textbf{Absorbancia aparente: }
	
	\item \textbf{Coordenadas de cromaticidad CIE xyz: }
	
	\item \textbf{Coordenadas del espacio CIELAB: }
	
	\item \textbf{Epidermis: }	
	
	\item \textbf{Coeficiente de absorci\'{o}n: }
	
	\item \textbf{Coeficiente de esparcimiento: }
	
	\item \textbf{\'{I}ndice de eritema: }
	
	\item \textbf{Hoja de c\'{a}lculo: }
	
\end{itemize}

\newpage

\section{Base de datos}

	\subsection{Diagrama ER de la base de datos}

	\begin{figure}[H]
		\centering
		\includegraphics[scale=0.4]{img/diagramaER.png}
			\caption[Diagrama ER de la base de datos]{\textit{Diagrama ER de la base de datos} (Fuente: Autor).}
	\end{figure}
	
	\subsection{Descripci\'{o}n de las tablas de la base de datos}
	
		\begin{itemize}
				
				\item \textbf{historia:} almacena los datos referentes a la historia m\'{e}dica de cada uno de los pacientes registrados.
				
				\item \textbf{usuario:} guarda la informaci\'{o}n de cada uno de los usuarios que pueden acceder al software, que pueden ser administradores, dermat\'{o}logos o investigadores.
				
				\item \textbf{muestra:} contiene los datos relevantes de las muestras que son tomadas a los pacientes. Dichas muestras siempre est\'{a}n interrelacionadas con la historia m\'{e}dica del paciente al que pertenece, y la c\'{e}dula del usuario que la tom\'{o}.

				\item \textbf{datos\_espectrales:} contiene los 31 puntos espectrales resultantes de la medici\'{o}n realizada de cada muestra por medio del MiniScan XE Plus.
				
				\item \textbf{datos\_absorcion:} contiene los 31 puntos espectrales resultantes del c\'{a}lculo del coeficiente de absorci\'{o}n asociado con los datos espectrales.
				
				\item \textbf{datos\_esparcimiento:} contiene los 31 puntos espectrales resultantes del c\'{a}lculo del coeficiente de esparcimiento asociado con los datos espectrales.
				
				\item \textbf{datos\_adicionales:} almacena los datos que son calculados a partir de los 31 puntos espectrales de cada muestra, estos son las coordenadas de cromaticiad CIE xyz, las coordenadas del espacio del color CIELAB y el \'{i}ndice de eritema.
				
		\end{itemize}

\section{Tecnolog\'{i}as y recursos utilizados}

	\subsection{Tecnolog\'{i}as}
	
		\begin{itemize}
			
			\item \textbf{Qt:} es un \textit{framework} de desarrollo de aplicaciones multiplataforma para sistemas operativos de escritorio, sistemas integrados y sistemas m\'{o}viles. Se utiliz\'{o} la versi\'{o}n \textit{open source} 5.4.1 de este \textit{framework} para el desarrollo del Spectrasoft.
			
			\item \textbf{Visual Studio:} es un entorno integrado de desarrollo o \textit{IDE} para crear aplicaciones en varias plataformas, como Windows, Android y iOS. La versi\'{o}n 2013 de este \textit{IDE} fue utilizada para desarrollar una librer\'{i}a escrita en Visual Basic.NET, la cual act\'{u}a como intermediaria entre el kit \textit{MSXE.ocx} y el \textit{framework} Qt, para as\'{i} utilizar las caracter\'{i}sticas del MiniScan XE Plus en el Spectrasoft.
			
			\item \textbf{PostgreSQL:} es un sistema \textit{open source} multiplataforma de bases de datos relacionales. Posee m\'{a}s de 15 a\~{n}os de desarrollo activo y una arquitectura comprobada que se ha ganado una fuerte reputaci\'{o}n por confiabilidad, integridad de datos y correctitud. Este sistema se utiliz\'{o} para desarrollar y administrar la base de datos con la que opera el Spectrasoft.
			
			\item \textbf{Gitlab:} es un servicio de control de versiones que ofrece alojamiento gratuito, tanto p\'{u}blico como privado, de respositorios para proyectos. Se utiliz\'{o} la versi\'{o}n en l\'{i}nea de este servicio para llevar un control de versiones durante el desarrollo del proyecto.
			
			\item \textbf{QCustomPlot:} es un \textit{widget open source} para Qt que permite realizar el trazado y la visualizaci\'{o}n de datos. Este \textit{widget} fue empleado por el Spectrasoft para visualizar la curva de reflectancia difusa y la curva de absorbancia aparente asociadas a los 31 puntos espectrales resultantes de las mediciones.
			
			\item \textbf{QtXlsx:} es una librer\'{i}a \textit{open source} para Qt que permite leer y escribir archivos con extensi\'{o}n xlsx. Esta librer\'{i}a fue utilizada para implementar en el Spectrasoft la opci\'{o}n de exportar los resultados de una muestra a un archivo port\'{a}til, manejable por medio de aplicaciones de hojas de c\'{a}lculo.
		\end{itemize}
	
	\subsection{Recursos}
	
		\begin{itemize}
		
			\item \textbf{MiniScan XE Plus:} es un instrumento de medici\'{o}n del color creado por la empresa HunterLab, de dise\~{n}o compacto y port\'{a}til, que emplea la t\'{e}cnica de espectroscop\'{i}a de reflectancia difusa, el cual se puede apreciar en la figura 4.8. Este instrumento mide la cantidad de luz que refleja una muestra dentro de un rango de longitudes de onda que va desde los 400 hasta los 700 nan\'{o}metros, generando como resultado 31 puntos espectrales dentro de ese rango, que son el insumo principal del Spectrasoft.
			
			\item \textbf{Adaptador RS232-USB:} es un cable adaptador que habilita la comunicaci\'{o}n de dispositivos que utilizan puerto serial con computadoras que disponen de puertos USB, creando puertos COM virtuales con las mismas mientras se realiza dicha comunicaci\'{o}n. Este cable es utilizado como adaptador para el cable de comunicaci\'{o}n RS232 DB-9 hembra a RJ-45 del MiniScan XE Plus, y as\'{i} habilitar la utilizaci\'{o}n de este instrumento en computadoras que no poseen puerto serial.
			
		\end{itemize}
		
	\begin{figure}[H]
		\centering
		\includegraphics[scale=1]{img/MiniScanXEPlus.png}
			\caption[MiniScan XE Plus]{\textit{MiniScan XE Plus} (Fuente: HunterLab, 2006).}
	\end{figure}
	
\section{Comunicaci\'{o}n con el MiniScan XE Plus}

	Para establecer la comunicaci\'{o}n entre el MiniScan XE Plus y el Spectrasoft, se recurri\'{o} a la documentaci\'{o}n del instrumento, en la cual se describe el MSXE.ocx, que es un archivo que implementa las funciones comunmente utilizadas por dicho instrumento. Se contact\'{o} al personal de soporte t\'{e}cnico de HunterLab por correo electr\'{o}nico, para solicitarle el c\'{o}digo fuente de dicho archivo y la documentaci\'{o}n relativa a su utilizaci\'{o}n que se pudiera proporcionar para la investigaci\'{o}n.
	
	Si bien el personal no comparti\'{o} el c\'{o}digo fuente del archivo, s\'{i} envi\'{o} la documentaci\'{o}n solicitada y un ejemplo de su uso escrito en Visual Basic for Applications (VBA). Primero se intent\'{o} cargar el archivo y utilizarlo directamente en Qt; sin embargo, ocurr\'{i}a un error de compatibilidad de datos al invocar algunas de sus funciones. La soluci\'{o}n a este problema fue desarrollar una librer\'{i}a escrita en Visual Basic .NET, con la cual se pueden invocar todas las funciones de este archivo sin problema alguno.

	As\'{i} pues, por medio del cable adaptador RS232-USB, y empleando la librer\'{i}a escrita en Visual Basic .NET, se logr\'{o} establecer la comunicaci\'{o}n entre el software que est\'{a} siendo desarrollado en Qt y el MiniScan XE Plus, por medio del archivo MSXE.ocx.

\newpage

\section{F\'{o}rmulas implementadas}
	A continuaci\'{o}n se muestra el c\'{o}digo fuente de las f\'{o}rmulas de \'{o}ptica y colorimetr\'{i}a que fueron definidas en las bases te\'{o}ricas, e implementadas en el Spectrasoft.
	
	\begin{itemize}
	
	\item \textbf{Funci\'{o}n de \'{i}ndice de eritema:}	
		\begin{lstlisting}
//Calcula el indice de eritema
float eritema(QVector<float> medicion){
    //promRojo: promedio ponderado del color rojo
    float promRojo = (medicion[24]/2.0 + medicion[25]
    + medicion[26] + medicion[27]/2.0)/3.0;

    //promVerde: promedio ponderado del color verde
    float promVerde = (medicion[16]/2.0 + medicion[17]
     + medicion[18]/2.0)/2.0;

    float resultado = 100.0*(log(1.0/promVerde)
     - log(1.0/promRojo));

    return resultado;
}
	\end{lstlisting}
	
\newpage	
	
		\item \textbf{Funci\'{o}n de absorbancia aparente:}
			\begin{lstlisting}
//Calcula los datos de absorbancia aparente
QVector<float> absorbancia(QVector<float> medicion){

	QVector<float> resultado;
	
   	for(int i = 0; i < 31; ++i)
		resultado.push_back(100.0 - medicion[i]);

    return resultado;
}
			\end{lstlisting}
			
		\item \textbf{Funci\'{o}n de coordenadas de cromaticidad CIE xyz:}		
			\begin{lstlisting}
//Calcula las coordenadas de cromaticidad CIE xyz
QVector<float> CIExyz(QVector<float> medicion){
    
    QVector<float> resultado;
    QVector<float> XYZ = CIEXYZ(medicion);
    float x, y, z;

    x = XYZ[0]/(XYZ[0] + XYZ[1] + XYZ[2]);
    y = XYZ[1]/(XYZ[0] + XYZ[1] + XYZ[2]);
    z = XYZ[2]/(XYZ[0] + XYZ[1] + XYZ[2]);

    resultado.push_back(x);
    resultado.push_back(y);
    resultado.push_back(z);

    return resultado;
}
			\end{lstlisting}
			
\newpage
		
		\item \textbf{Funci\'{o}n de valores triest\'{i}mulo CIE XYZ:}
			\begin{lstlisting}
//Calcula los valores triestimulo CIE XYZ
QVector<float> CIEXYZ(QVector<float> medicion){
    
    QVector<float> resultado;
    float auxK, auxX, auxY, auxZ, k, X, Y, Z;

    auxK = auxX = auxY = auxZ = 0.0;

    //realiza las sumatorias indicadas de las formulas
    for(int i = 0; i < 31; ++i){

		auxK+= iluCIED65[i]*yCIE10[i];
        auxX+= medicion[i]*iluCIED65[i]*xCIE10[i];
        auxY+= medicion[i]*iluCIED65[i]*yCIE10[i];
        auxY+= medicion[i]*iluCIED65[i]*zCIE10[i];
    }

    //calcula la constante k
    k = 100.0/auxK;

    //calcula los valores triestimulo XYZ
    X = k*auxX;
    Y = k*auxY;
    Z = k*auxZ;

    resultado.push_back(X);
    resultado.push_back(Y);
    resultado.push_back(Z);

    return resultado;
}
			\end{lstlisting}

\newpage
		
		\item \textbf{Funci\'{o}n de coordenadas CIELAB:}
			\begin{lstlisting}
//Calcula las coordenadas de del espacio CIELAB
QVector<float> CIELAB(QVector<float> medicion){
    
    QVector<float> resultado;
    QVector<float> XYZ = CIEXYZ(medicion);
    float constante, aux, fXfYfZ[3], L, a, b;

    //calcula la constante utilizada en la formula
    constante = 24.0/116.0;
    constante = pow(constante, 3);

    //calcula las funciones X/Xn, Y/Yn, Z/Zn
    for(int i = 0; i < 3; ++i){

        aux = XYZ[i]/XnYnZn[i];

        if(aux > constante){
            fXfYfZ[i] = pow(aux, 1.0/3.0);
        }else{
            fXfYfZ[i] = (841.0/108.0)*aux + (16.0/116.0);
        }
    }

    //calcula las coordenadas L*a*b*
    L = 116.0*fXfYfZ[1] - 16.0;
    a = 500.0*(fXfYfZ[0] - fXfYfZ[1]);
    b = 200.0*(fXfYfZ[1] - fXfYfZ[2]);

    resultado.push_back(L);
    resultado.push_back(a);
    resultado.push_back(b);

    return resultado;
}
			\end{lstlisting}

\newpage		
		
		\item \textbf{Funci\'{o}n de coeficiente de absorci\'{o}n:}
		
			\begin{lstlisting}
//Calcula el coeficiente de absorcion
QVector<float> absorcion(QVector<float> medicion){

    QVector<float> resultado;
    float rango, aux, Z, k, a0;
    int n, grado;
    bool op = false;//el sistema de ecuaciones a resolver tiene solucion
    
    //son 31 datos, y el grado del polinomio de ajuste es 8
    n = 31; grado = 8;

    double *x = new double[n];
    double *y = new double[n];

    //cargando los datos en x, y
    for(int i = 0; i < n; ++i){
        x[i] = i;
        y[i] = medicion.at(i);
    }

    double **matriz = new double*[grado + 1];

    for (int i = 0; i < grado + 1; ++i)
        matriz[i] = new double[grado + 2];

    //realizando el ajuste polinomial y calculando los coeficientes
    coeficientes(x, y, matriz, grado, n);
    recorrido(matriz, grado + 1, op);

    //a0 esta en la primera fila de la matriz resultante del ajuste
    a0 = matriz[0][grado + 1];
    Z = 0.2796; k = -7.174;
    rango = 400;

    //calculando el coeficiente de absorcion
    for(int i = 0; i < n; ++i){
        aux = Z*exp(k*a0)*6.6*pow(10, 11)*pow(rango, -3.3);
        resultado.push_back(aux);
        rango+=10;
    }

    //liberando la memoria dinamica asiganada
    delete x;
    delete y;

    for(int i = 0; i < grado + 1; ++i)
        delete matriz[i];

    delete matriz;

    return resultado;
}
			\end{lstlisting}

	\end{itemize}

\newpage

\section{Interfaz del Spectrasoft}

	A continuaci\'{o}n se muestran algunas de las vistas de la interfaz gr\'{a}fica de usuario del Spectrasoft, que comprenden la vista principal, la ventana de inicio de sesi\'{o}n de un usuario, la visualizaci\'{o}n de algunos de los resultados de una medici\'{o}n realizada, y los detalles de la historia m\'{e}dica de un paciente.

 \begin{itemize}
 	
 	\item \textbf{Vista principal:}
 	
 		\begin{figure}[H]
		\centering
		\includegraphics[scale=0.6]{img/vista-principal.jpg}
			\caption[Vista principal]{\textit{Vista principal} (Fuente: Autor).}
	\end{figure}
	
	 	\item \textbf{Inicio de sesi\'{o}n:}
 	
 		\begin{figure}[H]
		\centering
		\includegraphics[scale=0.6]{img/vista-inicio-sesion.jpg}
			\caption[Inicio de sesi\'{o}n]{\textit{Inicio de sesi\'{o}n} (Fuente: Autor).}
	\end{figure}

\newpage
	 	\item \textbf{Curva de reflectancia:}
 	
 		\begin{figure}[H]
		\centering
		\includegraphics[scale=0.4]{img/vista-reflectancia.jpg}
			\caption[Curva de reflectancia]{\textit{Curva de reflectancia} (Fuente: Autor).}
	\end{figure}
	
		 	\item \textbf{Historia m\'{e}dica de un paciente:}
 	
 		\begin{figure}[H]
		\centering
		\includegraphics[scale=0.6]{img/vista-historia.jpg}
			\caption[Historia m\'{e}dica de un paciente]{\textit{Historia m\'{e}dica de un paciente} (Fuente: Autor).}
	\end{figure}

\end{itemize}

\newpage

\section{Pruebas realizadas}

	\newpage
	\chapter{Conclusiones y recomendaciones}

\capV

\section{Conclusiones}

\section{Recomendaciones}
	\newpage
	\nocite{*}
	\renewcommand\bibname{Referencias}
	\bibliography{bibliografia}
\end{document}