\chapter{\label{cap:1}El Problema}

\begin{chapquote}{Richard Stallman}
<<Software privativo significa que priva a los usuarios de su libertad.>>
\end{chapquote}

	\section{Planteamiento del Problema}	
\cite{Bersha} indica que durante el diagn\'{o}stico de enfermedades de la piel, la observaci\'{o}n cuidadosa y la evaluaci\'{o}n visual del \'{a}rea sospechada es siempre el primer paso, y el m\'{a}s importante. Esto es seguido generalmente por una escisi\'{o}n o biopsia por punci\'{o}n, en la que se extrae una muestra de tejido de la piel para un an\'{a}lisis microsc\'{o}pico. La observaci\'{o}n visual suele ser subjetiva y los pacientes a menudo se someten a cicatrices y dolor durante la biopsia. Por otro lado, las t\'{e}cnicas \'{o}pticas son por lo general no invasivas y sus resultados son a menudo objetivos. Durante el diagn\'{o}stico no invasivo no se crea ninguna ruptura en la piel, y los pacientes no se someten al dolor ni a cicatrices durante el tratamiento.

Los avances tecnol\'{o}gicos en la actualidad permiten emplear t\'{e}cnicas de \'{o}ptica con la capacidad de estudiar  las propiedades estructurales y bioqu\'{i}micas del tejido biol\'{o}gico de manera precisa y no invasiva. Los instrumentos que emplean tales t\'{e}cnicas son de gran ayuda para los m\'{e}dicos dermat\'{o}logos, raz\'{o}n por la cual han tomado suma importancia en el \'{a}rea m\'{e}dica dermatol\'{o}gica.

Hoy d\'{i}a existen diferentes tipos de estudios \'{o}pticos in-situ, in-vivo e invitro del tejido biol\'{o}gico, como la Espectroscop\'{i}a de Reflectancia Difusa (ERD). \cite{Perez-Gallardo} asegura que con esta t\'{e}cnica es  posible estudiar las propiedades bioqu\'{i}micas y las condiciones estructurales de un tejido biol\'{o}gico, analizando la interacci\'{o}n luz-tejido de una manera no invasiva.

En este sentido, el Centro de Investigaciones M\'{e}dicas y Biotecnol\'{o}gicas de la Universidad de Carabobo (CIMBUC) dispone de un espectrofot\'{o}metro de reflexi\'{o}n difusa, denominado MiniScan XE Plus, creado por la empresa HunterLab. Esta empresa lo describe como un instrumento utilizado para medir la transmisi\'{o}n y/o reflectancia de espec\'{i}menes, como una funci\'{o}n de longitud de onda, que aplica la t\'{e}cnica de ERD. 

Ahora bien, el CIMBUC hace uso de este instrumento a trav\'{e}s del software disponible para su utilizaci\'{o}n, designado HunterLab Universal Software (HLUS), que es un software comercial y privativo de 16 bits, dise\~{n}ado para el sistema operativo Microsoft Windows versi\'{o}n 3.x, con la posibilidad de ejecutarse en Windows 95, Windows 2000 y Windows XP. Este software contiene funciones que abarcan la utilizaci\'{o}n del MiniScan XE Plus, y de otros instrumentos ofrecidos por HunterLab; adem\'{a}s fue descontinuado en el a\~{n}o 2008. La interfaz gr\'{a}fica de usuario de este software est\'{a} en ingl\'{e}s. Por \'{u}ltimo, los resultados que genera este software no poseen el formato de gesti\'{o}n de informaci\'{o}n de pacientes con el que trabajan los dermat\'{o}logos del CIMBUC.

Tomando en cuenta lo mencionado anteriormente, se tiene que el HLUS es un software privativo y que est\'{a} descontinuado, por lo tanto no existe la posibilidad de modificarlo ni extenderlo; ofrece funciones ajenas al uso exclusivo del MiniScan XE Plus, causando que la interfaz gr\'{a}fica de usuario contenga m\'{a}s opciones de las necesarias para manejar tal instrumento. Asimismo, como consecuencia de que la interfaz gr\'{a}fica de usuario est\'{e} en ingl\'{e}s, esta es dif\'{i}cil de entender por los dermat\'{o}logos. Sumado al hecho de que los resultados generados por este software no poseen el formato con el que trabajan los dermat\'{o}logos, haciendo necesario su traspaso manual, lo que produce a una ralentizaci\'{o}n en las consultas con pacientes. Todo esto conlleva a que los dermat\'{o}logos requieran de asistencia t\'{e}cnica especializada para la debida utilizaci\'{o}n de dicho software.

De lo antedicho se desprende que, el HLUS posee una interfaz gr\'{a}fica de usuario poco amigable, y el costo del tiempo de capacitaci\'{o}n para su uso correcto podr\'{i}a ser alto. Este software no podr\'{a} modificarse ni extenderse, por lo tanto no se fomentar\'{a} el uso del instrumento en cuesti\'{o}n en el campo m\'{e}dico (p\'{u}blico o privado), disminuyendo su potencial. Por \'{u}ltimo, tampoco se fomentar\'{a} el desarrollo de nuevas aplicaciones que utilicen sus resultados como insumo, sosegando as\'{i} la posibilidad de realizar an\'{a}lisis m\'{a}s complejos, y de proveer a los dermat\'{o}logos con resultados que les permitan establecer diagn\'{o}sticos m\'{a}s completos.

	\section{Justificaci\'{o}n}

Con respecto a software de calidad, \cite{Sommerville} explica lo siguiente: as\'{i} como los servicios que proveen, los productos de software tienen cierto n\'{u}mero de atributos asociados que reflejan su calidad. Estos atributos no est\'{a}n directamente relacionados con lo que hace el software. M\'{a}s bien, reflejan su comportamiento durante su ejecuci\'{o}n, en la estructura y organizaci\'{o}n del programa fuente, y en la documentaci\'{o}n asociada.

El conjunto espec\'{i}fico de atributos que se espera de un software depende obviamente de su aplicaci\'{o}n. Esto se generaliza en el conjunto de atributos que se muestran en la Tabla 1, la cual contiene las caracter\'{i}sticas esenciales de un software de calidad.

		\begin{table}[htb]
			\caption{\textbf{Tabla 1.} \textit{Atributos esenciales de un buen software} (Fuente: Sommerville, 2005).}
			\label{tabla_1}
			\centering
			\setlength{\extrarowheight}{3.5pt}
			\begin{tabulary}{15cm}{|c|J|}
				\hline
				\thead{\textbf{Caracter\'{i}stica}} & \thead{\textbf{Descripci\'{o}n}}\\ \hline
			\textbf{Mantenibilidad} & El software debe describirse de tal forma que pueda evolucionar  para cumplir las necesidades de cambio de los 					clientes. Este es un atributo cr\'{i}tico, debido a que el cambio en el software es una consecuencia inevitable de un cambio en el entorno de negocios.\\ \hline
			\textbf{Confiabilidad} & La confiabilidad del software tiene un gran n\'{u}mero de caracter\'{i}sticas, incluyendo la fiabilidad, protecci\'{o}n y seguridad. El software confiable no debe causar da\~{n}os f\'{i}sicos o econ\'{o}micos en el caso de una falla del sistema.\\ \hline
			\textbf{Eficiencia} & El software no debe hacer que se malgasten los recursos del sistema, como la memoria y los ciclos de procesamiento. Por lo tanto, la eficiencia incluye tiempos de respuesta y de procesamiento, utilizaci\'{o}n de la memoria, etc\'{e}tera.\\ \hline
			\textbf{Usabilidad} & El software debe ser f\'{a}cil de utilizar, sin esfuerzo adicional por el usuario para quien est\'{a} dise\~{n}ado. Esto significa que debe tener una interfaz gr\'{a}fica de usuario apropiada y una documentaci\'{o}n adecuada.\\ \hline
			\end{tabulary}
		\end{table}
		\FloatBarrier %you shall not pass table!!

Debido a que el HLUS es privativo, el CIMBUC no dispone de su c\'{o}digo fuente, de manera que este software no puede ser cambiado ni adaptarse a necesidades espec\'{i}ficas, y, por lo tanto, no posee el primer atributo esencial para un software de calidad: la mantenibilidad. Por la misma raz\'{o}n, no se puede determinar con certidumbre el segundo atributo: la confiabilidad (madurez del software y tolerancia a fallas); adem\'{a}s de que no se puede evaluar completamente el nivel de protecci\'{o}n y seguridad del mismo. Por \'{u}ltimo, la usabilidad de este software es baja, ya que la interfaz gr\'{a}fica de usuario es poco amigable. Por estas razones, se desarroll\'{o} un software que cumpliese con los atributos esenciales que debe poseer un buen software.

\cite{Sommerville} se\~{n}ala que un dise\~{n}o cuidadoso de la interfaz gr\'{a}fica de usuario es parte fundamental del proceso de dise\~{n}o general del software. Si un software debe alcanzar su potencial m\'{a}ximo, es fundamental que su interfaz gr\'{a}fica de usuario sea dise\~{n}ada para ajustarse a las habilidades, experiencia y expectativas de sus usuarios previstos. Un buen dise\~{n}o de la interfaz gr\'{a}fica de usuario es cr\'{i}tico para la confiabilidad del software. Muchos de los llamados ``errores de usuario'' son causados por el hecho de que las interfaces gr\'{a}ficas de usuario no consideran las habilidades de los usuarios reales y su entorno de trabajo.

El dise\~{n}o de la interfaz gr\'{a}fica de usuario del HLUS es la principal raz\'{o}n por la cual los dermat\'{o}logos requieren de personal t\'{e}cnico que los asista al momento de utilizarlo. Esto porque dicha interfaz est\'{a} en idioma ingl\'{e}s, contiene funcionalidades que no son necesarias para la utilizaci\'{o}n de Espectrofot\'{o}metro y no proporciona el formato con el que trabajan los dermat\'{o}logos, lo que dificulta la utilizaci\'{o}n de dicha interfaz. Por estas razones los dermat\'{o}logos perciben este software comercial como no intuitivo, ni auto descriptivo ni amigable, temiendo cometer errores al utilizarlo por su propia cuenta y generar resultados err\'{o}neos, poniendo en riesgo el diagn\'{o}stico, y en consecuencia, la salud de los pacientes en consulta.

En conclusi\'{o}n, siguiendo los lineamientos de dise\~{n}o y calidad del software que se consideraron pertinentes, se desarroll\'{o} un software amigable, modificable y extensible, el cual ofrece las funciones que necesitan los dermat\'{o}logos para establecer diagn\'{o}sticos, emplea el formato de historia m\'{e}dica con el que trabajan, permite la exportaci\'{o}n de los resultados a un formato de archivo portable; por \'{u}ltimo y no menos importante, se cre\'{o} una base sobre la cual se prodr\'{a}n trabajar proyectos futuros que necesiten utilizar los resultados de este software como insumo.
	\newpage

	\section{Objetivos de la Investigaci\'{o}n}
En la siguiente secci\'{o}n se especifican los objetivos del trabajo, distinguiendo entre el objetivo general y los objetivos espec\'{i}ficos.
		\subsection{Objetivo General}
	Desarrollar un software para el Espectrofot\'{o}metro ``MiniScan XE Plus'', usado en el diagn\'{o}stico de patolog\'{i}as dermatol\'{o}gicas en pacientes, tomando como caso de estudio el CIMBUC.
		\subsection{Objetivos Espec\'{i}ficos}
			\begin{itemize}
				\item Investigar el estado del arte referente a las caracter\'{i}sticas de software para Espectrofot\'{o}metros de reflexi\'{o}n difusa, dise\~{n}o y calidad de software.
				\item Seleccionar una metodolog\'{i}a que gu\'{i}e el dise\~{n}o y desarrollo del nuevo software para el Espectrofot\'{o}metro ``MiniScan XE Plus''.
				\item Dise\~{n}ar el nuevo software siguiendo la metodolog\'{i}a seleccionada.
				\item Desarrollar el nuevo software, siguiendo la metodolog\'{i}a seleccionada.
				\item Dise\~{n}ar las pruebas para el nuevo software.
				\item Elaborar el manual de usuario del nuevo software.
			\end{itemize}