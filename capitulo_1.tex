\chapter{\label{cap:1}El Problema}

	\section{Planteamiento del Problema}	
\cite{Bersha} indica que durante el diagn\'{o}stico de enfermedades de la piel, la observaci\'{o}n cuidadosa y la evaluaci\'{o}n visual del \'{a}rea sospechada es siempre el primer paso y el m\'{a}s importante. Esto es seguido generalmente por una escisi\'{o}n o biopsia por punci\'{o}n, en la que se extrae una muestra de tejido de la piel para un an\'{a}lisis microsc\'{o}pico. La observaci\'{o}n visual suele ser subjetiva y los pacientes a menudo se someten a cicatrices y dolor durante la escisi\'{o}n. Por otro lado, las t\'{e}cnicas \'{o}pticas son por lo general no invasivas y los resultados de estas son a menudo objetivos. Durante el diagn\'{o}stico no invasivo no se crea ninguna ruptura en la piel, y los pacientes no se someten al dolor y a cicatrices durante el tratamiento.

Los avances tecnol\'{o}gicos en la actualidad permiten emplear t\'{e}cnicas de \'{o}ptica con la capacidad de estudiar  las propiedades estructurales y bioqu\'{i}micas del tejido biol\'{o}gico de manera precisa y no invasiva. Los instrumentos que emplean tales t\'{e}cnicas son de gran ayuda para los m\'{e}dicos dermat\'{o}logos, raz\'{o}n por la cual han tomado suma importancia en el \'{a}rea m\'{e}dica dermatol\'{o}gica.

Hoy d\'{i}a existen diferentes tipos de estudios \'{o}pticos in-situ, in-vivo e invitro del tejido biol\'{o}gico, como lo es la espectroscop\'{i}a de reflectancia difusa. Con esta t\'{e}cnica es  posible estudiar las propiedades bioqu\'{i}micas y las condiciones estructurales de un tejido biol\'{o}gico, analizando la interacci\'{o}n luz-tejido de una manera no invasiva \cite{Perez-Gallardo}.

En este sentido, el Centro de Investigaciones M\'{e}dicas y Biotecnol\'{o}gicas de la Universidad de Carabobo (CIMBUC) dispone de un Espectrofot\'{o}metro de reflexi\'{o}n difusa denominado ``MiniScan XE Plus''. La empresa ``HunterLab'', creadora y distribuidora del ``MiniScan XE Plus'', lo describe como un instrumento utilizado para medir la transmisi\'{o}n y/o reflectancia de espec\'{i}menes, como una funci\'{o}n de longitud de onda, que aplica una t\'{e}cnica llamada espectroscop\'{i}a de reflectancia difusa. 

Ahora bien, para emplear el uso del instrumento en estudio, el CIMBUC ha tenido que utilizar el software comercial disponible para la utilizaci\'{o}n del mismo, denominado ``HunterLab Universal Software'', el cual es un software propietario de 16-bit dise\~{n}ado para el Sistema Operativo Microsoft Windows Version 3.x, con la posibilidad de ejecutarse en Windows 95, Windows 2000 y Windows NT, y el mismo fue descontinuado en el a\~{n}o 2008. Este software ofrece un conjunto de funcionalidades que abarcan no s\'{o}lo la utilizaci\'{o}n del Espectrofot\'{o}metro, sino tambi\'{e}n la utilizaci\'{o}n de otros instrumentos ofrecidos por la empresa ``HunterLab''. La interfaz gr\'{a}fica de usuario de dicho software esta en idioma ingl\'{e}s. Por \'{u}ltimo, los resultados que genera este software no poseen el formato de gesti\'{o}n de informaci\'{o}n de pacientes con el que trabajan los dermat\'{o}logos del CIMBUC.

Tomando en cuenta lo mencionado anteriormente se tiene que el software comercial es propietario y est\'{a} descontinuado, por lo tanto no existe la posibilidad de modificarlo, mejorarlo ni extenderlo; ofrece funcionalidades ajenas al uso exclusivo del Espectrofot\'{o}metro, causando que la interfaz gr\'{a}fica de usuario contenga m\'{a}s opciones disponibles de las necesarias para manejar el instrumento en estudio. Asimismo, como consecuencia de que la interfaz gr\'{a}fica de usuario est\'{e} en idioma ingl\'{e}s, \'{e}sta es dif\'{i}cil de entender por los dermat\'{o}logos. Aunado al hecho de que los resultados generados por dicho software no poseen el formato con el que trabajan los dermat\'{o}logos, haciendo necesario su traspaso manual, lo que produce a una ralentizaci\'{o}n en las consultas con pacientes. Todo esto conlleva a que los m\'{e}dicos requieran de asistencia t\'{e}cnica entrenada, disponible en todo momento para guiar el uso apropiado del software.

De lo antedicho se desprende que, el software comercial en utilizaci\'{o}n para el manejo del Espectrofot\'{o}metro posee una interfaz gr\'{a}fica de usuario poco amigable, y el costo del tiempo de capacitaci\'{o}n para su uso correcto podr\'{i}a ser alto. \'{E}ste software no podr\'{a} modificarse, mejorarse ni extenderse por el hecho de ser propietario, y por lo tanto no se fomentar\'{a} el uso del instrumento en cuesti\'{o}n en el campo m\'{e}dico (p\'{u}blico o privado). De igual manera, tampoco se fomentar\'{a} el desarrollo de nuevas aplicaciones que utilicen sus resultados como insumo, sosegando as\'{i} la posibilidad de realizar an\'{a}lisis m\'{a}s complejos y de proveer a los dermat\'{o}logos de resultados que les permitan establecer diagn\'{o}sticos m\'{a}s completos.

Motivado a todo lo anterior, se desarroll\'{o} un nuevo software para el Espectrofot\'{o}metro, con una interfaz gr\'{a}fica de usuario amigable, utilizando los lineamientos de la ingenier\'{i}a del software pertinentes y favoreciendo su integraci\'{o}n con nuevas aplicaciones que se desarrollen en proyectos futuros, logrando as\'{i} el nivel deseado de amigabilidad y extensibilidad.

Con esta investigaci\'{o}n se espera fomentar la utilizaci\'{o}n del nuevo software, una mejor capacitaci\'{o}n del personal m\'{e}dico para su debido uso y el aporte de una base s\'{o}lida sobre la cual se podr\'{a}n desarrollar nuevos proyectos.

	\section{Justificaci\'{o}n}
El estudio y diagn\'{o}stico de patolog\'{i}as dermatol\'{o}gicas en pacientes es un \'{a}rea cuyo campo est\'{a} en constante desarrollo, requiriendo que los procesos involucrados en \'{e}sta no solamente sean de calidad, sino que sean capaces de desarrollarse a la par; el software utilizado en dicha \'{a}rea no es una excepci\'{o}n. Que los dermat\'{o}logos experimenten dificultades al momento de utilizar el ``HunterLab Universal Software'' debido a que el mismo este en ingl\'{e}s, ofrezca funciones ajenas al instrumento en estudio, no emplee el formato de historia m\'{e}dica utilizado por ellos, y que adem\'{a}s no ofrezca la posibilidad de modificarlo, mejorarlo ni agregarle nuevas funciones, es un problema grave, ya que no s\'{o}lo ralentiza cada consulta con un paciente, sino que genera la necesidad de asistencia t\'{e}cnica disponible en todo momento para la debida utilizaci\'{o}n de dicho software; por \'{u}ltimo y no menos importante, disminuye el nivel de aprovechamiento potencial del instrumento de medici\'{o}n en estudio.

Con respecto a software de calidad, \cite{Sommerville} explica lo siguiente: As\'{i} como los servicios que proveen, los productos de software tienen cierto n\'{u}mero de atributos asociados que reflejan la calidad de ese software. Estos atributos no est\'{a}n directamente relacionados con lo que el software hace. M\'{a}s bien, reflejan su comportamiento durante su ejecuci\'{o}n, en la estructura y organizaci\'{o}n del programa fuente, y en la documentaci\'{o}n asociada. Ejemplos de estos atributos son el tiempo de respuesta del software a una pregunta del usuario y la comprensi\'{o}n del programa fuente.

El conjunto espec\'{i}fico de atributos que se espera de un software depende obviamente de su aplicaci\'{o}n. Esto se generaliza en el conjunto de atributos que se muestran en la Tabla 1, la cual contiene las caracter\'{i}sticas esenciales de un software bien dise\~{n}ado.

	\begin{table}[htb]
		\small
		\centering
		\setlength{\extrarowheight}{5pt}
		\begin{tabulary}{15cm}{|c|L|}
			\hline
			\textbf{Caracter\'{i}stica} & \textbf{Descripci\'{o}n}\\ \hline
			\textbf{Mantenibilidad} & El software debe describirse de tal forma que pueda evolucionar  para cumplir las necesidades de cambio de los 					clientes. Este es un atributo cr\'{i}tico, debido a que el cambio en el software es una consecuencia inevitable de un cambio en el entorno de negocios.\\ \hline
			\textbf{Confiabilidad} & La confiabilidad del software tiene un gran n\'{u}mero de caracter\'{i}sticas, incluyendo la fiabilidad, protecci\'{o}n y seguridad. El software confiable no debe causar da\~{n}os f\'{i}sicos o econ\'{o}micos en el caso de una falla del sistema.\\ \hline
			\textbf{Eficiencia} & El software no debe hacer que se malgasten los recursos del sistema, como la memoria y los ciclos de procesamiento. Por lo tanto, la eficiencia incluye tiempos de respuesta y de procesamiento, utilizaci\'{o}n de la memoria, etc\'{e}tera.\\ \hline
			\textbf{Usabilidad} & El software debe ser f\'{a}cil de utilizar, sin esfuerzo adicional por el usuario para quien est\'{a} dise\~{n}ado. Esto significa que debe tener una interfaz gr\'{a}fica de usuario apropiada y una documentaci\'{o}n adecuada.\\ \hline
		\end{tabulary}
			\caption{\textbf{Tabla 1.} \textit{Atributos esenciales de un buen software}		(Fuente: Sommerville, 2005).}
	\end{table}
			\FloatBarrier %you shall not pass table!!
Debido a que el ``HunterLab Universal Software'' es propietario, el CIMBUC no dispone del c\'{o}digo fuente del mismo, lo que se traduce en la inexistencia del primer atributo esencial para un buen software: la mantenibilidad; ya que el software propietario no puede ser cambiado ni adaptarse a necesidades espec\'{i}ficas. Por la misma raz\'{o}n de ser un software propietario del cual no se tiene el c\'{o}digo fuente, no se puede determinar con certidumbre el segundo atributo: la confiabilidad; debido a que no se puede evaluar completamente el nivel de protecci\'{o}n y seguridad existentes en dicho software. Por \'{u}ltimo y no menos importante, la usabilidad del software existente es baja, ya que la interfaz gr\'{a}fica de usuario es poco amigable, haciendo surgir la necesidad de disponer de personal t\'{e}cnico para la utilizaci\'{o}n correcta del mismo. Por estas razones, se desarroll\'{o} un software que cumpliese con los atributos esenciales que debe poseer un buen software.

\cite{Sommerville} se\~{n}ala que un dise\~{n}o cuidadoso de la interfaz gr\'{a}fica de usuario es parte fundamental del proceso de dise\~{n}o general del software. Si un software debe alcanzar su potencial m\'{a}ximo, es fundamental que su interfaz gr\'{a}fica de usuario sea dise\~{n}ada para ajustarse a las habilidades, experiencia y expectativas de sus usuarios previstos. Un buen dise\~{n}o de la interfaz gr\'{a}fica de usuario es cr\'{i}tico para la confiabilidad del software. Muchos de los llamados ``errores de usuario'' son causados por el hecho de que las interfaces gr\'{a}ficas de usuario no consideran las habilidades de los usuarios reales y su entorno de trabajo.

El dise\~{n}o de la interfaz gr\'{a}fica de usuario del ``HunterLab Universal Software'' es la principal raz\'{o}n por la cual los dermat\'{o}logos requieren de personal t\'{e}cnico que los asista al momento de utilizarlo. Esto porque dicha interfaz est\'{a} en idioma ingl\'{e}s, contiene funcionalidades que no son necesarias para la utilizaci\'{o}n de Espectrofot\'{o}metro y no proporciona el formato con el que trabajan los dermat\'{o}logos, lo que dificulta la utilizaci\'{o}n de dicha interfaz. Por estas razones los dermat\'{o}logos perciben este software comercial como no intuitivo, ni auto descriptivo ni amigable, temiendo cometer errores al utilizarlo por su propia cuenta y generar resultados err\'{o}neos, poniendo en riesgo el diagn\'{o}stico, y en consecuencia, la salud de los pacientes en consulta.

En conclusi\'{o}n, siguiendo los lineamientos de dise\~{n}o y calidad del software que se consideraron pertinentes, se desarroll\'{o} un software amigable, modificable y extensible, el cual ofrece las funciones que necesitan los dermat\'{o}logos para establecer diagn\'{o}sticos, emplea el formato de historia m\'{e}dica con el que trabajan, permite la exportaci\'{o}n de los resultados a un formato de archivo portable; por \'{u}ltimo y no menos importante, se cre\'{o} una base sobre la cual se prodr\'{a}n trabajar proyectos futuros que necesiten utilizar los resultados de este software como insumo.
	\newpage

	\section{Objetivos de la Investigaci\'{o}n}
En la siguiente secci\'{o}n se especifican los objetivos del trabajo, distinguiendo entre el objetivo general y los objetivos espec\'{i}ficos.
		\subsection{Objetivo General}
	Desarrollar un software para el Espectrofot\'{o}metro ``MiniScan XE Plus'', usado en el diagn\'{o}stico de patolog\'{i}as dermatol\'{o}gicas en pacientes, tomando como caso de estudio el CIMBUC.
		\subsection{Objetivos Espec\'{i}ficos}
			\begin{itemize}
				\item Investigar el estado del arte referente a las caracter\'{i}sticas de software para Espectrofot\'{o}metros de reflexi\'{o}n difusa, dise\~{n}o y calidad de software.
				\item Seleccionar una metodolog\'{i}a que gu\'{i}e el dise\~{n}o y desarrollo del nuevo software para el Espectrofot\'{o}metro ``MiniScan XE Plus''.
				\item Dise\~{n}ar el nuevo software siguiendo la metodolog\'{i}a seleccionada.
				\item Desarrollar el nuevo software, siguiendo la metodolog\'{i}a seleccionada.
				\item Dise\~{n}ar las pruebas para el nuevo software.
				\item Elaborar el manual de usuario del nuevo software.
			\end{itemize}