\chapter{El Problema}

En este cap\'{i}tulo se presenta la problem\'{a}tica que motiva al desarrollo de este trabajo de investigac\'{i}on, as\'{i} como tambi\'{e}n la importancia que tiene para el \'{a}mbito m\'{e}dico dermatol\'{o}gico, y los objetivos a lograr con la investigaci\'{o}n.

	\section{Planteamiento del problema}	
El color y la apariencia de la piel humana es importante en el campo de la medicina. Durante el diagn\'{o}stico de enfermedades de la piel, la observaci\'{o}n cuidadosa y la evaluaci\'{o}n visual del \'{a}rea sospechada es siempre el primer paso, y el m\'{a}s importante. Esto es seguido generalmente por una escisi\'{o}n o biopsia por punci\'{o}n, en la que se extrae una muestra de tejido de la piel para un an\'{a}lisis microsc\'{o}pico. La observaci\'{o}n visual suele ser subjetiva, y los pacientes a menudo se someten a cicatrices y dolor durante la biopsia. Por otro lado, las t\'{e}cnicas \'{o}pticas son por lo general no invasivas, y sus resultados son a menudo objetivos. Durante el diagn\'{o}stico no invasivo, no se crea ninguna ruptura en la piel, y los pacientes no se someten al dolor ni a cicatrices durante el tratamiento. \cite{Bersha}.

Los avances tecnol\'{o}gicos de la actualidad permiten emplear t\'{e}cnicas de \'{o}ptica, que tienen la capacidad de estudiar  las propiedades estructurales y bioqu\'{i}micas del tejido biol\'{o}gico, de manera precisa y no invasiva. Los instrumentos que emplean tales t\'{e}cnicas son de gran ayuda para los m\'{e}dicos dermat\'{o}logos, raz\'{o}n por la cual dichos instrumentos han tomado suma importancia en el \'{a}rea m\'{e}dica dermatol\'{o}gica.

Hoy en d\'{i}a existen diferentes tipos de estudios \'{o}pticos in-situ, in-vivo e invitro del tejido biol\'{o}gico, como la Espectroscop\'{i}a de Reflectancia Difusa (ERD). \cite{Perez} asegura que con esta t\'{e}cnica es  posible estudiar las propiedades bioqu\'{i}micas y las condiciones estructurales de un tejido biol\'{o}gico, analizando la interacci\'{o}n luz-tejido de una manera no invasiva.

En este sentido, el Centro de Investigaciones M\'{e}dicas y Biotecnol\'{o}gicas de la Universidad de Carabobo (CIMBUC) dispone de un espectrofot\'{o}metro de reflexi\'{o}n difusa, denominado MiniScan XE Plus, creado por la empresa HunterLab. Esta empresa lo describe como un instrumento utilizado para medir la transmisi\'{o}n y/o reflectancia de espec\'{i}menes, como una funci\'{o}n de longitud de onda, que aplica la t\'{e}cnica de ERD. 

Ahora bien, el CIMBUC hace uso de este instrumento a trav\'{e}s del software disponible para su utilizaci\'{o}n, designado HunterLab Universal Software, que es un software comercial y privativo de 16 bits, capaz de ejecutarse hasta en Windows XP. Este software contiene funciones que abarcan la utilizaci\'{o}n del MiniScan XE Plus, y de otros instrumentos ofrecidos por HunterLab; adem\'{a}s fue descontinuado en el a\~{n}o 2008. La interfaz gr\'{a}fica de usuario de este software est\'{a} en ingl\'{e}s. Por \'{u}ltimo, los resultados que genera este software no poseen el formato de gesti\'{o}n de informaci\'{o}n de pacientes con el que trabajan los dermat\'{o}logos del CIMBUC.

Tomando en cuenta lo mencionado anteriormente, se tiene que el HunterLab Universal Software es un software privativo y que est\'{a} descontinuado, por lo tanto no existe la posibilidad de modificarlo ni extenderlo; ofrece funciones ajenas al uso exclusivo del MiniScan XE Plus, causando que la interfaz gr\'{a}fica de usuario contenga m\'{a}s opciones de las necesarias para manejar tal instrumento. Asimismo, como consecuencia de que la interfaz gr\'{a}fica de usuario est\'{e} en ingl\'{e}s, esta es dif\'{i}cil de entender por los dermat\'{o}logos. Sumado al hecho de que los resultados generados por este software no poseen el formato con el que trabajan los dermat\'{o}logos, haciendo necesario su traspaso manual, lo que produce una ralentizaci\'{o}n en las consultas con los pacientes. Todo esto conlleva a que los dermat\'{o}logos requieran de asistencia t\'{e}cnica especializada para la debida utilizaci\'{o}n de dicho software.

De lo antedicho se desprende que, el HunterLab Universal Software posee una interfaz gr\'{a}fica de usuario poco amigable, y el costo del tiempo de capacitaci\'{o}n para su uso correcto podr\'{i}a ser alto. Dicho software no podr\'{a} modificarse ni extenderse, por lo tanto no se fomentar\'{a} el uso del instrumento en cuesti\'{o}n, disminuyendo su potencial. El formato de los resultados de este software convertir\'{a} a las consultas de los dermat\'{o}logos con los pacientes en una labor ineficiente en t\'{e}rminos de tiempo. Por \'{u}ltimo, no se fomentar\'{a} el desarrollo de nuevas funciones que utilicen sus resultados como insumo, sosegando as\'{i} la posibilidad de realizar an\'{a}lisis m\'{a}s complejos, y de proveer resultados que les permitan a los dermat\'{o}logos establecer diagn\'{o}sticos m\'{a}s completos de patolog\'{i}as en pacientes.

Motivado a lo anterior, se desarroll\'{o} un nuevo software modificable y extensible, con una interfaz gr\'{a}fica de usuario amigable, que emplea el uso del formato utilizado por los dermat\'{o}logos para registrar los resultados de las consultas, y con funciones para el uso exclusivo del MiniScan XE Plus, utilizando los lineamientos de la ingenier\'{i}a del software pertinentes.

Finalmente, con esta investigaci\'{o}n se espera fomentar la utilizaci\'{o}n del \mbox{MiniScan} XE Plus por medio del nuevo software, mejorar la capacitaci\'{o}n del personal m\'{e}dico dermatol\'{o}gico para su debido uso, reducir el tiempo de las consultas con los pacientes, y, por \'{u}ltimo, aportar una base s\'{o}lida sobre la cual se puedan desarrollar nuevos proyectos.

	\section{Justificaci\'{o}n de la investigaci\'{o}n}
	
	Empezando con la interfaz gr\'{a}fica de usuario, \cite{Sommerville} se\~{n}ala que el dise\~{n}o cuidadoso de la misma es una parte fundamental del proceso de dise\~{n}o general del software. Si un software debe alcanzar su potencial m\'{a}ximo, es fundamental que su interfaz gr\'{a}fica de usuario sea dise\~{n}ada para ajustarse a las habilidades, experiencia y expectativas de sus usuarios previstos. Un buen dise\~{n}o de la interfaz gr\'{a}fica de usuario es cr\'{i}tico para la confiabilidad del software. Muchos de los llamados errores de usuario son causados porque las interfaces gr\'{a}ficas de usuario no consideran las habilidades de los usuarios reales y su entorno de trabajo.

	Dicho lo anterior, el dise\~{n}o de la interfaz gr\'{a}fica de usuario del HunterLab Universal Software es la principal raz\'{o}n por la cual los dermat\'{o}logos requieren de personal t\'{e}cnico especializado, que los asista al momento de utilizarlo. Esto porque dicha interfaz est\'{a} en ingl\'{e}s, ofrece funciones que no son necesarias para la utilizaci\'{o}n del MiniScan XE Plus, y sus resultados no proporcionan el formato con el que trabajan. Por estas razones los dermat\'{o}logos perciben este software comercial como no intuitivo, ni auto descriptivo ni amigable, temiendo cometer errores al utilizarlo por su propia cuenta y generar resultados err\'{o}neos, poniendo en riesgo la fiabilidad del diagn\'{o}stico, y, en consecuencia, la salud de los pacientes en consulta.

	Con respecto a software de calidad, \cite{Sommerville} explica lo siguiente: as\'{i} como los servicios que proveen, los productos de software tienen un cierto n\'{u}mero de atributos asociados que reflejan su calidad. Estos atributos no est\'{a}n directamente relacionados con lo que hace el software; m\'{a}s bien, reflejan su comportamiento durante su ejecuci\'{o}n, en la estructura y la organizaci\'{o}n del programa fuente, y en la documentaci\'{o}n asociada.

El conjunto espec\'{i}fico de atributos que se esperan de un software de calidad depende obviamente de su aplicaci\'{o}n. Esto se generaliza en el conjunto de atributos que se muestran en la Tabla 1, en la cual se pueden apreciar las caracter\'{i}sticas esenciales de un software de calidad.

\FloatBarrier %you shall not pass table!!
		\begin{table}[htb]
			\small
			\caption{\textbf{Tabla 1.} \textit{Atributos esenciales de un software de calidad} (Fuente: Sommerville, 2005).}
			\centering
			\setlength{\extrarowheight}{\altocelda}
			\begin{tabulary}{\anchotabla}{|c|J|}
				
				\hline
				\thead{\textbf{\small{Caracter\'{i}stica}}} & \thead{\textbf{\small{Descripci\'{o}n}}}\\ \hline
			\textbf{Mantenibilidad} & El software debe describirse de tal forma que pueda evolucionar  para cumplir las necesidades de cambio de los 					clientes. Este es un atributo cr\'{i}tico, debido a que el cambio en el software es una consecuencia inevitable de un cambio en el entorno de negocios.\\ \hline
			\textbf{Confiabilidad} & Este atributo tiene un gran n\'{u}mero de caracter\'{i}sticas, incluyendo la fiabilidad, la protecci\'{o}n y la seguridad. El software confiable no debe causar da\~{n}os f\'{i}sicos o econ\'{o}micos en caso de que ocurra una falla del sistema.\\ \hline
			\textbf{Eficiencia} & El software no debe hacer que se malgasten los recursos del sistema, como la memoria y los ciclos de procesamiento. Por lo tanto, la eficiencia incluye tiempos de respuesta y de procesamiento, utilizaci\'{o}n de la memoria, etc\'{e}tera.\\ \hline
			\textbf{Usabilidad} & El software debe ser f\'{a}cil de utilizar, sin esfuerzo adicional por parte del usuario para quien est\'{a} dise\~{n}ado. Esto significa que debe tener una interfaz gr\'{a}fica de usuario apropiada, y una documentaci\'{o}n adecuada.\\ \hline
			\end{tabulary}
		\end{table}
\FloatBarrier %you shall not pass table!!

Debido a que el HunterLab Universal Software es privativo, el CIMBUC no dispone de su c\'{o}digo fuente, de manera que este software no puede modificarse ni adaptarse a necesidades espec\'{i}ficas, y, por lo tanto, no posee el primer atributo esencial para un software de calidad: la mantenibilidad. Por la misma raz\'{o}n, no se puede determinar con certidumbre el segundo atributo: la confiabilidad (madurez del software y tolerancia a fallas); adem\'{a}s de que no se puede evaluar completamente el nivel de protecci\'{o}n y seguridad del mismo. Por \'{u}ltimo, la usabilidad de este software es baja, ya que la interfaz gr\'{a}fica de usuario es poco amigable.

Las razones descritas anteriormente justifican la necesidad del desarrollo de un nuevo software para el uso del MiniScan XE Plus, que cumpla con los atributos esenciales de calidad, que sea amigable, modificable y extensible; que ofrezca las funciones que necesitan los dermat\'{o}logos para establecer diagn\'{o}sticos de patolog\'{i}as en pacientes, y que emplee el uso del formato de historia m\'{e}dica con el que trabajan. Por \'{u}ltimo, se ha tomado como caso de estudio el CIMBUC.

	\section{Objetivos}

		\subsection{Objetivo general}
	Desarrollar un software para el espectrofot\'{o}metro MiniScan XE Plus, usado en el diagn\'{o}stico de patolog\'{i}as dermatol\'{o}gicas en pacientes, tomando como caso de estudio el CIMBUC.
		\subsection{Objetivos espec\'{i}ficos}
			\begin{itemize}
				\item Investigar el estado del arte referente a las caracter\'{i}sticas de software para espectrofot\'{o}metros de reflexi\'{o}n difusa, el dise\~{n}o y la calidad del software.
				\item Seleccionar una metodolog\'{i}a de investigaci\'{o}n y una metodolog\'{i}a de desarrollo para el nuevo software.
				\item Dise\~{n}ar  y desarrollar el nuevo software, siguiendo las metodolog\'{i}as \mbox{seleccionadas}.
				\item Dise\~{n}ar y realizar las pruebas para el nuevo software.
				\item Elaborar el manual de usuario para el uso del nuevo software.
			\end{itemize}